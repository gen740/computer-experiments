\section{講義・実習の概要}

\begin{frame}[t]{講義・実習の目的}
  \begin{itemize}
    %\setlength{\itemsep}{1em}
  \item 理論・実験を問わず、学部〜大学院〜で必要となる現代的かつ普遍的な計算機の素養を身につける
  \item UNIX環境に慣れる(シェル、ファイル操作、エディタ)
  \item ネットワークの活用 (リモートログイン、共同作業)
  \item プログラムの作成(C言語、コンパイラ、プログラム実行)
  \item 基本的な数値計算アルゴリズム・数値計算の常識を学ぶ
  \item 科学技術文書作成に慣れる(\LaTeX, グラフ作成)
  \end{itemize}
\end{frame}

\begin{frame}[t]{身に付けて欲しいこと}
  \begin{itemize}
    %\setlength{\itemsep}{1em}
  \item ツールとしてないものは自分で作る (物理の伝統)
  \item すでにあるものは積極的に再利用する (車輪の再発明をしない)
  \item 数学公式と数値計算アルゴリズムは別物
  \item 刻み幅・近似度合いを変えて何度か計算を行う
  \item グラフ化して目で見てみる
  \item 計算量(コスト)のスケーリング(次数)に気をつける
  \item 記録に残す・再現性を確保する
  \end{itemize}
\end{frame}

\begin{frame}[t]{講義・実習内容}
  \begin{itemize}
    % \setlength{\itemsep}{1em}
  \item UNIX操作・ネットワーク
  \item プログラミング: C言語、数値計算ライブラリの利用
  \item ツール: エディタ、コンパイラ、\LaTeX、Gnuplot
  \item 数値計算の基礎
  \item 常微分方程式の解法
  \item 連立一次方程式の解法
  \item 行列の対角化
  \item 線形回帰
  \end{itemize}
\end{frame}

\begin{frame}[t,fragile]{講義と実習}
  \begin{itemize}
    %\setlength{\itemsep}{1em}
  \item 講義
    \begin{itemize}
    \item ツールなどの基礎知識: 実習の準備
    \item アルゴリズムの基本概念: 実習の準備
    \item より高度なアルゴリズムの紹介
    \end{itemize}
  \item 実習 (2グループに分けて)
    \begin{itemize}
    \item 練習課題: 各自で取り組む
    \item レポート課題: 個別レポートとして提出
    \item 応用課題: 余裕のある人は積極的に取り組む
    \end{itemize}
  \item スタッフ \href{mailto:computer@exa.phys.s.u-tokyo.ac.jp}{computer@exa.phys.s.u-tokyo.ac.jp}
    \begin{itemize}
    \item 講義: 藤堂
    \item 実習: 鈴木(元早野研)、斉藤(古澤研)
    \item 実習TA: 鈴木(藤堂研M2)、堤(古澤研M1)
    \end{itemize}
  \end{itemize}    
\end{frame}

\begin{frame}[t,fragile]{評価方法・レポート}
  \begin{itemize}
    %\setlength{\itemsep}{1em}
  \item 評価
    \begin{itemize}
    \item 出席(講義・実習)
    \item レポート
    \end{itemize}    
  \item レポート
    \begin{itemize}
    \item 各自が \LaTeX で作成の上提出 (計2回)
    \item 提出方法・締切については、後日指示
    \end{itemize}
  \end{itemize}    
\end{frame}

\begin{frame}[t]{講義資料}
  \begin{itemize}
    \setlength{\itemsep}{1em}
  \item 「計算機実験」ハンドブック (配布済)
    \begin{itemize}
    \item UNIX入門
    \item C言語入門
    \item \LaTeX 入門
    \item バージョン管理システム
    \end{itemize}
  \item 講義資料、実習資料、追加資料、参考書
    \begin{itemize}
    \item ITC-LMSで配布・提示 \url{https://itc-lms.ecc.u-tokyo.ac.jp/lms/course/view.php?id=131372}
    \item 公開資料については、\url{http://exa.phys.s.u-tokyo.ac.jp/ja/lectures/2018s-computer1}でも配布
    \item 主な資料の \LaTeX ソースコードもGitHubで公開 \\
    {\footnotesize \url{https://github.com/todo-group/computer-experiments}}
    \end{itemize}
  \end{itemize}
\end{frame}

\begin{frame}[t]{質問がある場合には、、、}
  \begin{enumerate}
    %\setlength{\itemsep}{1em}
  \item ITC-LMS の掲示板を見る
  \item ハンドブック、講義資料を確認
  \item まわりの人に質問してみる
  \item ネットで検索
  \item 計算機実験担当者(\href{mailto:computer@exa.phys.s.u-tokyo.ac.jp}{computer@exa.phys.s.u-tokyo.ac.jp}) に相談
  \end{enumerate}
  メールで質問するときに注意すべきこと
  \begin{itemize}
  \item (メールの)標題をきちんとつける、きちんと名乗る
  \item 実行環境を明示する
  \item 問題を再現する手順を明記する
  \item 関連するファイル(Cや \LaTeX のソースコード等)を添付する
  \item エラーメッセージを添付する
  \end{itemize}
\end{frame}

\begin{frame}[t,fragile]{実習環境}
  \begin{itemize}
    \setlength{\itemsep}{1em}
  \item 情報基盤センター大演習室 (iMac端末)
    \begin{itemize}
    \item Cプログラミング、\LaTeX、Gnuplotなどに利用
    \end{itemize}
  \item 計算機端末室
    \begin{itemize}
    \item 理学部4号館1215室 (iMac 16台)
    \end{itemize}
  \item 物理学教室ワークステーションクラスタ photon
    \begin{itemize}
    \item SSHでリモートログインして使用する(ハンドブック2.2節)
    \item あらかじめ公開鍵の登録が必要(実習EX0 準備練習2,3)
    \end{itemize}
  \item MateriApps LIVE! (USBメモリで配布)
    \begin{itemize}
    \item Mac, Windows PC 上で動作する仮想UNIX環境
    \item Cコンパイラ、\LaTeX 環境、Gnuplot、エディタ、SSHクライアントなど一式揃っている
    \item インストール方法はUSBメモリ内の\href{https://github.com/cmsi/MateriAppsLive/wiki/MateriAppsLive-ltx}{README.html}、\href{https://github.com/cmsi/MateriAppsLive-setup/blob/master/ova/setup.pdf}{setup.pdf}を参照のこと
    \end{itemize}
  \end{itemize}
\end{frame}
