\begin{frame}[t,fragile]{様々な分布}
  \begin{itemize}
    %\setlength{\itemsep}{1em}
  \item 乱数発生器は通常、一様な整数乱数あるいは実数乱数を生成
  \item 一様分布以外の分布にしたがう乱数の発生方法の代表例
  \item 逆関数法
    \begin{itemize}
      \item 確率分布関数$F(x)$の逆関数$F^{-1}(y)$ と(0,1)の一様乱数$u$から $v=F^{-1}(u)$
      \item 例: 指数分布 $p(x) = \frac{1}{\mu} e^{-x/\mu}$

      $F(x) = 1 - e^{-x/\mu}$ \ \ \ $F^{-1}(y) = - \mu \log(1-y)$
      \item 一般の確率分布関数について逆関数を求めるのは困難
    \end{itemize}
  \item 棄却法
    \begin{itemize}
      \item 確率密度関数を完全に囲むような箱を用意し、その箱の中で一様乱数を生成
      \item 確率密度関数の下側の点が生成されたら、その$x$座標を乱数として採用。上側の点の場合には再度生成
      \item もとの確率密度関数よりも箱が大きくなりすぎると非効率
    \end{itemize}
  \end{itemize}
\end{frame}
