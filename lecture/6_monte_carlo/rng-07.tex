\begin{frame}[t,fragile]{疑似乱数とは}
  \begin{itemize}
    %\setlength{\itemsep}{1em}
  \item 計算機でプログラムに従って生成する乱数(のようなもの)
  \item 乱数は何に役立つか?
    \begin{itemize}
    \item 等式のチェック、例外の発見
    \item 初期値にランダムネスを入れることで最悪の場合を避ける
    \item サンプリングを使ったシミュレーション (→計算機実験II)
    \end{itemize}
  \item 乱数を使う場合の注意
    \begin{itemize}
    \item 計算式に従って生成するため周期は有限であり、必ず何らかの相関がある
    \item 初期化(種の設定)を正しく行う
    \item 実際にそれらしい乱数が生成されているか目で見て確認する
    \end{itemize}
  \item 代表的な乱数発生器のひとつ: メルセンヌ・ツイスター
    \begin{itemize}
    \item 周期 $2^{19937}-1$、高速、日本製!
    \item ヘッダファイル: \href{https://github.com/todo-group/computer-experiments/blob/master/exercise/monte_carlo/mersenne_twister.h}{mersenne\_twister.h}
    \item サンプルプログラム: \href{https://github.com/todo-group/computer-experiments/blob/master/exercise/monte_carlo/random.c}{random.c}
    \end{itemize}
  \end{itemize}
\end{frame}
