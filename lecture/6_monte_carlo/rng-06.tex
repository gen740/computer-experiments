%-*- coding:utf-8 -*-

\begin{frame}[t,fragile]{Box-Muller法}
  \begin{itemize}
    %\setlength{\itemsep}{1em}
  \item 一様分布乱数から正規分布乱数を生成する方法
    \begin{itemize}
    \item 2次元の(標準)ガウス分布を考える
      \[
      f(x,y)\,dx\,dy= \frac{1}{2\pi} e^{-(x^2+y^2)/2} \,dx\,dy
      \]
    \item 極座標$(r,\theta)$に変換 ($x=r\cos\theta$, $y=r\sin\theta$)
      \[
      \frac{1}{2\pi} e^{-(x^2+y^2)/2} \,dx\,dy = \frac{1}{2\pi} r \, e^{-r^2/2} \,dr\,d\theta
      \]
    \item $\theta$は$(0,2\pi)$の一様分布
    \item $r$は$f(r) = r \, e^{-r^2/2}$に従う
      \begin{align*}
        F(r) = \int_0^r f(r) \, dr = 1 - e^{-r^2/2}, \qquad r = F^{-1}(q) = \sqrt{- 2 \log(1-q)}
      \end{align*}
    \item 二つの一様乱数から二つの独立な正規分布乱数が生成される
    \end{itemize}
  \end{itemize}
\end{frame}
