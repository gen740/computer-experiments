%-*- coding:utf-8 -*-

\begin{frame}[t,fragile]{境界条件と力の計算}
  \begin{itemize}
    \setlength{\itemsep}{1em}
  \item 気体・液体・固体などの熱力学的極限を調べたい場合
    \begin{itemize}
    \item 端の効果を取り除くために周期境界条件を採用
    \item 周期的に同じパターンが続く
    \end{itemize}
  \item ポテンシャルの計算
    \begin{align*}
      U = \frac{1}{2} \sum_i \sum_{j \ne i} \sum_{n_x=-\infty}^{\infty} \sum_{n_y=-\infty}^{\infty} \sum_{n_z=-\infty}^{\infty} U[\mathbf{r}_i - \mathbf{r}_j + L(n_x,n_y,n_z)]
    \end{align*}
  \item 短距離力
    \begin{itemize}
    \item カットオフを入れる
    \item 最も近いイメージだけ考慮(minimum image convention)
    \end{itemize}
  \item 長距離力
    \begin{itemize}
    \item カットオフを入れると物理が変わる
    \item エバルト法、ツリー法、高速多重極展開
    \end{itemize}
  \end{itemize}
\end{frame}
