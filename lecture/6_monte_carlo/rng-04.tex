\begin{frame}[t,fragile]{乱数発生器の選び方}
  \begin{itemize}
    %\setlength{\itemsep}{1em}
  \item 疑似乱数を使う場合の注意
    \begin{itemize}
    \item 万能乱数発生器は存在しない
    \item 計算式に従って生成するため、必ず周期は有限、何らかの相関あり
    \item 乱数の性質について、数学的に厳密な証明やテスト結果があるが、特定のシミュレーションに使う場合について何も保証してくれない
    \item 自分で乱数発生器を「発明」してはいけない
    \item 自分で乱数発生器をプログラムしてはいけない(既存のライブラリを使う)
    \item 初期化(種・シードの設定)を正しく行う
    \item 実際にそれらしい乱数が生成されているか、グラフに描いて確認する
    \item 二種類以上の乱数発生器を使ってみて、結果が一致するか確認
    \end{itemize}
  \item 代表的な乱数発生器のひとつ: メルセンヌ・ツイスター
    \begin{itemize}
    \item 周期 $2^{19937}-1$、高速、日本製!
    \item ヘッダファイル: \href{https://github.com/todo-group/computer-experiments/blob/master/exercise/monte_carlo/mersenne_twister.h}{mersenne\_twister.h}
    \item サンプルプログラム: \href{https://github.com/todo-group/computer-experiments/blob/master/exercise/monte_carlo/random.c}{random.c}
    \end{itemize}
  \end{itemize}
\end{frame}
