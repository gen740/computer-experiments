%-*- coding:utf-8 -*-

\begin{frame}[t,fragile]{ツリー法}
  \begin{itemize}
    %\setlength{\itemsep}{1em}
  \item ある粒子から見て
    \begin{itemize}
    \item 見込み角があるしきい値よりも小さい場合は、その重心を使って計算
    \item そうでない場合にはまじめに計算
    \end{itemize}
  \item 計算量 $O(N \log N)$
  \item 近似精度を上げるには
    \begin{itemize}
    \item 見込み角のしきい値を小さくする→計算量が急速に増える
    \item 重心だけでなく多重極展開も使う

      原点中心半径$a$の球の中の粒子の作るポテンシャルの多重極展開係数
      \[
      \alpha_\ell^m = \sum_{i=1}^{N} m_i \Big( \frac{r_i}{a} \Big)^\ell Y_\ell^{-m}(\theta_i, \phi_i)
      \]
      $Y_\ell^m(\theta,\phi)$: $\ell$次の球面調和関数

      位置$(r,\theta,\phi)$におけるポテンシャル($r>a$)
      \[
      \Phi(r,\theta,\phi) \approx \sum_\ell \sum_{m=-\ell}^{\ell} \alpha_\ell^m \frac{a^\ell}{r^{\ell+1}} Y_\ell^m(\theta,\phi)
      \]
      
    \end{itemize}
  \end{itemize}
\end{frame}

