%-*- coding:utf-8 -*-

\begin{frame}[t,fragile]{厳密な数え上げ}
  \begin{itemize}
    %\setlength{\itemsep}{1em}
  \item 全ての状態についてボルツマン重みを計算し足し合わせる
    \begin{itemize}
    \item 状態数 $= 2^N$ \ ($N$スピン数)
    \end{itemize}
  \item $N$重のforループを書く代わりに、状態を{\color{red}$N$ビットの整数}($s=0,\cdots,2^N-1$)で表し一つのループに
    \begin{itemize}
    \item $i$番目($i=0,\cdots,N-1$)の格子点のスピンの状態を$i$ビット目に保存
    \item $\sigma = \pm 1$をビットの 1 or 0で表現
    \item シフト演算(\verb+>>+)とAND演算(\verb+&+)でスピン状態を取り出す

      例: 4番目($i=3$)のビットを取り出す: \verb+(s>>3)&1+

      例: $\sigma_i \sigma_j$の計算: \verb+(2.0*((s>>i)&1)-1)*(2.0*((s>>j)&1)-1)+

    \item 論理AND (\verb+&&+)とビットAND (\verb+&+)の違いに注意
    \end{itemize}
  \item 格子構造(一次元鎖、二次元正方格子等)を表す関数を用意する必要あり

    例:  \href{https://github.com/todo-group/computer-experiments/blob/master/exercise/monte_carlo/square_lattice.c}{square\_lattice.c}
  \end{itemize}
\end{frame}
