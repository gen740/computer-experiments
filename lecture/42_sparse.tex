\section{疎行列に対する反復法}

\begin{frame}[t,fragile]{反復法}
  \begin{itemize}
    \setlength{\itemsep}{1em}
  \item 疎行列の場合、行列ベクトル積は高速に行える
  \item Givens変換、Householder変換などを行うと疎行列性が失われる
  \item 行列ベクトル積のみを用いる反復法が効果的
    \begin{itemize}
    \item べき乗法
    \item Lanczos法
    \end{itemize}
  \end{itemize}
\end{frame}

\begin{frame}[t,fragile]{べき乗法(Power Method)}
  \begin{itemize}
    %\setlength{\itemsep}{1em}
  \item 適当なベクトル$v_1$から出発する
  \item $v_1$が最大固有ベクトル$\xi_1$と直交していないとすると
    \[
    v_1 = c_1 \xi_1 + c_2 \xi_2 + c_3 \xi_3 + \cdots + c_N \xi_N
    \]
    と展開できる($c_1 \ne 0$)。この両辺に$A$を次々掛けて行くと
    \begin{align*}
      v_2 = A v_1 &= c_1 \lambda_1 \xi_1 + c_2 \lambda_2 \xi_2 + c_3 \lambda_3 \xi_3 + \cdots + c_N \lambda_N \xi_N \\
      v_3 = A^2 v_1 &= c_1 \lambda_1^2 \xi_1 + c_2 \lambda_2^2 \xi_2 + c_3 \lambda_3^2 \xi_3 + \cdots + c_N \lambda_N^2 \xi_N \\
      \vdots \\
      v_{n+1} = A^n v_1 &= c_1 \lambda_1^n \xi_1 + c_2 \lambda_2^n \xi_2 + c_3 \lambda_3^n \xi_3 + \cdots + c_N \lambda_N^n \xi_N \\
      &= c_1 \lambda_1^n \Big[ \xi_1 + \sum_{k=2}^N \frac{c_k}{c_1} \big( \frac{\lambda_k}{\lambda_1}\big)^n \xi_k \Big] \approx c_1 \lambda_1^n \xi_1 \\
    \end{align*}
  \end{itemize}
\end{frame}

\begin{frame}[t,fragile]{べき乗法の収束}
  \begin{itemize}
    \setlength{\itemsep}{1em}
  \item べき乗法による固有値
    \[
    \frac{v_{n+1}^T v_{n+1}}{v_{n+1}^T v_n} = \lambda_1 + O\Big( \big(\frac{\lambda_2}{\lambda_1} \big)^{2n}\Big)
    \]
  \item 誤差の収束
    \[
    \frac{v_{n+1}^T v_{n+1}}{v_{n+1}^T v_n} \approx \lambda_1 + e^{-2n \ln (\lambda_1/\lambda_2)}
    \]
  \item $1 / \ln (\lambda_1/\lambda_2)$ 程度の反復が必要
  \item $\lambda_1$と$\lambda_2$が近い場合には、反復回数が非常に多くなる
  \end{itemize}
\end{frame}

\begin{frame}[t,fragile]{第2固有値・第3固有値$\cdots$}
  \begin{itemize}
    %\setlength{\itemsep}{1em}
  \item 第1固有ベクトル$\xi_1$の成分を行列から差し引く(減次)
    \[
    A_1 = A - \lambda_1 \xi_1 \xi_1^T
    \]
    この行列は、固有値 $0,\lambda_2,\lambda_3,\cdots,\lambda_N$を持つ
  \item 行列$A_1$に対してべき乗法を使うと、固有値$\lambda_2$と固有ベクトル$\xi_2$が得られる
  \item 第$k$固有値まで求まっている場合
    \[
    A_k = A - \sum_{i=1}^k \lambda_i \xi_i \xi_i^T
    \]
  \item 実際には数値誤差のため、ベクトルの直交性は厳密ではない
  \item 大きい方から数個程度を求めるのが限界
  \end{itemize}
\end{frame}

\begin{frame}[t,fragile]{Rayleigh-Ritzの方法}
  \begin{itemize}
    \setlength{\itemsep}{1em}
  \item $N \times N$行列$A$について、互いに正規直交するベクトル$v_1,v_2,\cdots,v_M$ ($M < N$)が張る部分空間の中で「最良の」固有ベクトルを求めたい
  \item $N \times M$行列
    \[
    V=(v_1 v_2 \cdots v_M)
    \]
    を定義すると、$V^TV=I$が成り立つ(ただし$VV^T \ne I$)
  \item 部分空間内のベクトルを$w = \sum_i a_i v_i$と表すと、$\frac{w^TAw}{w^Tw}$が極大値を取る(本当の固有ベクトルにできるだけ平行になる)条件は、
    \[
    \frac{\partial}{\partial a_i} \frac{w^TAw}{w^Tw} \sim \sum_j H_{ij}a_j - \lambda a_i = 0
    \]
  \end{itemize}
\end{frame}

\begin{frame}[t,fragile]{Rayleigh-Ritzの方法}
  \begin{itemize}
    \setlength{\itemsep}{1em}
  \item 行列
    \[
    H_{ij} = v_i^T A_{ij} v_j
    \]
    に対する固有値問題: $H a = \lambda a$
  \item $\lambda$: もとの行列の近似固有値(Ritz値)
  \item $Va$: もとの行列の近似固有ベクトル(Ritzベクトル)
  \item 最大固有値に対する良い近似固有値が欲しい場合 $\Rightarrow$ $v_1,v_2,\cdots,v_M$を最大固有ベクトルになるべく近い(しかし、互いに直交する)ベクトルに選べば良い
  \end{itemize}
\end{frame}

\begin{frame}[t,fragile]{Lanczos法}
  \begin{itemize}
    \setlength{\itemsep}{1em}
  \item 初期(ランダム)ベクトル$v_1$に加えて
    \[
    Av_1, Av_2, \cdots A^{M-1}v_1
    \]
    を正規直交化して$v_1,v_2,\cdots,v_M$を作る(Krylov部分空間)
  \item 部分空間でのRitz値を固有値の近似値とする
  \item $A^nv_1$はどんどん最大固有ベクトルに近づいていくので、$M \ll N$でも良い近似固有値が得られると期待される
  \item Lanczos法 (Arnordi法)
  \end{itemize}
\end{frame}

\begin{frame}[t,fragile]{Lanczos法}
  \begin{itemize}
    %\setlength{\itemsep}{1em}
  \item 正規化された初期(ランダム)ベクトル$v_1$から出発する %($v_0=0$とする)
  \item $v_2,v_3,\cdots$を生成する
    \begin{align*}
      v_2 &= (Av_1 - \alpha_1 v_1)/\beta_1 \\
      v_3 &= (Av_2 - \beta_1 v_1 - \alpha_2 v_2)/\beta_2 \\
      \vdots
    \end{align*}
    ここで
    \begin{align*}
      \alpha_i &= v_i^T A v_i \\
      \beta_i &= | A v_i - \beta_{i-1} v_{i-1} - \alpha_i v_i |, \ \beta_0 = 0
    \end{align*}
    と選ぶ
  \end{itemize}
\end{frame}

\begin{frame}[t,fragile]{Lanczos法}
  \begin{itemize}
    \setlength{\itemsep}{1em}
  \item $v_1,v_2,v_3,\cdots,v_{M+1}$は正規直交
  \item 漸化式を書き換えると
    \begin{align*}
      Av_1 &= \alpha_1 v_1 + \beta_1 v_2 \\
      Av_2 &= \beta_1 v_1 + \alpha_2 v_2 + \beta_2 v_3 \\
      Av_3 &= \beta_2 v_2 + \alpha_3 v_3 + \beta_3 v_4 \\
      \vdots \\
      Av_{M} &= \beta_{M-1} v_{M-1} + \alpha_M v_M + \beta_M v_{M+1}
    \end{align*}
  \end{itemize}
\end{frame}

\begin{frame}[t,fragile]{Lanczos法}
  \begin{itemize}
    \setlength{\itemsep}{1em}
  \item 行列で表現すると
    \begin{align*}
      \hspace*{-2em}
      A
      (v_1v_2\cdots v_M)
      &=
      (v_1v_2\cdots v_M v_{M+1})
      \begin{pmatrix}
        \alpha_1 & \beta_1\\
        \beta_1 & \alpha_2 & \beta_2 \\
        & \beta_2 & \alpha_3 & \beta_3 \\
        & & \beta_3 & \alpha_4 & \beta_4 \\
        & & & \ddots & \ddots & \ddots \\
        & & & & \beta_{M-1} & \alpha_M \\
        & & & & & \beta_M \\
      \end{pmatrix}
    \end{align*}
    両辺に左から$(v_1v_2\cdots v_M)^T$をかけると
    \[
    (v_1v_2\cdots v_M)^T A (v_1v_2\cdots v_M)
    \]
    は3重対角行列となることがわかる
  \end{itemize}
\end{frame}

\begin{frame}[t,fragile]{Lanczos法}
  \begin{itemize}
    %\setlength{\itemsep}{1em}
  \item 原理的には、$N$ステップ目で$\beta_N=0$となり、3重対角化が完了する
  \item 実際には、数値誤差のため$v_1,v_2,v_3\cdots$の直交性が崩れる
    \begin{itemize}
      \item $M$を大きくしすぎると、おかしな固有値が出てくる
      \item 全ての固有値が欲しい場合にはHouseholder法を使う
    \end{itemize}
  \item Lanczos法では、大きな固有値に対応する固有ベクトルにできるだけ近いものから部分空間を作っていく
    \begin{itemize}
      \item 100万次元以上の行列の場合でも$M=100 \sim 200$程度で最初の数個の固有値は精度良く求まる
    \end{itemize}
  \item 必要な操作は、行列とベクトルの積、ベクトルの内積・スケーリング・和のみ
    \begin{itemize}
      \item 疎行列の場合、非常に効率が良い
    \end{itemize}
  \end{itemize}
\end{frame}
