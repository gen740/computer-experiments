\begin{frame}[t,fragile]{物理の問題にあらわれる行列演算}
  \begin{itemize}
    %\setlength{\itemsep}{1em}
  \item 連立一次方程式・逆行列
    \begin{itemize}
    \item 偏微分方程式の境界値問題
    \item 非線形連立方程式に対するニュートン法
    \end{itemize}
  \item 対角化・特異値分解
    \begin{itemize}
    \item 固有値問題・行列関数
    \item 最小二乗近似
    \end{itemize}
  \item 計算機は大規模行列演算が得意
    \begin{itemize}
    \item 直接法: $\sim10^4$次元
    \item 疎行列に対する反復解法: $\sim10^{9}$次元
    \end{itemize}
  \item 行列演算についてはライブラリがよく整備されている
    \begin{itemize}
    \item それぞれの原理とその特徴を理解して正しく使うことが重要
    \item 適切なライブラリを使うことで数十倍あるいはそれ以上速くなることも
    \end{itemize}
  \end{itemize}
\end{frame}
