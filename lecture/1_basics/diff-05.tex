\begin{frame}[t,fragile]{勾配の計算}
  \begin{itemize}
    %\setlength{\itemsep}{1em}
  \item $f(x_1,x_2,\cdots,x_n)$ の勾配
    \[
      \nabla f = \Large( \frac{\partial f}{\partial x_1}, \frac{\partial f}{\partial x_2}, \cdots, \frac{\partial f}{\partial x_n} \Large)
    \]
  \item 差分による計算
    \begin{itemize}
      \item 関数$f$の値を最低でも$2n$回評価する必要がある
      \item 方向によって適切な$h$の値は異なる → 関数評価の回数はさらに増加
    \end{itemize}
  \item 数式処理(MATLAB, Mathematica等)による勾配の計算
    \begin{itemize}
      \item $f$が複雑になると勾配の表式が長大に
      \item 勾配の値を評価するコストも大きくなる
    \end{itemize}
  \item 自動微分(automatic differentiation; autodiff)
    \begin{itemize}
      \item 関数fの値の評価に必要なコストの定数倍の手間で勾配を評価可能
      \item ハイパーパラメータ$h$が存在しない
    \end{itemize}
  \end{itemize}
\end{frame}
