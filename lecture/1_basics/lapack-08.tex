\begin{frame}[t,fragile]{二次元配列}
  \begin{itemize}
    \setlength{\itemsep}{1em}
  \item C言語では、二次元配列は一次元配列の先頭をさす(ポインタ)の配列として表される(と理解しておけば良い)
  \item \verb+a[i]+は、要素\verb+a[i][0]+を指すポインタ
    \begin{itemize}
    \item \verb+a+ と \verb+&a[0]+ は等価 (\verb+&a[0][0]+ ではない)
    \item \verb+a[0]+ と \verb+&a[0][0]+ は等価
    \item \verb+a[2]+ と \verb+&a[2][0]+ は等価
    \item \verb^(a+2)^ と \verb^&a[2]^ は等価
    \item \verb^(*(a+2))[3]^ と \verb^*(*(a+2)+3)^ と \verb^a[2][3]^ は等価
    \item \verb^*(a+2)[3]^ と \verb^*((a+2)[3])^ と \verb^*(a[5])^ と\verb^a[5][0]^ は等価
    \item \verb^[]^は\verb^*^よりも強い
    \end{itemize}
  \item ポインタ確認プログラム: \href{https://github.com/todo-group/computer-experiments/blob/master/exercise/matrix/pointer-matrix.c}{pointer-matrix.c}
  \end{itemize}
\end{frame}
