\begin{frame}[t,fragile]{ニュートン法の収束}
  \begin{itemize}
    \setlength{\itemsep}{1em}
  \item $x_n$が$x_0$に十分近い時
    \begin{align*}
      f(x_n) &\approx f'(x_0) (x_n-x_0) + f''(x_0) \frac{(x_n - x_0)^2}{2} \\
      f'(x_n) &\approx f'(x_0) + f''(x_0) (x_n - x_0)
    \end{align*}
  \item ニュートン法で一回反復すると
    \begin{align*}
      x_{n+1} =  x_n - \frac{f(x_n)}{f'(x_n)} &\approx x_n - (1-\frac{f''(x_0)}{f'(x_0)}\frac{(x_n-x_0)}{2})(x_n-x_0) \\
      (x_{n+1}-x_0) &\approx \frac{f''(x_0)}{2f'(x_0)} (x_n - x_0)^2
    \end{align*}
    \item 一回の反復で誤差が2乗で減る(正しい桁数が倍に増える) ⇒ 二次収束
  \end{itemize}
\end{frame}
