\begin{frame}[t,fragile]{複素数}
  \begin{itemize}
    %\setlength{\itemsep}{1em}
  \item {\tt complex.h}をincludeすることで、double complex 型 (float complex 型)が使えるようになる
  \item 虚数単位 {\tt I} が定義されている
  \item {\tt cexp}, {\tt csin}, {\tt clog}等の初等関数が使える
  \item 実部は{\tt creal}, 虚部は{\tt cimag}で取り出せる
  \item プログラム例: \href{https://github.com/todo-group/computer-experiments/blob/master/exercise/basics/complex.c}{complex.c}
\begin{lstlisting}
#include <complex.h>
...
double complex x, y;
x = 0 + 1 * I; /* 虚数単位 */
y = cexp(x * M_PI);
printf("i = (%lf,%lf)\n", creal(x), cimag(x));
printf("e^{i*pi} = (%lf,%lf)\n", creal(y), cimag(y));
\end{lstlisting}
  \end{itemize}
\end{frame}
