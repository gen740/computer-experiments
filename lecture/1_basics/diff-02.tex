\begin{frame}[t,fragile]{差分の一般式}
  \begin{itemize}
    %\setlength{\itemsep}{1em}
  \item 関数のテイラー展開: $\displaystyle f(x+h) = \sum_{k} \frac{h^k}{k!} f^{(k)}(x)$
  \item $f^{(m)}(x)$を$n$個の$f(x+h_j)$の線形結合で表す($n \ge m+1$)
    \begin{align*}
      f^{(m)}(x) &\approx \sum_j a_j f(x+h_j) = \sum_j a_j \sum_{k} \frac{h_j^k}{k!} f^{(k)}(x) \\
      & = \sum_{k} C_k f^{(k)}(x) \qquad (C_k \equiv \sum_j a_j \frac{h_j^k}{k!})
    \end{align*}
  \item $C_k = \delta_{k,m}$ ($k = 0 \cdots n-1$)となるように$a_0 \cdots a_{n-1}$を決める
  \item 行列$\displaystyle G_{kj} = \frac{h_j^k}{k!}$と列ベクトル$a_j$と$b_k = \delta_{k,m}$を導入すると、条件式は$G a = b$と書ける (連立一次方程式)
  \end{itemize}
\end{frame}
