\section{密行列の対角化}

\begin{frame}[t,fragile]{基本方針}
  \begin{itemize}
    %\setlength{\itemsep}{1em}
  \item やってはいけない方法: 特性方程式
    \[
    |\lambda I - A| = 0
    \]
    の係数を求めて、代数方程式として解く
    \begin{itemize}
    \item 数値的に不安定 (代数方程式の解は係数の誤差に対して敏感)
    \item 計算コスト大[$\sim O(N!)$]
    \end{itemize}
  \item スタンダードな方法: 行列を次々に直交変換して、対角行列(あるいは三重対角行列)に近づけていく
    \[
    A \rightarrow U_1^T A U_1 \rightarrow U_2^T (U_1^T A U_1) U_2 \rightarrow U_3^T (U_2^T (U_1^T A U_1) U_2) U_3 \rightarrow \cdots
    \]
  \item 固有値は変換された行列の固有値、固有ベクトルは変換後の行列の固有ベクトルに左から$U_1 U_2 U_3 \cdots$を掛けたもの
  \end{itemize}
\end{frame}

\begin{frame}[t,fragile]{Jacobi法}
  \begin{itemize}
    \setlength{\itemsep}{1em}
  \item 直交行列$U_{pq}$を以下のように選ぶ ($(p,p),(p,q),(q,p),(q,q)$成分を除くと単位行列)
    \[
    U_{pq} =
    \begin{pmatrix}
      1 \\
      & \ddots \\
      & & 1 \\
      & & & \cos \theta & & & \sin \theta \\
      & & & & 1 \\
      & & & & & 1 \\
      & & & -\sin \theta & & & \cos \theta \\
      & & & & & & & 1 \\
      & & & & & & & & \ddots \\
      & & & & & & & & & 1 \\
    \end{pmatrix}
    \]
  \end{itemize}
\end{frame}

\begin{frame}[t,fragile]{Jacobi法による相似変換}
  \begin{itemize}
    \setlength{\itemsep}{1em}
  \item $B=U_{pq}^{-1} A U_{pq}$により、$A$の$p$行、$q$行、$p$列、$q$列のみが変更を受ける
    \begin{align*}
      b_{pk} &= b_{kp} = a_{pk} \cos \theta - a_{qk} \sin \theta \qquad k \ne p,q \\
      b_{qk} &= b_{kq} = a_{pk} \sin \theta + a_{qk} \cos \theta \qquad k \ne p,q \\
      b_{pp} &= \frac{a_{pp}+a_{qq}}{2} + \frac{a_{pp}-a_{qq}}{2} \cos 2 \theta - a_{pq} \sin 2 \theta \\
      b_{qq} &= \frac{a_{pp}+a_{qq}}{2} - \frac{a_{pp}-a_{qq}}{2} \cos 2 \theta + a_{pq} \sin 2 \theta \\
      b_{pq} &= b_{qp} = \frac{a_{pp}-a_{qq}}{2} \sin 2 \theta + a_{pq} \cos 2 \theta
    \end{align*}
  \item $b_{pq} = b_{qp} = 0$とするには、$\theta$を次のように選べば良い
    \[
    \tan 2 \theta = - \frac{2 a_{pq}}{a_{pp}-a_{qq}}
    \]
  \end{itemize}
\end{frame}

\begin{frame}[t,fragile]{Jacobi法の収束}
  \begin{itemize}
    \setlength{\itemsep}{1em}
  \item 相似変換により対角和は不変に保たれるので
    \[
      {\rm tr} \, A^T A = {\rm tr} \, B^T B \ \ \Rightarrow \ \
      \sum_{i,j} a_{ij}^2 = \sum_{i,j} b_{ij}^2
    \]
  \item 一方、この変換で
    \[
    b_{pp}^2 + b_{qq}^2 = b_{pp}^2 + 2 b_{pq}^2 + b_{qq}^2 = a_{pp}^2 + 2 a_{pq}^2 + a_{qq}^2
    \]
    すなわち、変換により、対角成分の二乗和は増加する $\Rightarrow$ 非対角成分の二乗和は単調減少
  \item 全ての非対角成分が十分小さくなるまで繰り返す
  \item 固有値=対角成分、固有ベクトル$=U_1 U_2 U_3 \cdots$
  \end{itemize}
\end{frame}

\begin{frame}[t,fragile]{3重対角化}
  \begin{itemize}
    %\setlength{\itemsep}{1em}
  \item 対角化は有限回の手続きでは行えない
  \item 3重対角化であれば、$O(N^3)$の有限回の計算で決定論的に行える
  \item Givens変換: Jacobi変換と同じ相似変換を利用
    \begin{itemize}
    \item $U_{32}$で(3,1)と(1,3)を消去 $\Rightarrow$ $U_{42}$で(4,1)と(1,4)を消去 $\Rightarrow$ $U_{52}$で(5,1)と(1,5)を消去 $\Rightarrow$ $U_{62},\cdots,U_{N,2}$ $\Rightarrow$ $U_{43},U_{53},\cdots,U_{N,3}$ $\Rightarrow$ $\cdots$ $\Rightarrow$ $U_{n,n-1}$で($n,n-2$)と($n-2,n$)を消去
    \item $(4/3)N^3$回の乗算と$(2/3)N^3$回の加減算で3重対角化される
    \end{itemize}
  \item Householder変換: $U = I - 2 w w^T / |w|^2$
    \begin{itemize}
    \item $(2/3)N^3$回の乗算と加減算で3重対角化される
    \item Givens変換に比べ少し効率的なので、こちらが広く使われている
    \end{itemize}
  \end{itemize}
\end{frame}

\begin{frame}[t,fragile]{3重対角行列の対角化}
  \begin{itemize}
    %\setlength{\itemsep}{1em}
  \item 二分法、QR分解、分割統治法、MRRRなど様々な方法が知られている
  \item 固有ベクトル
    \begin{itemize}
    \item QR分解では3重対角行列の固有ベクトルも同時に求まる
    \item あるいは、固有値を求めた後、逆反復法を用いて固有ベクトルを求める
    \end{itemize}
  \item 逆反復法
    \begin{itemize}
    \item 近似固有値を$\mu$とするとき、行列$(A - \mu I)^{-1}$を考えると、固有ベクトルは$A$と同じ、固有値は$(\lambda-\mu)^{-1}$。
    \item $\mu$が十分に正確であれば、$(\lambda-\mu)^{-1}$は絶対値最大の固有値。行列$(A - \mu I)^{-1}$を適当な初期ベクトルにかけ続けると$\lambda$に対応する固有ベクトルに収束(c.f. べき乗法)
    \item 実際には$(A-\mu I) x' = x$という連立方程式を繰り返し解く
    \end{itemize}
  \end{itemize}
\end{frame}

\begin{frame}[t,fragile]{LAPACKの対角化ルーチン}
  \begin{itemize}
    %\setlength{\itemsep}{1em}
  \item 様々な対角化ルーチンが準備されている
    \begin{itemize}
    \item 倍精度実対称行列の対角化 {\tt dsyev}
      \url{http://www.netlib.org/lapack/explore-html/dd/d4c/dsyev_8f.html}
    \item Fortranによる関数宣言
\begin{lstlisting}
subroutine dsyev(character JOBZ, character UPLO,
  integer N, double precision, dimension(lda, *) A,
  integer LDA, double precision, dimension(*) W,
  double precision, dimension(*) WORK,
  integer LWORK, integer INFO)		
\end{lstlisting}
    \end{itemize}
  \item 他にも{\tt dsyevd}、{\tt dsyevr}、{\tt dsyevx}などがある \\
    3重対角化までは同じ。3重対角行列の対角化が異なる
  \item 単精度版の{\tt ssyev}、複素(エルミート行列)版の{\tt zheev}など
  \end{itemize}
\end{frame}

\begin{frame}[t,fragile]{複素エルミート行列の固有分解}
  \begin{itemize}
    %\setlength{\itemsep}{1em}
  \item 固有値は実数
  \item これまでの方法がそのまま使える (ただし、転置 $\rightarrow$ 複素転置)
  \item 実対称行列用のサブルーチンを使っての対角化も可能
    \begin{itemize}
    \item エルミート行列を実部と虚部に分ける: $A = R + iW$
    \item エルミート行列の固有値問題 $(R + iW)(u+iv) = \lambda(u+iv)$を$2N \times 2N$の実対称行列の問題に書き換える
      \[
      \begin{pmatrix} R & -W \\ W & R \end{pmatrix}
      \begin{pmatrix} u \\ v \end{pmatrix}
      = 
      \lambda \begin{pmatrix} u \\ v \end{pmatrix}
      \]
    \item 固有値は同じ固有値が2度づつ現れる
    \item 対応する複素行列の固有ベクトルは、$u+iv$と$-v+iu$
    \end{itemize}
  \end{itemize}
\end{frame}
