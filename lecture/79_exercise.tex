\section{実習その7}

\begin{frame}[t,fragile]{EX7-1: Nelder-Meadの滑降シンプレックス法}
  \begin{itemize}
    %\setlength{\itemsep}{1em}
  \item[7-1-1] 一次元関数$f(x)=(5x+x^2)+70\sin(x)$は多くの極小をもつ。\href{https://github.com/todo-group/computer-experiments/blob/master/exercise/optimization/nelder_mead_1d.c}{exercise/optimization/nelder\_mead\_1d.c}は、一次元のNelder-Meadの滑降シンプレックス法によりこの関数の極値を求めるプログラムである。コンパイルして実行せよ。初期値を変えてその収束の様子を観察せよ
  \item[7-1-2] {\tt nelder\_mead\_1d.c}のソースコードを解析せよ
  \item[7-1-3] Nelder-Meadの滑降シンプレックス法を用いて、二次元関数$g(x,y)=-10(x^2+y^2)+(x^2+y^2)^2-2(x+y)$の極値(最小値)を計算するプログラムを作成せよ({\tt nelder\_mead\_1d.c}を参考に\href{https://github.com/todo-group/computer-experiments/blob/master/exercise/optimization/nelder_mead_2d.c}{exercise/optimization/nelder\_mead\_2d.c}を完成させよ)
  \end{itemize}
\end{frame}

\begin{frame}[t,fragile]{EX7-2: 黄金分割、シミュレーテッド・アニーリング}
  \begin{itemize}
    %\setlength{\itemsep}{1em}
  \item[7-2-1] 黄金分割による囲い込み法を用いて、一次元関数$f(x)=(5x+x^2)+70\sin(x)$の極値を求めるプログラムを作成せよ。結果をEX7-1-1と比較せよ。(\href{https://github.com/todo-group/computer-experiments/blob/master/exercise/optimization/golden_section.c}{exercise/optimization/golden\_section.c}に初期囲い込みまでをおこなうプログラムがある。これに黄金分割探索
を追加せよ)
  \item[7-2-2] シミュレーテッド・アニーリングにより、離散最適化問題(巡回セールスマン問題など)を解くプログラムを作成せよ。温度のスケジューリングを変えることで、正解を得られる確率がどのように変化するか調べよ
  \item[7-2-3] 共役勾配法を用いた連立一次方程式の解法では「前処理」が非常に重要であることが知られている。「前処理」とは何か? 前処理が必要となる理由は? また、実際の数値計算ではどのような前処理方法が使われているか、調べてみよ
  \end{itemize}
\end{frame}
