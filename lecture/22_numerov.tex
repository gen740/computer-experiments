\section{Numerov法}

\begin{frame}[t,fragile]{時間依存しないシュレディンガー方程式}
  \begin{itemize}
    \setlength{\itemsep}{1em}
  \item 井戸型ポテンシャル中の一粒子問題
    \begin{align*}
      \big[ -\frac{\hbar^2}{2m}\frac{d^2}{dx^2} + V(x) \big] \psi(x) = E \psi(x) \\
      V(x) = \begin{cases}
        0 & \text{$a \le x \le b$} \\ \infty & \text{otherwise}
      \end{cases}
    \end{align*}
  \item $\hbar^2/2m = 1$、$a=0$、$b=1$となるように変数変換して
    \begin{align*}
      \big( \frac{d^2}{dx^2} + E \big) \psi(x) = 0 \qquad 0 \le x \le 1
    \end{align*}
    を境界条件$\psi(0) = \psi(1) = 0$のもとで解けば良い
  \end{itemize}
\end{frame}

\begin{frame}[t,fragile]{微分方程式の積分による解法}
  \begin{itemize}
    \setlength{\itemsep}{1em}
  \item Numerov法
    \begin{itemize}
    \item 二階の常微分方程式で一階の項がない場合に使える
    \item 4次の陰解法
    \item 方程式が線形の場合は陽解法に書き直せる
    \end{itemize}
  \item 微分方程式
    \[
    \frac{d^2y}{dx^2} = f(x,y)
    \]
  $y=y(x)$を$x=x_i$のまわりでテイラー展開する。$x_{i \pm 1} = x_i \pm h$での表式は
      \[
      y(x_{i \pm 1}) = y(x_i) \pm h y'(x_i) + \frac{h^2}{2} y''(x_i) \pm \frac{h^3}{6} y'''(x_i) + \frac{h^4}{24} y''''(x_i)  + O(h^5)
      \]
  \end{itemize}
\end{frame}

\begin{frame}[t,fragile]{Numerov法}
  \begin{itemize}
    \setlength{\itemsep}{1em}
  \item 二階微分の差分近似 ($y_i \equiv y(x_i)$等と書く)
    \[
    \frac{y_{i+1} - 2 y_i + y_{i-1}}{h^2} = y''_{i} + \frac{h^2}{12} y''''_{i} + O(h^4)
    \]
  一方で、微分方程式より
    \[
    y''''_i = \frac{d^2f}{dx^2}\Big|_{x=x_i} = \frac{f_{i+1}-2f_i+f_{i-1}}{h^2} + O(h^2)
    \]
    組み合わせると
    \[
    y_{i+1} = 2y_i - y_{i-1} + \frac{h^2}{12} (f_{i+1} + 10f_{i} + f_{i-1}) + O(h^6)
    \]
  \end{itemize}
\end{frame}

\begin{frame}[t,fragile]{Numerov法}
  \begin{itemize}
    \setlength{\itemsep}{1em}
  \item 方程式が線形の場合、$f(x,y) = -a(x) y(x)$を代入すると
    \[
    y_{i+1} = 2y_i - y_{i-1} - \frac{h^2}{12} (a_{i+1}y_{i+1} + 10a_{i}y_{i} + a_{i-1}y_{i-1}) + O(h^6)
    \]
  $y_{i+1}$を左辺に集めると、陽解法となる
    \[
    y_{i+1} = \frac{2 (1-\frac{5h^2}{12} a_i)y_i - (1 + \frac{h^2}{12} a_{i-1}) y_{i-1}}{1 + \frac{h^2}{12} a_{i+1}} + O(h^6)
    \]
  \end{itemize}
\end{frame}

\begin{frame}[t,fragile]{Numerov法による固有値問題の解法}
  \begin{itemize}
    \setlength{\itemsep}{1em}
  \item $x_i=h \times i$ ($h=1/n$)、$x_0=0$、$x_n=1$とする
  \item $\psi(x_0)=0$、$\psi(x_1) = 1$を仮定 ($\psi'(x_0)=1/h$と与えたことに相当)
  \item $E = 0$とおく
  \item Numerov法を用いて、$x=x_n$まで積分
  \item $\psi(x_n)$の符号がかわるまで、$E$を少しずつ増やす
  \item 符号が変わったら、$E$の区間を半分ずつに狭めていき、$\psi(x_n)=0$となる$E$ (固有エネルギー)と$\psi(x)$ (波動関数)を得る
  \end{itemize}
\end{frame}
