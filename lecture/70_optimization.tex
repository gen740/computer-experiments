\section{最適化問題}

\begin{frame}[t,fragile]{最適化問題}
  \begin{itemize}
    \setlength{\itemsep}{1em}
  \item 目的関数$f(x)$の最小値(あるいは最大値)とその場所を求めたい
  \item どういう問題を解くのに使えるか?
    \begin{itemize}
    \item 変分原理が成り立つ問題: 最小作用の原理、最小エネルギーの原理
    \item コスト関数が定義できる問題: 最小二乗法(線形回帰、非線形回帰)、(連立)方程式の求階、機械学習
    \end{itemize}
  \item ほとんど全ての問題はコスト関数をうまく定義することで、最適化問題に書き換えることができる
    \begin{itemize}
    \item (一般に)最適化問題として解くのは最終手段
    \item もっと良い方法があるときはそちらを使う
    \end{itemize}
  \end{itemize}
\end{frame}

\begin{frame}[t,fragile]{最適化問題}
  \begin{itemize}
    %\setlength{\itemsep}{1em}
  \item 最適化問題の種類
    \begin{itemize}
    \item 連続最適化問題
    \item 離散最適化(組み合わせ最適化)問題 $\Leftarrow$ 難しい
    \end{itemize}
  \item 真の(大局的な)最小値(最大値)を求めるのは難しい
  \item 一般的には極値を求めることしかできない
  \item 多次元では極小を囲い込むことができない
  \item 導関数を使う方法: ニュートン法、準ニュートン法、最急降下法、勾配降下法、共役勾配法、$\cdot$
  \item 使わない方法: 囲い込み法、Nelder-Meadの滑降シンプレックス法、シミュレーテッド・アニーリング、$\cdot$
  \end{itemize}
\end{frame}
