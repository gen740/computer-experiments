\begin{frame}[t,fragile]{シンプレクティック法の一般論}
  \begin{itemize}
    %\setlength{\itemsep}{1em}
  \item 二次のシンプレクティック法
    \[
    e^{h (A+B)} = e^{\frac{h}{2}B} e^{hA} e^{\frac{h}{2}B} + O(h^3)
    \]
    から
    \begin{align*}
      \hat{S}_2 &= e^{\frac{h}{2}\hat{D}(T)} e^{h\hat{D}(V)} e^{\frac{h}{2}\hat{D}(T)}
    \end{align*}
    $\Rightarrow$ リープ・フロッグ法
  \item 四次のシンプレクティック法 (吉田の方法)
    \begin{align*}
      \hat{S}_4 &= e^{c_1h\hat{D}(T)} e^{d_1 h\hat{D}(V)} e^{c_2 h\hat{D}(T)} e^{d_2 h\hat{D}(V)} e^{c_3h\hat{D}(T)} e^{d_3 h\hat{D}(V)} e^{c_4h\hat{D}(T)} \\
      & c_1 = c_4 = \frac{1}{2(2-2^{1/3})}, \ c_2 = c_3 = \frac{1-2^{1/3}}{2(2-2^{1/3})}, \\
      & d_1 = d_3 = \frac{1}{2-2^{1/3}}, \ d_2 = \frac{2^{1/3}}{2-2^{1/3}}
    \end{align*}
  \end{itemize}
\end{frame}
