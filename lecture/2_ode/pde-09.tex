\begin{frame}[t]{クランク・ニコルソン法}
  \begin{itemize}
  \item クランク・ニコルソン法
    \[
    \Psi^{n+1} = \frac{1 -  i \frac{\Delta t}{2} H}{1 +  i \frac{\Delta t}{2} H} \Psi^{n}
    \]
  \item (数値精度の範囲で)ユニタリー行列であるので、ノルムは保存
  \item $(1 +  i \frac{\Delta t}{2} H)^{-1}$を掛ける $\Rightarrow$ 連立一次方程式を解く必要がある
    \begin{itemize}
    \item まず、$\Psi = (1 - i \frac{\Delta t}{2} H) \Psi^n$ を計算
    \item 次に、$(1 +  i \frac{\Delta t}{2} H) \Psi^{n+1} = \Psi$ を解く(連立一次方程式)
    \end{itemize}
  \item 陰解法の一種
  \end{itemize}
\end{frame}
