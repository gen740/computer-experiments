\begin{frame}[t,fragile]{ガウスの消去法のコード}
\begin{lstlisting}
for (k = 0; k < n; ++k) {
  for (i = k + 1; i < n; ++i) {
    for (j = k + 1; j < n; ++j) {
      mat_elem(a,i,j) -= mat_elem(a,k,j) * mat_elem(a,i,k) / mat_elem(a,k,k);
    }
    b[i] -= b[k] * mat_elem(a,i,k) / mat_elem(a,k,k);
  }
}
for (k = n-1; k >= 0; --k) {
  for (j = k + 1; j < n; ++j) {
    b[k] -= mat_elem(a,k,j) * b[j];
  }
  b[k] /= mat_elem(a,k,k);
}
\end{lstlisting}
\begin{itemize}
\item C言語では配列の添字が0から始まることに注意
\item \verb+mat_elem(a,i,j)+は行列aの(i,j)成分を表す({\tt cmatrix.h}中で定義)
\item サンプルコード: \href{https://github.com/todo-group/computer-experiments/blob/master/exercise/linear_system/gauss.c}{gauss.c}
\item 入力ファイル: \href{https://github.com/todo-group/computer-experiments/blob/master/exercise/linear_system/input1.dat}{input1.dat}
\end{itemize}
\end{frame}
