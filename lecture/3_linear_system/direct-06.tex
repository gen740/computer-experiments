\begin{frame}[t,fragile]{ガウスの消去法}
  \begin{itemize}
    %\setlength{\itemsep}{1em}
  \item 最終的には、左辺が右上三角形をした連立方程式となる
    \begin{align*}
    a_{11}^{(1)} x_1 + a_{12}^{(1)} x_2 + a_{13}^{(1)} x_3 + \cdots + a_{1n}^{(1)} x_n &= b_{1}^{(1)} \\
    a_{22}^{(2)} x_2 + a_{23}^{(2)} x_3 + \cdots + a_{2n}^{(2)} x_n &= b_{2}^{(2)} \\
    a_{33}^{(3)} x_3 + \cdots + a_{3n}^{(3)} x_n &= b_{3}^{(3)} \\
    \cdots \\
    a_{n-1,n-1}^{(n-1)} x_{n-1} + a_{n-1,n}^{(n-1)} x_n &= b_{n-1}^{(n-1)} \\
    a_{nn}^{(n)} x_n &= b_{n}^{(n)}
    \end{align*}
  \item これを下から順に解いていけばよい(後退代入)
  \end{itemize}
\end{frame}
