\begin{frame}[t,fragile]{ピボット選択}
  \begin{itemize}
    %\setlength{\itemsep}{1em}
  \item ガウスの消去法の途中で$a_{kk}^{(k)}$が零になると、計算を先に進めることができなくなる
  \item 行を入れ替えても、方程式の解は変わらない $\Rightarrow$ $k$行以降で、$a_{ik}^{(k)}$が非零の行と入れ替える (ピボット選択)
  \item 実際のコードでは、情報落ちを防ぐため、$a_{kk}^{(k)}$が零でない場合でも、$a_{ik}^{(k)}$の絶対値が最大の行と入れ替える
  \item ピボット選択が必要となる例: \href{https://github.com/todo-group/computer-experiments/blob/master/exercise/linear_system/input2.dat}{input2.dat}
    \begin{align*}
      \begin{bmatrix} 1 & 4 & 7 \\ 2 & 8 & 5 \\ 3 & 6 & 10 \end{bmatrix} \begin{bmatrix} x_1 \\ x_2 \\ x_3 \end{bmatrix} = \begin{bmatrix} 30 \\ 33 \\ 45 \end{bmatrix}
    \end{align*}
    \item 行列がrank落ちしている場合は、ピボット選択を行っても途中で0になる (cf. 特異値分解を用いた最小二乗解)
  \end{itemize}
\end{frame}
