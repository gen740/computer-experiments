\section{実習2}

\begin{frame}[t,fragile]{EX2-1: 常微分方程式の初期値問題}
  \begin{itemize}
    \setlength{\itemsep}{1em}
  \item[2-1-1] 空気による摩擦のあるバネの問題を考える。壁にバネが繋が
    れ、バネの先には質量$m$の物体が繋がっている。床との摩擦は考えない
    ものとする。バネの伸びる方向に$x$座標を取り、自然長の位置を原点と
    すると、物体の運動方程式は以下のように与えられる。
    \[
    m\frac{\mathrm{d} ^2x}{\mathrm{d} t^2} = -kx - \kappa \frac{\mathrm{d} x}{\mathrm{d} t} 
    \]
    ここで、$k$はバネ定数、$\kappa$は摩擦の比例定数とする。Euler法を使い$x(t)$を30 [sec]まで計算せよ。その際、刻み幅$h$の大きさを変化させ、解の変わる様子を確認せよ。ただし、$k$ = 2 [N/m], $\kappa$ = 0.2 [kg/sec]、$m$ = 1 [kg]、初期条件は$x(0)$ = 10 [m]、$x'(0)$ = 0 [m/sec] とする
  \end{itemize}
\end{frame}

\begin{frame}[t,fragile]{EX2-2: 高次の解法}
  \begin{itemize}
    \setlength{\itemsep}{1em}
  \item[2-2-1] 中点法、3次のRunge-Kutta法、4次のRunge-Kutta法を用いて同様の計算を行い、精度の向上の様子を調べよ
  \item[2-2-2] 空気抵抗も床との摩擦も無い場合についてシミュレーションを行い、全エネルギー(運動エネルギーとポテンシャルエネルギーの和)の時間変化の様子を観察せよ。なぜ、全エネルギーが保存しないのか? 一方で、ある一定の誤差の範囲内で全エネルギーを保存する手法として、Symplectic法が知られている。この方法について調べ、実際にプログラムを作成し、シミュレーション結果について考察せよ
  \end{itemize}
\end{frame}
