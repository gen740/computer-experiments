\begin{frame}[t,fragile]{「共役な方向」とは}
  \begin{itemize}
    \setlength{\itemsep}{1em}
  \item あるベクトル${\bf p}$にそった一次元の最適化が完了したとする
    \begin{itemize}
    \item その点における${\bf p}$方向の勾配は零。すなわち${\bf p}^T (\nabla f)=0$
    \item ${\bf p}$方向の勾配の値を変化させないようにしたい
  \end{itemize}
  \item 次に、${\bf q}$にそって、${\bf x}+\epsilon {\bf q}$と移動するとする。その時の勾配の変化は
    \[
      \delta(\nabla f) = A \times (\epsilon {\bf q}) \sim A {\bf q}
      \]
      これが${\bf p}$に垂直であるためには
    \[
      {\bf p}^T A {\bf q} = 0
      \]
    \item この関係が成り立つ時、${\bf p}$と${\bf q}$は「互いに共役」という
  \end{itemize}
\end{frame}
