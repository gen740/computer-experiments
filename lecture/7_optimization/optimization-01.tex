\begin{frame}[t,fragile]{最適化問題}
  \begin{itemize}
    %\setlength{\itemsep}{1em}
  \item 目的関数(コスト関数)$f(x)$の最小値(あるいは最大値)とその場所を求めたい
  \item どういう問題を解くのに使えるか?
    \begin{itemize}
    \item 変分原理が成り立つ問題: 最小作用の原理、最小エネルギーの原理$\cdots$
    \item 目的関数が定義できる問題: 最小二乗法(線形回帰、非線形回帰)、(連立)方程式、常/偏微分方程式、機械学習$\cdots$
    \end{itemize}
  \item ほぼ全ての問題は、目的関数をうまく定義することで最適化問題に書き換えることができる
    \begin{itemize}
    \item (一般に)最適化問題として解くのは最終手段
    \item それぞれの問題に特化したより良い方法があるときはそちらを使う
    \end{itemize}
  \end{itemize}
\end{frame}
