\begin{frame}[t,fragile]{囲い込み法(一次元の最適化)}
  \begin{itemize}
    %\setlength{\itemsep}{1em}
  \item 関数$f(x)$の極小(あるいは極大)点を求める
  \item $f(a) > f(b) < f(c)$を満たす3点の組$a < b < c$の領域を狭めていく
  \item $[a,b]$、$[b,c]$の広い方(例えば後者)を$b$から見て、黄金比
    [$1:(1+\sqrt{5})/2 \approx 0.382:0.618$]に内分する点を$x$とする
    \begin{itemize}
    \item $f(b) > f(x)$の場合: $[b,c]$を新しい領域にとる
    \item $f(b) < f(x)$の場合: $[a,x]$を新しい領域にとる
    \end{itemize}
  \item もともとの$b$が$[a,c]$を$0.382:0.618$に内分する点だった場合、
    新しい領域の幅は、どちらの場合も0.618
  \item 最初の比率が黄金比からずれていたとしても、黄金比に収束
  \item 黄金分割法(golden section)とも呼ばれる
  \end{itemize}
\end{frame}
