\begin{frame}[t,fragile]{共役勾配法(conjugate gradient)}
  \begin{itemize}
    \setlength{\itemsep}{1em}
  \item 目的関数がある点のまわりで
    \[
    f({\bf x}) \approx c - {\bf b}^T {\bf x} + \frac{1}{2} {\bf x}^T A {\bf x}
    \]
    と近似できるとする
  \item ${\bf x}$における勾配は、連立方程式$A{\bf x}={\bf b}$の「残差」の形で書ける
    \[
    -\nabla f = {\bf b} - A {\bf x}
    \]
  \item 新しい勾配方向ではなく、それまでとは「共役な方向」に進みたい
  \end{itemize}
\end{frame}
