\begin{frame}[t,fragile]{Nelder-Meadの滑降シンプレックス法}
  \begin{itemize}
    %\setlength{\itemsep}{1em}
  \item 関数値のみ。導関数の情報を必要としない
  \item プログラミングが簡単
  \item 収束は遅いが、安定に極小値が求まる
  \item $N+1$個の頂点からなる$N$次元の超多面体(シンプレックス)を変形しながら、極小値を探す
    \begin{itemize}
    \item 2次元: 三角形
    \item 3次元: 四面体
    \end{itemize}
  \item 別名「アメーバ法」
  \end{itemize}
\end{frame}
