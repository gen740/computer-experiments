\begin{frame}[t,fragile]{離散最適化問題への応用}
  \begin{itemize}
    \setlength{\itemsep}{1em}
  \item 微分を必要としないので、離散最適化問題にも適用可
    \begin{itemize}
    \item 例: 巡回セールスマン問題、数独、ナップザック問題
    \end{itemize}
  \item いかに状態とエネルギーを定義するかが重要
    \begin{itemize}
    \item 例: $n \times n$魔法陣 (行・列・ななめの和$M = n(n^2+1)/2$)
    \item 「状態」C: $1\sim n^2$の自然数をある順序でます目に並べたもの
    \item 「エネルギー」
      \[
      E(C) = \sum_{\rm row} (S_r-M)^2 + \sum_{\rm col} (S_c-M)^2 + \sum_{\rm diag} (S_d-M)^2
      \]
    \item 「正しい」魔方陣: $E(C) = 0$
    \end{itemize}
  \item 解の数(絶対零度のエントロピー)を求めるのにも利用できる
  \end{itemize}
\end{frame}
