\section{実習1}

\begin{frame}[t,fragile]{EX1-1: フィボナッチ数列、数値微分、ニュートン法}
  \begin{itemize}
    \setlength{\itemsep}{1em}
  \item[1-1-1] フィボナッチ数列($a_{n+2}=a_{n+1}+a_n$ ($n \ge 0$), $a_0=0$, $a_1=1$)を計算するプログラムを作成し、$a_{20}$, $a_{30}$, $a_{40}$, $a_{50}$, $a_{60}$を求めよ。桁あふれに注意すること。結果は、\LaTeX の{\tt tabular}環境を使って表にまとめよ
  \item[1-1-2] $f(x)=\sin x$について、$x=0.3\pi$における$f'(x)$の値を数値微分により計算するプログラムを作成せよ。数値微分の刻みを$h=1,1/2,1/4,1/8,\cdots$と減少させていった時、誤差がどのように振る舞うか図示せよ
  \item[1-1-3] $\sqrt[3]{x}$を求めるNewtonの反復式を書け。これを用いて、$\sqrt[3]{10}$を求めるプログラムを作成せよ。反復にしたがって、値がどのように真値に近づいていくか図示せよ
  \end{itemize}    
\end{frame}

\begin{frame}[t,fragile]{EX1-2: 代数方程式の解}
  \begin{itemize}
    \setlength{\itemsep}{1em}
  \item[1-2-1] 次数低下法を用いて、代数方程式の全ての解を求めるプログラムを作成せよ
  \item[1-2-2] Durand-Kerner-Aberth法を用いて、代数方程式の全ての解を求めるプログラムを作成せよ。方程式の次数を増やすにつれ、収束までにかかる時間がどのように増えるか調べよ
  \end{itemize}    
\end{frame}

\begin{frame}[t,fragile]{EX1-3: C言語におけるポインタと配列}
  \begin{itemize}
    \setlength{\itemsep}{1em}
  \item[1-3-1] \verb+double m[10][10];+ で宣言されたCの二次元配列について、\verb^*(m+2)[3]^ と \verb^(*(m+2))[3]^ はそれぞれどの要素の値を返すか? なぜこの2つは異なる要素の値を返すのか?
  \item[1-3-2] C言語におけるポインタの振る舞いをテストするプログラム(\href{https://github.com/todo-group/computer-experiments/blob/master/exercise/linear_system/pointer.c}{exercise/linear\_system/pointer.c})のソースコードを見て、どのような出力が生成されるか予想せよ。実際にコンパイル・実行して予想を確かめてみよ
  \end{itemize}
\end{frame}
