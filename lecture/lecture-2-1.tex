%-*- coding:utf-8 -*-

\documentclass[dvipdfmx]{beamer}
\usepackage{tutorial}

\title{計算機実験II (L1) --- 対角化と量子力学}
\date{2018/10/05}

\begin{document}

\begin{frame}
  \titlepage
  \tableofcontents
\end{frame}

\section{講義・実習の概要}

\begin{frame}[t]{講義・実習の目的}
  \begin{itemize}
    %\setlength{\itemsep}{1em}
  \item 理論・実験を問わず、学部〜大学院〜で必要となる現代的かつ普遍的な計算機の素養を身につける
  \item {\color{gray}UNIX環境に慣れる(シェル、ファイル操作、エディタ)}
  \item {\color{gray}ネットワークの活用 (リモートログイン、共同作業)}
  \item {\color{gray}プログラムの作成(C言語、コンパイラ、プログラム実行)}
  \item 基本的な数値計算アルゴリズム・数値計算の常識を学ぶ
  \item {\color{gray}科学技術文書作成に慣れる(\LaTeX, グラフ作成)}
  \item {\color{red}物理学における具体的な問題を通して実践的な知識と経験を身につける}
  \end{itemize}
\end{frame}

\begin{frame}[t]{身に付けて欲しいこと}
  \begin{itemize}
    %\setlength{\itemsep}{1em}
  \item ツールとしてないものは自分で作る (物理の伝統)
  \item すでにあるものは積極的に再利用する (車輪の再発明をしない)
  \item 数学公式と数値計算アルゴリズムは別物
  \item 刻み幅・近似度合いを変えて何度か計算を行う
  \item グラフ化して目で見てみる
  \item 計算量(コスト)のスケーリング(次数)に気をつける
  \item 記録に残す・再現性を確保する
  \item {\color{red}問題の解き方は一通りではない}
  \end{itemize}
\end{frame}

\begin{frame}[t]{講義・実習内容}
  \begin{itemize}
    \setlength{\itemsep}{1em}
  \item 問題解決型: 計算機実験Iで身に付けた知識をもとに、より高度な数値計算手法・アルゴリズムを学び、物理学における具体的な問題への応用を通して実践的な知識と経験を身につける
    \begin{itemize}
    \item 数値対角化と量子力学
    \item モンテカルロ法・分子動力学と統計物理
    \item 最適化問題
    \end{itemize}
  \item スタッフ \href{mailto:computer@exa.phys.s.u-tokyo.ac.jp}{computer@exa.phys.s.u-tokyo.ac.jp}
    \begin{itemize}
    \item 講義: 藤堂
    \item 実習: 鈴木(元早野研)、斉藤(古澤研)
    \item 実習TA: 山本(藤堂研D1)、近藤(藤堂研M1)
    \end{itemize}
  \end{itemize}    
\end{frame}

\begin{frame}[t]{質問がある場合には、、、}
  \begin{enumerate}
    %\setlength{\itemsep}{1em}
  \item ITC-LMS の掲示板を見る
  \item ハンドブック、講義資料を確認
  \item まわりの人に質問してみる
  \item ネットで検索
  \item 計算機実験担当者(\href{mailto:computer@exa.phys.s.u-tokyo.ac.jp}{computer@exa.phys.s.u-tokyo.ac.jp}) に相談
  \end{enumerate}
  メールで質問するときに注意すべきこと
  \begin{itemize}
  \item (メールの)標題をきちんとつける、きちんと名乗る
  \item 実行環境を明示する
  \item 問題を再現する手順を明記する
  \item 関連するファイル(Cや \LaTeX のソースコード等)を添付する
  \item エラーメッセージを添付する
  \end{itemize}
\end{frame}


\section{二重井戸ポテンシャル}

\begin{frame}[t,fragile]{二重井戸ポテンシャル中の粒子}
  \begin{itemize}
    %\setlength{\itemsep}{1em}
  \item 時間依存しないシュレディンガー方程式
    \begin{align*}
      \big[ -\frac{d^2}{dx^2} + V(x) \big] \psi(x) = E \psi(x)
    \end{align*}
    ($\hbar^2/2m = 1$となるように単位をとった)
  \item 二重井戸ポテンシャル
    \begin{align*}
      V(x) = \begin{cases}
        \infty & \text{$x < 0$, $x > 1$} \\
        0 & \text{$0 < x < a$, $b < x < 1$} \\
        v & \text{$a < x < b$}
      \end{cases}
    \end{align*}
    ただし、$0<a<b<1$とする
  \item 境界条件: $\psi(0) = \psi(1) = 0$、$0 < x < 1$で$\psi(x)$とその導関数が連続
  \end{itemize}
\end{frame}

\begin{frame}[t,fragile]{シュレディンガー方程式の解法}
  \begin{itemize}
    % \setlength{\itemsep}{1em}
  \item シューティング
    \begin{itemize}
    \item 計算機実験I (L2)
    \item シューティングに用いる積分法: 2階常微分方程式の2次元1階連立微分方程式への書き換え、オイラー法とその改良、Numerov法
    \end{itemize}
  \item ハミルトニアンの対角化
    \begin{itemize}
    \item 計算機実験I (L4)
    \item 対角化手法: ハウスホルダー法(LAPACK)、べき乗法、Lanczos法
    \end{itemize}
  \item 変分法: 変分関数のパラメータの最適化
  \item その他の方法: 手で解けるところはあらかじめ解いて次元を減らす
  \item それぞれのコスト(=計算時間・メモリ)は?
  \end{itemize}
\end{frame}


\section{シューティング}

\begin{frame}[t,fragile]{準備: 微分方程式の書き換え}
  \begin{itemize}
    %\setlength{\itemsep}{1em}
  \item 2階の常微分方程式の一般形
    \[
    \frac{d^2y}{dx^2} + p(x)\frac{dy}{dx} + q(x)y = r(x)
    \]
  \item $y_1 \equiv y$, $y_2 \equiv \frac{dy}{dx}$とおくと
    \[
    \left\{
    \begin{array}{ccl}
      \frac{dy_1}{dx} & = & y_2 \\
      \frac{dy_2}{dx} & = & r(x) - p(x) y_2 - q(x) y_1
    \end{array}
    \right.
    \]
  \item さらに$\bm{y}\equiv(y_1, y_2)$, $\bm{f}(x, \bm{y})\equiv \left(y_2, r(x)-p(x)y_2 - q(x)y_1\right)$
    \[
    \frac{d\bm{y}}{dx} = \bm{f}(x, \bm{y})
    \]
  \item $n$階常微分方程式 $\Rightarrow$ $n$次元の1階常微分方程式
  \end{itemize}
\end{frame}

\begin{frame}[t,fragile]{初期値問題の解法 (Euler法)}
  \begin{itemize}
    %\setlength{\itemsep}{1em}
  \item $h$を微小量として微分を差分で近似する(前進差分)
    \[
    \frac{dy}{dt} \approx \frac{y(t+h) - y(t)}{h} = f(t, y)
    \]
  \item $t=0$における$y(t)$の初期値を$y_0$、$t_n \equiv nh$、$y_n$を$y(t_n)$の近似値とおくと、
    \[
    y_{n+1}-y_n = h f( t_n, y_n)
    \]
  \item Euler法
    \begin{itemize}
    \item $y_0$からはじめて、$y_1,y_2,\cdots$を順次求めていく
    \end{itemize}
  \end{itemize}
\end{frame}

\begin{frame}[t,fragile]{高次のRunge-Kutta法}
  \begin{itemize}
    %\setlength{\itemsep}{1em}
  \item 3次Runge-Kutta法
    \[
    \begin{array}{rcl}
      k_1 & = & h f(t_n, y_n) \\
      k_2 & = & h f(t_n + \frac{2}{3}h, y_n + \frac{2}{3}k_1) \\
      k_3 & = & h f(t_n + \frac{2}{3}h, y_n + \frac{2}{3}k_2) \\
      y_{n+1} & = & y_n + \frac{1}{4}k_1 + \frac{3}{8}k_2
      + \frac{3}{8}k_3
    \end{array}
    \]
  \item 4次Runge-Kutta法
    \[
    \begin{array}{rcl}
      k_1 & = & h f(t_n, y_n) \\
      k_2 & = & h f(t_n + \frac{1}{2}h, y_n + \frac{1}{2}k_1) \\
      k_3 & = & h f(t_n + \frac{1}{2}h, y_n + \frac{1}{2}k_2) \\
      k_4 & = & h f(t_n + h, y_n + k_3) \\
      y_{n+1} & = & y_n + \frac{1}{6}k_1 + \frac{1}{3}k_2
      + \frac{1}{3}k_3 + \frac{1}{6}k_4
    \end{array}
    \]
  \item 4次までは次数と$f$の計算回数が等しい
  \end{itemize}
\end{frame}

\begin{frame}[t,fragile]{計算コストと精度}
  \begin{itemize}
    %\setlength{\itemsep}{1em}
  \item 実際の計算では$f(t,y)$の計算にほとんどのコストがかかる
  \item 計算回数と計算精度の関係
    \begin{center}
      \begin{tabular}[h]{c|cccc}
        & 1次(Euler法) & 2次(中点法) & 3次 & 4次 \\
        \hline
        計算精度 & $O(h)$ & $O(h^2)$ & $O(h^3)$ & $O(h^4)$ \\
        計算回数 & $N$ & $2N$ & $3N$ & $4N$
      \end{tabular}
    \end{center}
  \item 高次のRunge-Kuttaを使う方が効率的
  \item どれくらい小さな$h$が必要となるか、前もっては分からない
  \item 刻み幅を変えて($h,h/2,h/4,\dots$)計算してみることが大事
    \begin{itemize}
    \item 誤差の評価
    \item 公式の間違いの発見
    \end{itemize}
  \end{itemize}
\end{frame}

\begin{frame}[t,fragile]{Numerov法}
  \begin{itemize}
    %\setlength{\itemsep}{1em}
  \item Numerov法
    \begin{itemize}
    \item 二階の常微分方程式で一階の項がない場合に使える
    \item 連立微分方程式に直さずに直接二階微分方程式を解く
    \item 4次の陰解法
    \item 方程式が線形の場合は陽解法に書き直せる
    \end{itemize}
  \item 微分方程式
    \[
    \frac{d^2y}{dx^2} = f(x,y)
    \]
  $y=y(x)$を$x=x_i$のまわりでテイラー展開する。$x_{i \pm 1} = x_i \pm h$での表式は
      \[
      y(x_{i \pm 1}) = y(x_i) \pm h y'(x_i) + \frac{h^2}{2} y''(x_i) \pm \frac{h^3}{6} y'''(x_i) + \frac{h^4}{24} y''''(x_i)  + O(h^5)
      \]
  \end{itemize}
\end{frame}

\begin{frame}[t,fragile]{Numerov法}
  \begin{itemize}
    \setlength{\itemsep}{1em}
  \item 二階微分の差分近似 ($y_i \equiv y(x_i)$等と書く)
    \[
    \frac{y_{i+1} - 2 y_i + y_{i-1}}{h^2} = y''_{i} + \frac{h^2}{12} y''''_{i} + O(h^4)
    \]
  一方で、微分方程式より
    \[
    y''''_i = \frac{d^2f}{dx^2}\Big|_{x=x_i} = \frac{f_{i+1}-2f_i+f_{i-1}}{h^2} + O(h^2)
    \]
    組み合わせると
    \[
    y_{i+1} = 2y_i - y_{i-1} + \frac{h^2}{12} (f_{i+1} + 10f_{i} + f_{i-1}) + O(h^6)
    \]
  \end{itemize}
\end{frame}

\begin{frame}[t,fragile]{Numerov法}
  \begin{itemize}
    %\setlength{\itemsep}{1em}
  \item 方程式が線形の場合、$f(x,y) = -a(x) y(x)$を代入すると
    \[
    y_{i+1} = 2y_i - y_{i-1} - \frac{h^2}{12} (a_{i+1}y_{i+1} + 10a_{i}y_{i} + a_{i-1}y_{i-1}) + O(h^6)
    \]
  $y_{i+1}$を左辺に集めると、陽解法となる
    \[
    y_{i+1} = \frac{2 (1-\frac{5h^2}{12} a_i)y_i - (1 + \frac{h^2}{12} a_{i-1}) y_{i-1}}{1 + \frac{h^2}{12} a_{i+1}} + O(h^6)
    \]
  \end{itemize}
\end{frame}

\begin{frame}[t,fragile]{ポアソン方程式の境界値問題}
  \begin{itemize}
    %\setlength{\itemsep}{1em}
  \item 二次元ポアソン方程式
    \[ \frac{\partial^2 u(x,y)}{\partial x^2} + \frac{\partial^2 u(x,y)}{\partial y^2} = f(x,y) \qquad 0 \le x \le 1, \ 0 \le y \le 1\]
  \item ディリクレ型境界条件: $u(x,y) = g(x,y)$ on $\partial \Omega$
  \item 有限差分法により離散化
    \begin{itemize}
    \item $x$方向、$y$方向をそれぞれ$n$等分: $(x_i,y_j) = (i/n, j/n)$
    \item $(n+1)^2$個の格子点の上で$u(x_i,y_j)=u_{ij}$が定義される
    \item そのうち$4n$個の値は境界条件で定まる
    \item ポアソン方程式を中心差分で近似 ($h=1/n$)
      \[
      \frac{u_{i+1,j}-2u_{ij}+u_{i-1,j}}{h^2} + \frac{u_{i,j+1}-2u_{ij}+u_{i,j-1}}{h^2} = f_{ij}
      \]
      残り$(n-1)^2$個の未知数に対する連立一次方程式
    \end{itemize}
  \end{itemize}
\end{frame}

% -*- coding: utf-8 -*-

\section{二分法}

\begin{frame}[t,fragile]{二分法}
  \begin{itemize}
    % \setlength{\itemsep}{1em}
  \item 反復法により一次元の方程式$f(x)=0$の解を求める
  \item 導関数を使わず関数値のみを利用 (c.f. ニュートン法)
  \item 初期条件として、$f(a) \times f(b) < 0$を満たす2点の組($a<b$)で解をはさみ込み、領域を狭めていく
  \item $a$と$b$の中点$x=(a+b)/2$を考える
    \begin{itemize}
    \item $|f(x)|$が十分小さい場合: $x$が解
    \item $f(a) \times f(x) < 0$の場合: $[a,x]$を新しい領域にとる
    \item $f(x) \times f(b) < 0$の場合: $[x,b]$を新しい領域にとる
    \end{itemize}
  \item 領域$[a,b]$の幅が十分小さくなったら終了
  \item 反復のたびに領域の幅は半分になる
  \item 全ての解を得られる保証はない
  \item 二分法の例: \href{https://github.com/todo-group/computer-experiments/blob/master/exercise/basics/bisection.c}{bisection.c}
  \end{itemize}
\end{frame}



\section{対角化による解法}

\begin{frame}[t,fragile]{対角化}
  \begin{itemize}
    \setlength{\itemsep}{1em}
  \item シュレディンガー方程式の行列表示
  \item ハウスホルダー法(LAPACK)
  \item べき乗法
  \item Lanczos法
  \item ハウスホルダー法によるプログラムの例: \href{https://github.com/todo-group/computer-experiments/blob/master/exercise/eigenvalue_problem/double_well.c}{example-2-L1/double\_well.c}
  \end{itemize}
\end{frame}

\begin{frame}[t,fragile]{シュレディンガー方程式の行列表示}
  \begin{itemize}
    %\setlength{\itemsep}{1em}
  \item シュレディンガー方程式
    \[
    [-\frac{d^2}{dx^2}+V(x)]\psi(x) = E \psi(x)
    \]
  \item 連立差分方程式を行列の形で表す($\psi(x_0)=\psi(x_n)=0$)
    \begin{footnotesize}
    \[
    \begin{pmatrix}
      \frac{2}{h^2}+V(x_1) & -\frac{1}{h^2} \\
      -\frac{1}{h^2} & \frac{2}{h^2}+V(x_2) & -\frac{1}{h^2} \\
      & -\frac{1}{h^2} & \frac{2}{h^2}+V(x_3) & -\frac{1}{h^2} \\
      & & \ddots & \ddots \\
      & & & -\frac{1}{h^2} & \frac{2}{h^2}+V(x_{n-1}) \\
    \end{pmatrix}
    \begin{pmatrix}
      \psi(x_1) \\
      \psi(x_2) \\
      \psi(x_3) \\
      \vdots \\
      \psi(x_{n-1}) \\
    \end{pmatrix}
    = \cdots % E
    %% \begin{pmatrix}
    %%   \psi(x_1) \\
    %%   \psi(x_2) \\
    %%   \psi(x_3) \\
    %%   \vdots \\
    %%   \psi(x_{n-1}) \\
    %% \end{pmatrix}
    \]
    \end{footnotesize}
  \item $(n-1) \times (n-1)$の疎行列の固有値問題
    \begin{itemize}
    \item 固有値: 固有エネルギー
    \item 固有ベクトル: 波動関数
    \end{itemize}
  \end{itemize}
\end{frame}

\begin{frame}[t,fragile]{行列の数値対角化}
  \begin{itemize}
    %\setlength{\itemsep}{1em}
  \item 一般的に次元が5以上の行列の固有値は、あらかじめ定まる有限回の手続きでは求まらない
    \begin{itemize}
    \item 必ず何らかの反復法(+収束判定)が必要となる
    \end{itemize}
  \item 密行列向きの方法
    \begin{itemize}
    \item Jacobi法
    \item Givens変換・Householder法(三重対角化) + QR法など
    \end{itemize}
  \item 疎行列向きの方法
    \begin{itemize}
    \item べき乗法
    \item Lanczos法(三重対角化) + QR法など
    \end{itemize}
  \item 固有ベクトル
    \begin{itemize}
    \item QR法で求めたものを逆変換
    \item 逆反復法で精度改善
    \end{itemize}
  \end{itemize}
\end{frame}

\begin{frame}[t,fragile]{基本方針}
  \begin{itemize}
    %\setlength{\itemsep}{1em}
  \item やってはいけない方法: 特性方程式
    \[
    |\lambda E - A| = 0
    \]
    の係数を求めて、代数方程式として解く
    \begin{itemize}
    \item 数値的に不安定 (代数方程式の解は係数の誤差に対して敏感)
    \item 計算コスト大[$\sim O(N!)$]
    \end{itemize}
  \item スタンダードな方法: 行列を次々に直交変換して、対角行列(あるいは三重対角行列)に近づけていく
    \[
    A \rightarrow U_1^T A U_1 \rightarrow U_2^T (U_1^T A U_1) U_2 \rightarrow U_3^T (U_2^T (U_1^T A U_1) U_2) U_3 \rightarrow \cdots
    \]
  \item 固有値は変換された行列の固有値、固有ベクトルは変換後の行列の固有ベクトルに左から$U_1 U_2 U_3 \cdots$を掛けたもの
  \end{itemize}
\end{frame}

\begin{frame}[t,fragile]{LAPACKの対角化ルーチン}
  \begin{itemize}
    %\setlength{\itemsep}{1em}
  \item 様々な対角化ルーチンが準備されている
    \begin{itemize}
    \item 倍精度実対称行列の対角化 {\tt dsyev}
      \url{http://www.netlib.org/lapack/explore-html/dd/d4c/dsyev_8f.html}
    \item Fortranによる関数宣言
\begin{lstlisting}
subroutine dsyev(character JOBZ, character UPLO,
  integer N, double precision, dimension(lda, *) A,
  integer LDA, double precision, dimension(*) W,
  double precision, dimension(*) WORK,
  integer LWORK, integer INFO)		
\end{lstlisting}
    \end{itemize}
  \item 他にも{\tt dsyevd}、{\tt dsyevr}、{\tt dsyevx}などがある \\
    3重対角化までは同じ。3重対角行列の対角化が異なる
  \item 単精度版の{\tt ssyev}、複素(エルミート行列)版の{\tt zheev}など
  \item {\tt dsyev}の使用例: \href{https://github.com/todo-group/computer-experiments/blob/master/exercise/eigenvalue_problem/diag.c}{diag.c}
  \end{itemize}
\end{frame}

\begin{frame}[t,fragile]{CからBLAS/LAPACKを呼び出す際の注意事項}
  \begin{itemize}
    %\setlength{\itemsep}{1em}
  \item (もともとFortran言語で書かれていたことによる制限)
  \item 関数名はすべて小文字、最後に \verb+_+ (下線)を付ける
  \item スカラー、ベクトル、行列は全て「ポインタ渡し」とする
  \item ベクトルや行列は最初の要素へのポインタを渡す (サイズは別に渡す)
  \item 行列の要素は(0,0) $\rightarrow$ (1,0) $\rightarrow$ (2,0) $\rightarrow\cdots\rightarrow$ $(m-1,0)$ $\rightarrow$ (0,1) $\rightarrow$ (1,1) $\rightarrow\cdots\rightarrow$ $(m-1,n-1)$の順で連続して並んでいなければならない(column-major)
    \begin{itemize}
    \item C言語の二次元配列では \verb+a[i][j]+ の次には \verb%a[i][j+1]%が入っている(row-major)
    \item 行列が転置されて解釈されてしまう!
    \end{itemize}
  \item コンパイル時には{\tt -llapack -lblas}オプションを指定し、LAPACKライブラリとBLASライブラリをリンクする(ハンドブック2.1.6節)
  \end{itemize}
\end{frame}

\begin{frame}[t,fragile]{cmatrix.hライブラリ}
  \begin{itemize}
    %\setlength{\itemsep}{1em}
  \item Column-major形式の二次元配列の確保({\tt alloc\_dmatrix})、開放({\tt free\_dmatrix})、出力({\tt print\_dmatrix})、読み込み({\tt read\_dmatrix})を行うためのユーティリティ関数、(i,j)成分にアクセスするためのマクロ({\tt mat\_elem})他を準備
  \item ソースコード: \href{https://github.com/todo-group/computer-experiments/blob/master/exercise/matrix/cmatrix.h}{cmatrix.h}
  \item 使用例
\begin{lstlisting}
#include "cmatrix.h"
...
double **mat;
mat = alloc_dmatrix(m, n);
mat_elem(mat, 1, 3) = 5.0;
...
free_dmatrix(mat);
\end{lstlisting}
  \item サンプルコード: \href{https://github.com/todo-group/computer-experiments/blob/master/exercise/matrix/matrix_example.c}{matrix\_example.c}
  \end{itemize}
\end{frame}

\begin{frame}[t,fragile]{alloc\_dmatrixでの動的二次元配列の確保}
  \begin{itemize}
    %\setlength{\itemsep}{1em}
  \item 長さ$m \times n$の一次元配列を用意し、各列(それぞれ$m$要素)の先頭アドレスを長さ$n$のポインター配列に格納する (ハンドブック2.12.3節)
\begin{lstlisting}
double **a;
m = 10;  
n = 10;  
a = (double**)malloc((size_t)(n * sizeof(double*));
a[0] = (double*)malloc((size_t)(m*n * sizeof(double));
for (int i = 1; i < n; ++i)
  a[i] = a[i-1] + m;
\end{lstlisting}
\item 行列の(i,j)成分を\verb+a[j][i]+に格納することにする (column-major)
  \end{itemize}
\end{frame}

\begin{frame}[t,fragile]{要素アクセス・先頭アドレス}
  \begin{itemize}
    % \setlength{\itemsep}{1em}
  \item 行列の(i,j)成分は\verb+a[j][i]+に格納されている
    \begin{itemize}
      \item \href{https://github.com/todo-group/computer-experiments/blob/master/exercise/matrix/cmatrix.h}{cmatrix.h}ではマクロ(\verb+mat_elem+)を準備
\begin{lstlisting}
#define mat_elem(mat, i, j) (mat)[j][i]
\end{lstlisting}
\item このマクロを使うと、例えば(i,j)成分への代入は以下のように書ける
\begin{lstlisting}
mat_elem(a, i, j) = 1;
\end{lstlisting}
\end{itemize}
  \item LAPACKにベクトルや行列の最初の要素へのポインタを渡す
    \begin{itemize}
      \item ベクトルの最初の要素(0)へのポインタ: \verb+&v[0]+
      \item 行列の最初の要素(0,0)へのポインタ: \verb+&a[0][0]+
      \item \href{https://github.com/todo-group/computer-experiments/blob/master/exercise/matrix/cmatrix.h}{cmatrix.h}にマクロ({\tt vec\_ptr}、{\tt mat\_ptr})が準備されているのでそれぞれ、{\tt vec\_ptr(v)}、{\tt mat\_ptr(a)}と書ける
    \end{itemize}
  \end{itemize}
\end{frame}

\begin{frame}[t,fragile]{反復法}
  \begin{itemize}
    %\setlength{\itemsep}{1em}
  \item 疎行列の場合、行列ベクトル積は高速に行える
  \item Givens変換、Householder変換などを行うと疎行列性が失われる
  \item 行列ベクトル積のみを用いる反復法が効果的
    \begin{itemize}
    \item べき乗法
    \item Lanczos法
    \end{itemize}
  \end{itemize}
\end{frame}

\begin{frame}[t,fragile]{べき乗法(Power Method)}
  \begin{itemize}
    %\setlength{\itemsep}{1em}
  \item 適当なベクトル$v_1$から出発する
  \item $v_1$が最大固有ベクトル$\xi_1$と直交していないとすると
    \[
    v_1 = c_1 \xi_1 + c_2 \xi_2 + c_3 \xi_3 + \cdots + c_N \xi_N
    \]
    と展開できる($c_1 \ne 0$)。この両辺に$A$を次々掛けて行くと
    \begin{align*}
      v_2 = A v_1 &= c_1 \lambda_1 \xi_1 + c_2 \lambda_2 \xi_2 + c_3 \lambda_3 \xi_3 + \cdots + c_N \lambda_N \xi_N \\
      v_3 = A^2 v_1 &= c_1 \lambda_1^2 \xi_1 + c_2 \lambda_2^2 \xi_2 + c_3 \lambda_3^2 \xi_3 + \cdots + c_N \lambda_N^2 \xi_N \\
      \vdots \\
      v_{n+1} = A^n v_1 &= c_1 \lambda_1^n \xi_1 + c_2 \lambda_2^n \xi_2 + c_3 \lambda_3^n \xi_3 + \cdots + c_N \lambda_N^n \xi_N \\
      &= c_1 \lambda_1^n \Big[ \xi_1 + \sum_{k=2}^N \frac{c_k}{c_1} \big( \frac{\lambda_k}{\lambda_1}\big)^n \xi_k \Big] \approx c_1 \lambda_1^n \xi_1 \\
    \end{align*}
  \end{itemize}
\end{frame}

\begin{frame}[t,fragile]{べき乗法の収束}
  \begin{itemize}
    %\setlength{\itemsep}{1em}
  \item べき乗法による固有値
    \[
    \frac{v_{k+1}^T v_{k+1}}{v_{k+1}^T v_k} = \lambda_1 + O\Big( \big(\frac{\lambda_2}{\lambda_1} \big)^{2k}\Big)
    \]
  \item 誤差の収束
    \[
    \frac{v_{k+1}^T v_{k+1}}{v_{k+1}^T v_k} \approx \lambda_1 + e^{-2k \ln (\lambda_1/\lambda_2)}
    \]
  \item $1 / \ln (\lambda_1/\lambda_2)$ 程度の反復が必要
  \item $\lambda_1$と$\lambda_2$が近い場合には、反復回数が非常に多くなる
  \end{itemize}
\end{frame}

\begin{frame}[t,fragile]{Lanczos法}
  \begin{itemize}
    \setlength{\itemsep}{1em}
  \item 初期(ランダム)ベクトル$v_1$に加えて
    \[
    Av_1, Av_1, \cdots A^{m-1}v_1
    \]
    を正規直交化して$v_1,v_2,\cdots,v_m$を作る(Krylov部分空間)
  \item 部分空間でのRitz値を固有値の近似値とする
  \item $A^kv_1$はどんどん最大固有ベクトルに近づいていくので、$m \ll n$でも良い近似固有値が得られると期待される
  \end{itemize}
\end{frame}

\begin{frame}[t,fragile]{Lanczos法}
  \begin{itemize}
    %\setlength{\itemsep}{1em}
  \item 正規化された初期(ランダム)ベクトル$v_1$から出発する %($v_0=0$とする)
  \item $v_2,v_3,\cdots$を生成する
    \begin{align*}
      v_2 &= (Av_1 - \alpha_1 v_1)/\beta_1 \\
      v_3 &= (Av_2 - \beta_1 v_1 - \alpha_2 v_2)/\beta_2 \\
      \vdots
    \end{align*}
    ここで
    \begin{align*}
      \alpha_i &= v_i^T A v_i \\
      \beta_i &= | A v_i - \beta_{i-1} v_{i-1} - \alpha_i v_i |, \ \beta_0 = 0
    \end{align*}
    と選ぶ
  \end{itemize}
\end{frame}

\begin{frame}[t,fragile]{Lanczos法}
  \begin{itemize}
    % \setlength{\itemsep}{1em}
  \item $v_1,v_2,v_3,\cdots,v_{m+1}$は正規直交
  \item 漸化式を書き換えると
    \begin{align*}
      Av_1 &= \alpha_1 v_1 + \beta_1 v_2 \\
      Av_2 &= \beta_1 v_1 + \alpha_2 v_2 + \beta_2 v_3 \\
      Av_3 &= \beta_2 v_2 + \alpha_3 v_3 + \beta_3 v_4 \\
      \vdots \\
      Av_{m} &= \beta_{m-1} v_{m-1} + \alpha_m v_m + \beta_m v_{m+1}
    \end{align*}
  \end{itemize}
\end{frame}

\begin{frame}[t,fragile]{Lanczos法}
  \begin{itemize}
    \setlength{\itemsep}{1em}
  \item 行列で表現すると
    \begin{align*}
      \hspace*{-2em}
      A
      (v_1v_2\cdots v_M)
      &=
      (v_1v_2\cdots v_M v_{M+1})
      \begin{pmatrix}
        \alpha_1 & \beta_1\\
        \beta_1 & \alpha_2 & \beta_2 \\
        & \beta_2 & \alpha_3 & \beta_3 \\
        & & \beta_3 & \alpha_4 & \beta_4 \\
        & & & \ddots & \ddots & \ddots \\
        & & & & \beta_{M-1} & \alpha_M \\
        & & & & & \beta_M \\
      \end{pmatrix}
    \end{align*}
    両辺に左から$(v_1v_2\cdots v_M)^T$をかけると
    \[
    (v_1v_2\cdots v_M)^T A (v_1v_2\cdots v_M)
    \]
    は3重対角行列となることがわかる
  \end{itemize}
\end{frame}

\begin{frame}[t,fragile]{Lanczos法}
  \begin{itemize}
    %\setlength{\itemsep}{1em}
  \item 原理的には、$N$ステップ目で$\beta_N=0$となり、3重対角化が完了する
  \item 実際には、数値誤差のため$v_1,v_2,v_3\cdots$の直交性が崩れる
    \begin{itemize}
      \item $M$を大きくしすぎると、おかしな固有値が出てくる
      \item 全ての固有値が欲しい場合にはHouseholder法を使う
    \end{itemize}
  \item Lanczos法では、大きな固有値に対応する固有ベクトルにできるだけ近いものから部分空間を作っていく
    \begin{itemize}
      \item 100万次元以上の行列の場合でも$M=100 \sim 200$程度で最初の数個の固有値は精度良く求まる
    \end{itemize}
  \item 必要な操作は、行列とベクトルの積、ベクトルの内積・スケーリング・和のみ
    \begin{itemize}
      \item 疎行列の場合、非常に効率が良い
    \end{itemize}
  \end{itemize}
\end{frame}


\section{解析計算による次元削減}

\begin{frame}[t,fragile]{シュレディンガー方程式の一般解}
  \begin{itemize}
    %\setlength{\itemsep}{1em}
  \item 二重井戸ポテンシャル
    \begin{align*}
      V(x) = \begin{cases}
        \infty & \text{$x < 0$, $x > 1$} \\
        0 & \text{$0 < x < a$, $b < x < 1$} \\
        v & \text{$a < x < b$}
      \end{cases}
    \end{align*}
  \item それぞれの領域内では手で解ける
  \item 領域1 ($0 < x < a$)、領域3 ($b < x < 1$)では
    \begin{align*}
      -\frac{d^2}{dx^2}\psi(x) = E\psi(x)
    \end{align*}
  \end{itemize}
\end{frame}

\begin{frame}[t,fragile]{シュレディンガー方程式の一般解}
  \begin{itemize}
    %\setlength{\itemsep}{1em}
  \item 領域1 ($0 < x < a$)、領域3 ($b < x < 1$)における一般解
    \begin{align*}
      \psi(x) &= A_1 e^{i\sqrt{E}x} + B_1 e^{-i\sqrt{E}x} \\
      \psi(x) &= A_3 e^{i\sqrt{E}x} + B_3 e^{-i\sqrt{E}x}
    \end{align*}
    あきらかに$E>0$であるので
    \begin{align*}
      \psi(x) &= \alpha_1 \cos(\sqrt{E}x) + \beta_1 \sin(\sqrt{E}x) \\
      \psi(x) &= \alpha_3 \cos(\sqrt{E}x) + \beta_3 \sin(\sqrt{E}x)
    \end{align*}
  \end{itemize}
\end{frame}

\begin{frame}[t,fragile]{シュレディンガー方程式の一般解}
  \begin{itemize}
    %\setlength{\itemsep}{1em}
  \item 領域2 ($a < x < b$)における一般解
    \begin{align*}
      \psi(x) &= A_2 e^{i\sqrt{(E-v)}\,x} + B_1 e^{-i\sqrt{(E-v)}\,x}
    \end{align*}
  \item $E < v$の場合
    \begin{align*}
      \psi(x) &= \alpha_2 \exp(-\sqrt{(v-E)}\,x) + \beta_2 \exp(\sqrt{(v-E)}\,x)
    \end{align*}
  \item $E > v$の場合
    \begin{align*}
      \psi(x) &= \alpha_2 \cos(\sqrt{(E-v)}\,x) + \beta_2 \sin(\sqrt{(E-v)}\,x)
    \end{align*}
  \end{itemize}
\end{frame}

\begin{frame}[t,fragile]{シュレディンガー方程式の一般解}
  \begin{itemize}
    %\setlength{\itemsep}{1em}
  \item 境界条件($E < v$の場合)
    \begin{align*}
      &\alpha_1 = 0 \\
      &\alpha_1 \cos(\sqrt{E}a) + \beta_1 \sin(\sqrt{E}a) \\
      & \qquad =
      \alpha_2 \exp(-\sqrt{(v-E)}\,a) + \beta_2 \exp(\sqrt{(v-E)}\,a) \\
      &-\alpha_1 \sqrt{E} \sin(\sqrt{E}a) + \beta_1 \sqrt{E} \sin(\sqrt{E}a) \\
      & \qquad =
      - \alpha_2 \sqrt{(v-E)} \exp(-\sqrt{(v-E)}\,a) + \beta_2 \sqrt{(v-E)} \exp(\sqrt{(v-E)}\,a) \\
      &\cdots
    \end{align*}
  \item $\beta_1$, $\alpha_2$, $\beta_2$, $\alpha_3$, $\beta_3$ に関する連立方程式: $M x = 0$
  \item $5 \times 5$行列$M$は$E$の(非線形な)関数
  \item 非自明な解が存在するための条件: $\det M=0$
  \end{itemize}
\end{frame}


\begin{frame}[t,fragile]{行列式の計算}
  \begin{itemize}
    \setlength{\itemsep}{1em}
  \item ガウスの消去法とLU分解
  \item LU分解による行列式の計算
  \item LAPACKによるLU分解
  \item LU分解のプログラムの例: \href{https://github.com/todo-group/computer-experiments/blob/master/exercise/linear_system/lu_decomp.c}{example-2-L1/lu\_decomp.c}
  \item 行列式計算のプログラムの例: \href{https://github.com/todo-group/computer-experiments/blob/master/exercise/linear_system/determinant.c}{example-2-L1/determinant.c}
  \end{itemize}
\end{frame}

\begin{frame}[t,fragile]{逆行列の「間違った」求め方}
  \begin{itemize}
    \setlength{\itemsep}{1em}
  \item 線形代数の教科書に載っている公式
    \[
    A^{-1} = \frac{\tilde{A}}{|A|}
    \]
    $|A|$: $A$の行列式、$\tilde{A}$: $A$の余因子行列
  \item $n \times n$行列の行列式を定義通り計算すると、計算量〜$O(n!)$
  \item したがって、上の方法で逆行列を計算すると、計算量〜$O(n!)$
  \item $n=100$の場合: $n! \approx 9.3 \times 10^{157}$
  \end{itemize}
\end{frame}

\begin{frame}[t,fragile]{逆行列の「正しい」求め方}
  \begin{itemize}
    \setlength{\itemsep}{1em}
  \item 連立一次方程式 $A {\bf x} = {\bf e}_j$ を全ての${\bf e}_j$について解く
  \item Gaussの消去法による連立一次方程式の解法: 計算量〜$O(n^3)$
  \item Gaussの消去法の途中で出てくる下三角行列(L)と上三角行列(U)行列を再利用(LU分解)すれば、逆行列全体を求めるための計算量も$O(n^3)$
  \item 行列式も$O(n^3)$で計算可
  \item $n=100$の場合: $n^3 = 10^6 \ll 9.3 \times 10^{157}$
  \end{itemize}
\end{frame}

\begin{frame}[t,fragile]{ガウスの消去法}
  \begin{itemize}
    \setlength{\itemsep}{1em}
  \item 解くべき連立方程式
    \begin{align*}
    a_{11}^{(1)} x_1 + a_{12}^{(1)} x_2 + a_{13}^{(1)} x_3 + \cdots + a_{1n}^{(1)} x_n &= b_{1}^{(1)} \\
    a_{21}^{(1)} x_1 + a_{22}^{(1)} x_2 + a_{23}^{(1)} x_3 + \cdots + a_{2n}^{(1)} x_n &= b_{2}^{(1)} \\
    a_{31}^{(1)} x_1 + a_{32}^{(1)} x_2 + a_{33}^{(1)} x_3 + \cdots + a_{3n}^{(1)} x_n &= b_{3}^{(1)} \\
    \cdots \\
    a_{n1}^{(1)} x_1 + a_{n2}^{(1)} x_2 + a_{n3}^{(1)} x_3 + \cdots + a_{nn}^{(1)} x_n &= b_{n}^{(1)}
    \end{align*}
  \item ある行を定数倍しても、方程式の解は変わらない
  \item ある行の定数倍を他の行から引いても、方程式の解は変わらない
  \end{itemize}
\end{frame}

\begin{frame}[t,fragile]{ガウスの消去法}
  \begin{itemize}
    %\setlength{\itemsep}{1em}
  \item 1行目を$m_{i1} = a_{i1}^{(1)}/a_{11}^{(1)}$倍して、$i$行目($i \ge 2$)から引く
    \begin{align*}
    a_{11}^{(1)} x_1 + a_{12}^{(1)} x_2 + a_{13}^{(1)} x_3 + \cdots + a_{1n}^{(1)} x_n &= b_{1}^{(1)} \\
    a_{22}^{(2)} x_2 + a_{23}^{(2)} x_3 + \cdots + a_{2n}^{(2)} x_n &= b_{2}^{(2)} \\
    a_{32}^{(2)} x_2 + a_{33}^{(2)} x_3 + \cdots + a_{3n}^{(2)} x_n &= b_{3}^{(2)} \\
    \cdots \\
    a_{n2}^{(2)} x_2 + a_{n3}^{(2)} x_3 + \cdots + a_{nn}^{(2)} x_n &= b_{n}^{(2)}
    \end{align*}
  \item ここで
    \begin{align*}
      a_{ij}^{(2)} &= a_{ij}^{(1)} - m_{i1} a_{1j}^{(1)} \qquad i \ge 2, j \ge 2 \\
      b_{i}^{(2)} &= b_{i}^{(1)} - m_{i1} b_{1}^{(1)} \qquad i \ge 2
    \end{align*}
  \end{itemize}
\end{frame}

\begin{frame}[t,fragile]{ガウスの消去法}
  \begin{itemize}
    %\setlength{\itemsep}{1em}
  \item 2行目を$m_{i2} = a_{i2}^{(2)}/a_{22}^{(2)}$倍して、$i$行目($i \ge 3$)から引く
    \begin{align*}
    a_{11}^{(1)} x_1 + a_{12}^{(1)} x_2 + a_{13}^{(1)} x_3 + \cdots + a_{1n}^{(1)} x_n &= b_{1}^{(1)} \\
    a_{22}^{(2)} x_2 + a_{23}^{(2)} x_3 + \cdots + a_{2n}^{(2)} x_n &= b_{2}^{(2)} \\
    a_{33}^{(3)} x_3 + \cdots + a_{3n}^{(3)} x_n &= b_{3}^{(3)} \\
    \cdots \\
    a_{n3}^{(3)} x_3 + \cdots + a_{nn}^{(3)} x_n &= b_{n}^{(3)}
    \end{align*}
  \item ここで
    \begin{align*}
      a_{ij}^{(3)} &= a_{ij}^{(2)} - m_{i2} a_{2j}^{(2)} \qquad i \ge 3, j \ge 3 \\
      b_{i}^{(3)} &= b_{i}^{(2)} - m_{i2} b_{2}^{(2)} \qquad i \ge 3
    \end{align*}
  \end{itemize}
\end{frame}

\begin{frame}[t,fragile]{ガウスの消去法}
  \begin{itemize}
    \setlength{\itemsep}{1em}
  \item 最終的には、左辺が右上三角形をした連立方程式となる
    \begin{align*}
    a_{11}^{(1)} x_1 + a_{12}^{(1)} x_2 + a_{13}^{(1)} x_3 + \cdots + a_{1n}^{(1)} x_n &= b_{1}^{(1)} \\
    a_{22}^{(2)} x_2 + a_{23}^{(2)} x_3 + \cdots + a_{2n}^{(2)} x_n &= b_{2}^{(2)} \\
    a_{33}^{(3)} x_3 + \cdots + a_{3n}^{(3)} x_n &= b_{3}^{(3)} \\
    \cdots \\
    a_{n-1,n-1}^{(n-1)} x_{n-1} + a_{n-1,n}^{(n-1)} x_n &= b_{n-1}^{(n-1)} \\
    a_{nn}^{(n)} x_n &= b_{n}^{(n)}
    \end{align*}
  \item これを下から順に解いていけばよい(後退代入)
  \end{itemize}
\end{frame}

\begin{frame}[t,fragile]{ピボット選択}
  \begin{itemize}
    %\setlength{\itemsep}{1em}
  \item ガウスの消去法の途中で$a_{kk}^{(k)}$が零になると、計算を先に進めることができなくなる
  \item 行を入れ替えても、方程式の解は変わらない $\Rightarrow$ $k$行以降で、$a_{ik}^{(k)}$が非零の行と入れ替える (ピボット選択)
  \item 実際のコードでは、情報落ちを防ぐため、$a_{kk}^{(k)}$が零でない場合でも、$a_{ik}^{(k)}$の絶対値が最大の行と入れ替える
  \item ピボット選択が必要となる例: \href{https://github.com/todo-group/computer-experiments/blob/master/exercise/linear_system/input2.dat}{input2.dat}
    \begin{align*}
      \begin{bmatrix} 1 & 4 & 7 \\ 2 & 8 & 5 \\ 3 & 6 & 10 \end{bmatrix} \begin{bmatrix} x_1 \\ x_2 \\ x_3 \end{bmatrix} = \begin{bmatrix} 30 \\ 33 \\ 45 \end{bmatrix}
    \end{align*}
    \item 行列がrank落ちしている場合は、ピボット選択を行っても途中で0になる (cf. 特異値分解を用いた最小二乗解)
  \end{itemize}
\end{frame}

\begin{frame}[t,fragile]{ガウスの消去法の行列表示}
  \begin{itemize}
    %\setlength{\itemsep}{1em}
  \item $a_{kk}^{(k)}$を用いた$a_{ik}^{(k)}$ ($i>k$)の消去は、方程式の両辺に左から
    \begin{align*}
      M_k = 
      \begin{bmatrix}
        1 & \\
        0 & 1 \\
        0 & 0 & \ddots \\
        \vdots & \vdots & & 1 \\
        \vdots & \vdots & & -m_{k+1,k} & 1 & \\
        \vdots & \vdots & & -m_{k+2,k} & 0 & \ddots \\
        \vdots & \vdots & & \vdots & \vdots & & 1 & \\
        0 & 0 & \hdots & -m_{nk} & 0 & \hdots & 0 & 1
      \end{bmatrix}
    \end{align*}
    を掛けるのと等価: $M_k A^{(k)} = A^{(k+1)}$、$M_k {\bf b}^{(k)} = {\bf b}^{(k+1)}$
  \end{itemize}
\end{frame}

\begin{frame}[t,fragile]{LU分解}
  \begin{itemize}
    %\setlength{\itemsep}{1em}
  \item $M_k$の逆行列
    \begin{align*}
      L_k = M_k^{-1} = 
      \begin{bmatrix}
        1 & \\
        0 & 1 \\
        0 & 0 & \ddots \\
        \vdots & \vdots & & 1 \\
        \vdots & \vdots & & m_{k+1,k} & 1 & \\
        \vdots & \vdots & & m_{k+2,k} & 0 & \ddots \\
        \vdots & \vdots & & \vdots & \vdots & & 1 & \\
        0 & 0 & \hdots & m_{nk} & 0 & \hdots & 0 & 1
      \end{bmatrix}
    \end{align*}
    から$L=L_1L_2\cdots L_{n-1}$を定義すると、$L$は下三角行列、また$U = A^{(n)}$ (上三角行列)とすると、$A = LU$
  \end{itemize}
\end{frame}

\begin{frame}[t,fragile]{LU分解}
  \begin{itemize}
    %\setlength{\itemsep}{1em}
  \item LU分解による連立一次方程式の解法
    \begin{itemize}
    \item 方程式は$A{\bf x} = LU{\bf x} = {\bf b}$と書ける
    \item まず、$L{\bf y} = {\bf b}$を解いて、${\bf y}$を求める(前進代入)
    \item 次に、$U{\bf x} = {\bf y}$を解いて、${\bf x}$を求める(後退代入)
    \end{itemize}
  \item 計算量はガウスの消去法と変わらない
  \item 一度LU分解をしておけば、異なる${\bf b}$に対する解も簡単に求められる \\[2em]
  \item 行列式は$U$の対角成分の積で与えられる (ピボット選択する場合は、行の入れ替えにより符号が変わることに注意)
  \end{itemize}
\end{frame}


\begin{frame}[t,fragile]{LAPACKによる連立一次方程式の求解}
  \begin{itemize}
    \setlength{\itemsep}{1em}
  \item LU分解を行うサブルーチン {\tt dgetrf} \\
    \url{http://www.netlib.org/lapack/explore-html/d3/d6a/dgetrf_8f.html}
  \item Fortranによる関数宣言
\begin{lstlisting}
subroutine dgetrf(integer M, integer N,
         double precision, dimension(lda, *) A,
         integer LDA, integer, dimension(*) IPIV,
         integer INFO)
\end{lstlisting}
\item {\tt A}: 左辺の行列、{\tt M,N}: 次元、{\tt IPIV}: 選択されたピボット行のリスト、{\tt lda}: 通常{\tt M} (行数)と同じで良い
  \end{itemize}
\end{frame}

\begin{frame}[t,fragile]{LAPACKによる連立一次方程式の求解}
  \begin{itemize}
    \setlength{\itemsep}{1em}
  \item C言語から呼び出すための関数宣言を作成 (ハンドブック2.7.4節)
\begin{lstlisting}
void dgetrf_(int *M, int *N, double *A,
             int *LDA, int*IPIV, int *INFO);
\end{lstlisting}
関数名は全て小文字。関数名の最後に {\tt \_} (下線)を付ける
\item LU分解の例
\begin{lstlisting}
m = 10;
n = 10;
a = alloc_dmatrix(m, n);
...
dgetrf_(&m, &n, mat_ptr(a), &m, vec_ptr(ipiv), &info);
\end{lstlisting}
完全なソースコード: \href{https://github.com/todo-group/computer-experiments/blob/master/exercise/linear_system/lu_decomp.c}{lu\_decomp.c}
  \end{itemize}
\end{frame}


\begin{frame}[t,fragile]{まとめ}
  \begin{itemize}
    \setlength{\itemsep}{1em}
  \item 一次元(一粒子)シュレディンガー方程式の固有値問題
    \begin{itemize}
    \item シューティングによる方法
    \item 対角化による方法
    \item 特別な場合に有効な方法
    \end{itemize}
  \item 計算精度は何で決まるか?
  \item それぞれのコスト(=計算時間・メモリ)は?
  \end{itemize}
\end{frame}

\end{document}
