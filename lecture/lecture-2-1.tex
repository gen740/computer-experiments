%-*- coding:utf-8 -*-

\documentclass[dvipdfmx]{beamer}
\usepackage{tutorial}

\title{計算機実験II (L1) --- 対角化と量子力学}
\date{2017/09/29}

\begin{document}

\begin{frame}
  \titlepage
  \tableofcontents
\end{frame}

\section{講義・実習の概要}

\begin{frame}[t]{講義・実習の目的}
  \begin{itemize}
    %\setlength{\itemsep}{1em}
  \item 理論・実験を問わず、学部〜大学院〜で必要となる現代的かつ普遍的な計算機の素養を身につける
  \item {\color{gray}UNIX環境に慣れる(シェル、ファイル操作、エディタ)}
  \item {\color{gray}ネットワークの活用 (リモートログイン、共同作業)}
  \item {\color{gray}プログラムの作成(C言語、コンパイラ、プログラム実行)}
  \item 基本的な数値計算アルゴリズム・数値計算の常識を学ぶ
  \item {\color{gray}科学技術文書作成に慣れる(\LaTeX, グラフ作成)}
  \item {\color{red}物理学における具体的な問題を通して実践的な知識と経験を身につける}
  \end{itemize}
\end{frame}

\begin{frame}[t]{身に付けて欲しいこと}
  \begin{itemize}
    %\setlength{\itemsep}{1em}
  \item ツールとしてないものは自分で作る (物理の伝統)
  \item すでにあるものは積極的に再利用する (車輪の再発明をしない)
  \item 数学公式と数値計算アルゴリズムは別物
  \item 刻み幅・近似度合いを変えて何度か計算を行う
  \item グラフ化して目で見てみる
  \item 計算量(コスト)のスケーリング(次数)に気をつける
  \item 記録に残す・再現性を確保する
  \item {\color{red}問題の解き方は一通りではない}
  \end{itemize}
\end{frame}

\begin{frame}[t]{講義・実習内容}
  \begin{itemize}
    \setlength{\itemsep}{1em}
  \item 問題解決型: 計算機実験Iで身に付けた知識をもとに、より高度な数値計算手法・アルゴリズムを学び、物理学における具体的な問題への応用を通して実践的な知識と経験を身につける
    \begin{itemize}
    \item 数値対角化と量子力学
    \item モンテカルロ法・分子動力学と統計物理
    \item 最適化問題
    \end{itemize}
  \item スタッフ \href{mailto:computer@exa.phys.s.u-tokyo.ac.jp}{computer@exa.phys.s.u-tokyo.ac.jp}
    \begin{itemize}
    \item 講義: 藤堂
    \item 実習: 鈴木(元早野研)、斉藤(古澤研)
    \item 実習TA: 山本(藤堂研D1)、近藤(藤堂研M1)
    \end{itemize}
  \end{itemize}    
\end{frame}

\begin{frame}[t]{質問がある場合には、、、}
  \begin{enumerate}
    %\setlength{\itemsep}{1em}
  \item ITC-LMS の掲示板を見る
  \item ハンドブック、講義資料を確認
  \item まわりの人に質問してみる
  \item ネットで検索
  \item 計算機実験担当者(\href{mailto:computer@exa.phys.s.u-tokyo.ac.jp}{computer@exa.phys.s.u-tokyo.ac.jp}) に相談
  \end{enumerate}
  メールで質問するときに注意すべきこと
  \begin{itemize}
  \item (メールの)標題をきちんとつける、きちんと名乗る
  \item 実行環境を明示する
  \item 問題を再現する手順を明記する
  \item 関連するファイル(Cや \LaTeX のソースコード等)を添付する
  \item エラーメッセージを添付する
  \end{itemize}
\end{frame}

\section{二重井戸ポテンシャル}

\begin{frame}[t,fragile]{二重井戸ポテンシャル中の粒子}
  \begin{itemize}
    %\setlength{\itemsep}{1em}
  \item 時間依存しないシュレディンガー方程式
    \begin{align*}
      \big[ -\frac{d^2}{dx^2} + V(x) \big] \psi(x) = E \psi(x)
    \end{align*}
    ($\hbar^2/2m = 1$となるように単位をとった)
  \item 二重井戸ポテンシャル
    \begin{align*}
      V(x) = \begin{cases}
        \infty & \text{$x < 0$, $x > 1$} \\
        0 & \text{$0 < x < a$, $b < x < 1$} \\
        v & \text{$a < x < b$}
      \end{cases}
    \end{align*}
    ただし、$0<a<b<1$とする
  \item 境界条件: $\psi(0) = \psi(1) = 0$、$0 < x < 1$で$\psi(x)$とその導関数が連続
  \end{itemize}
\end{frame}

\begin{frame}[t,fragile]{シュレディンガー方程式の解法}
  \begin{itemize}
    \setlength{\itemsep}{1em}
  \item シューティング
    \begin{itemize}
    \item 計算機実験I (L2)
    \item シューティングに用いる積分法: 2階常微分方程式の2次元1階連立微分方程式への書き換え、オイラー法とその改良、Numerov法
    \end{itemize}
  \item ハミルトニアンの対角化
    \begin{itemize}
    \item 計算機実験I (L4)
    \item 対角化手法: ハウスホルダー法(LAPACK)、べき乗法、Lanczos法
    \end{itemize}
  \item その他の方法: 手で解けるところはあらかじめ解いて次元を減らす
  \item それぞれのコスト(=計算時間・メモリ)は?
  \end{itemize}
\end{frame}


\section{積分による解法}

\begin{frame}[t,fragile]{シューティング}
  \begin{itemize}
    \setlength{\itemsep}{1em}
  \item シューティング[計算機実験I (L3) p.16]
  \item 積分法
    \begin{itemize}
    \item 2階常微分方程式の2次元1階連立微分方程式への書き換え[計算機実験I (L2) p.2]
    \item オイラー法[計算機実験I (L2) p.4-5]
    \item オイラー法の改良[計算機実験I (L2) p.6-9]
    \item Numerov法[計算機実験I (L2) p.12-14]
    \end{itemize}
  \item シューティングに用いる二分法
  \end{itemize}
\end{frame}

\section{二分法}

\begin{frame}[t,fragile]{二分法}
  \begin{itemize}
    % \setlength{\itemsep}{1em}
  \item 反復法により一次元の方程式$f(x)=0$の解を求める
  \item 導関数を使わず関数値のみを利用 (c.f. ニュートン法)
  \item 初期条件として、$f(a) \times f(b) < 0$を満たす2点の組($a<b$)で解をはさみ込み、領域を狭めていく
  \item $a$と$b$の中点$x=(a+b)/2$を考える
    \begin{itemize}
    \item $|f(x)|$が十分小さい場合: $x$が解
    \item $f(a) \times f(x) < 0$の場合: $[a,x]$を新しい領域にとる
    \item $f(x) \times f(b) < 0$の場合: $[x,b]$を新しい領域にとる
    \end{itemize}
  \item 領域$[a,b]$の幅が十分小さくなったら終了
  \item 反復のたびに領域の幅は半分になる
  \item 全ての解を得られる保証はない
  \item 例: \href{https://github.com/todo-group/computer-experiments/blob/master/exercise/basics/bisection.c}{exercise/basics/bisection.c}
  \end{itemize}
\end{frame}


\section{対角化による解法}

\begin{frame}[t,fragile]{対角化}
  \begin{itemize}
    \setlength{\itemsep}{1em}
  \item シュレディンガー方程式の行列表示[計算機実験I (L3) p.17]
  \item ハウスホルダー法(LAPACK) [計算機実験I (L3) p.26-28]
  \item べき乗法[計算機実験I (L3) p.30-32]
  \item Lanczos法[計算機実験I (L3) p.34-40]
  \end{itemize}
\end{frame}

\section{解析計算による次元削減}

\begin{frame}[t,fragile]{シュレディンガー方程式の一般解}
  \begin{itemize}
    %\setlength{\itemsep}{1em}
  \item 二重井戸ポテンシャル
    \begin{align*}
      V(x) = \begin{cases}
        \infty & \text{$x < 0$, $x > 1$} \\
        0 & \text{$0 < x < a$, $b < x < 1$} \\
        v & \text{$a < x < b$}
      \end{cases}
    \end{align*}
  \item それぞれの領域内では手で解ける
  \item 領域1 ($0 < x < a$)、領域3 ($b < x < 1$)では
    \begin{align*}
      -\frac{d^2}{dx^2}\psi(x) = E\psi(x)
    \end{align*}
  \end{itemize}
\end{frame}

\begin{frame}[t,fragile]{シュレディンガー方程式の一般解}
  \begin{itemize}
    %\setlength{\itemsep}{1em}
  \item 領域1 ($0 < x < a$)、領域3 ($b < x < 1$)における一般解
    \begin{align*}
      \psi(x) &= A_1 e^{i\sqrt{E}x} + B_1 e^{-i\sqrt{E}x} \\
      \psi(x) &= A_3 e^{i\sqrt{E}x} + B_3 e^{-i\sqrt{E}x}
    \end{align*}
    あきらかに$E>0$であるので
    \begin{align*}
      \psi(x) &= \alpha_1 \cos(\sqrt{E}x) + \beta_1 \sin(\sqrt{E}x) \\
      \psi(x) &= \alpha_3 \cos(\sqrt{E}x) + \beta_3 \sin(\sqrt{E}x)
    \end{align*}
  \end{itemize}
\end{frame}

\begin{frame}[t,fragile]{シュレディンガー方程式の一般解}
  \begin{itemize}
    %\setlength{\itemsep}{1em}
  \item 領域2 ($a < x < b$)における一般解
    \begin{align*}
      \psi(x) &= A_2 e^{i\sqrt{(E-v)}\,x} + B_1 e^{-i\sqrt{(E-v)}\,x}
    \end{align*}
  \item $E < v$の場合
    \begin{align*}
      \psi(x) &= \alpha_2 \exp(-\sqrt{(v-E)}\,x) + \beta_2 \exp(\sqrt{(v-E)}\,x)
    \end{align*}
  \item $E > v$の場合
    \begin{align*}
      \psi(x) &= \alpha_2 \cos(\sqrt{(E-v)}\,x) + \beta_2 \sin(\sqrt{(E-v)}\,x)
    \end{align*}
  \end{itemize}
\end{frame}

\begin{frame}[t,fragile]{シュレディンガー方程式の一般解}
  \begin{itemize}
    %\setlength{\itemsep}{1em}
  \item 境界条件($E < v$の場合)
    \begin{align*}
      &\alpha_1 = 0 \\
      &\alpha_1 \cos(\sqrt{E}a) + \beta_1 \sin(\sqrt{E}a) \\
      & \qquad =
      \alpha_2 \exp(-\sqrt{(v-E)}\,a) + \beta_2 \exp(\sqrt{(v-E)}\,a) \\
      &-\alpha_1 \sqrt{E} \sin(\sqrt{E}a) + \beta_1 \sqrt{E} \sin(\sqrt{E}a) \\
      & \qquad =
      - \alpha_2 \sqrt{(v-E)} \exp(-\sqrt{(v-E)}\,a) + \beta_2 \sqrt{(v-E)} \exp(\sqrt{(v-E)}\,a) \\
      &\cdots
    \end{align*}
  \item $\beta_1$, $\alpha_2$, $\beta_2$, $\alpha_3$, $\beta_3$ に関する連立方程式
    \begin{align*}
      M x = 0
    \end{align*}
  \item $5 \times 5$行列$M$は$E$の関数
  \item 非自明な解が存在するための条件: $\det M=0$
  \end{itemize}
\end{frame}


\begin{frame}[t,fragile]{行列式の計算}
  \begin{itemize}
    \setlength{\itemsep}{1em}
  \item ガウスの消去法とLU分解[計算機実験I (L2) p.26-34]
  \item LU分解による行列式の計算[計算機実験I (L2) p.35]
  \item LAPACKによるLU分解[計算機実験I (L3) p.9-10]
  \end{itemize}
\end{frame}

\end{document}
