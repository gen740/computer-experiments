\section{実習その6}

\begin{frame}[t,fragile]{EX6-1: 乱数の生成}
  \begin{itemize}
    %\setlength{\itemsep}{1em}
  \item[6-1-1] \href{https://github.com/todo-group/computer-experiments/blob/master/exercise/monte_carlo/random.c}{exercise/monte\_carlo/random.c}は、Mersenne-Twister乱数発生器(\href{https://github.com/todo-group/computer-experiments/blob/master/exercise/monte_carlo/mersenne_twister.c}{exercise/monte\_carlo/mersenne\_twister.h})により、$(0,1)$の範囲で一様分布する実数乱数を生成するプログラムである。コマンドライン引数により乱数の種(seed)を指定できるようにプログラムを修正せよ。種を変えて何度か乱数を生成し、その時系列を比較してみよ
  \item[6-1-2] $X$を$(0,1)$で一様分布する(実数)確率変数とする。このとき$X^2$, $1/(X+1)$, $\log X$のそれぞれの期待値を(解析的に)求めよ。また、実際に乱数を生成させて期待値を計算し、解析的な結果と比較せよ
  \end{itemize}
\end{frame}

\begin{frame}[t,fragile]{EX6-2: イジング模型のシミュレーション}
  \begin{itemize}
    %\setlength{\itemsep}{1em}
  \item[6-2-1] マルコフ連鎖モンテカルロ法により、二次元イジング模型のエネルギーと比熱の期待値を計算せよ。システムサイズを変えると、エネルギーや比熱はどのように振る舞うか?
  \item[6-2-2] 正規分布にしたがう乱数の生成方法について調べよ。また、平均値$\mu_i$と分散共分散行列$\Sigma_{ij}$をもつ多次元正規分布にしたがう乱数の生成方法を考えよ
  \end{itemize}
\end{frame}
