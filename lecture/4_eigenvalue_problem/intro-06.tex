\begin{frame}[t,fragile]{行列のべき乗・指数関数}
  \begin{itemize}
    %\setlength{\itemsep}{1em}
  \item 行列のべき乗
    \begin{align*}
      A^p &= (U \Lambda U^T)(U \Lambda U^T) \cdots (U \Lambda U^T) \\
      &= U \Lambda^p U^T \qquad \Lambda^p = \text{diag}(\lambda_1^p,\cdots,\lambda_n^p)
    \end{align*}
  \item 行列の指数関数
    \begin{align*}
      e^{xA} &= \sum_{k=0}^{\infty} \frac{1}{k!}(xA)^k = U \Big[ \sum_{k=0}^{\infty} \frac{1}{k!}(x\Lambda)^k \Big] U^T \\
      &= U e^{x \Lambda} U^T \qquad e^{x \Lambda} = \text{diag}(e^{x\lambda_1},\cdots,e^{x\lambda_n})
    \end{align*}
  \item \mbox{} [逆行列 $A^{-1} = U \Lambda^{-1} U^T$] → 逆行列をあらわに求めるかわりに連立方程式を解く
  \item \mbox{} [行列式 $|A| = \prod_i \lambda_i$] → 対角化ではなくLU分解を使う
  \item \mbox{} [対角和(トレース) ${\rm tr} A = \sum_i \lambda_i$] → 対角化不要
  \end{itemize}
\end{frame}
