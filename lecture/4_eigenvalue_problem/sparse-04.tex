\begin{frame}[t,fragile]{第2固有値・第3固有値$\cdots$}
  \begin{itemize}
    %\setlength{\itemsep}{1em}
  \item 第1固有ベクトル$\xi_1$の成分を行列から差し引く(減次)
    \[
    A_1 = A - \lambda_1 \xi_1 \xi_1^T
    \]
    この行列は、固有値 $0,\lambda_2,\lambda_3,\cdots,\lambda_n$を持つ
  \item 行列$A_1$に対してべき乗法を使うと、固有値$\lambda_2$と固有ベクトル$\xi_2$が得られる
  \item 第$k$固有値まで求まっている場合
    \[
    A_k = A - \sum_{i=1}^k \lambda_i \xi_i \xi_i^T
    \]
  \item 実際には数値誤差のため、ベクトルの直交性は厳密ではない
  \item 大きい方から数個程度を求めるのが限界
  \end{itemize}
\end{frame}
