\begin{frame}[t,fragile]{QR法}
  \begin{itemize}
    %\setlength{\itemsep}{1em}
  \item QR分解
    \begin{itemize}
    \item 行列$A$を直交(ユニタリー)行列$Q$と上三角行列$R$の積に分解: $A=QR$
    \item Gram-Schmidtの直交化と等価
    \end{itemize}
  \item QR法による固有値と固有ベクトルの計算
    \begin{itemize}
    \item 行列$A_1$をQR分解($A_1=Q_1R_1$) → $A_2 = R_1Q_1$ 
    \item 行列$A_2$をQR分解($A_2=Q_2R_2$) → $A_3 = R_2Q_2$ 
    \item 行列$A_k$をQR分解($A_k=Q_kR_k$) → $A_{k+1} = R_kQ_k$
    \item 繰り返していくと対角より下の全ての成分は零に収束し、対角成分は固有値に収束する(証明略)
    \end{itemize}
  \item 連続した直交変換: $A_{k+1} = R_kQ_k = Q_k^{-1}Q_kR_kQ_k = Q_k^{-1}A_kQ_k$
    \begin{itemize}
    \item $A_1$が対称(エルミート)三重対角行列の場合、$A_k$も対称(エルミート)三重対角
    \item 密行列に対して最初からQR法を適用するより、Householder法で三重対角化した後で使う方が効率がよい
    \end{itemize}
  \end{itemize}
\end{frame}
