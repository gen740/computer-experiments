\begin{frame}[t,fragile]{Rayleigh-Ritzの方法}
  \begin{itemize}
    %\setlength{\itemsep}{1em}
  \item $n \times n$行列$A$について、互いに正規直交するベクトル$v_1,v_2,\cdots,v_m$ ($m < n$)が張る部分空間の中で「最良の」固有ベクトルを求めたい
  \item $n \times m$行列
    \[
    V=\big[v_1 v_2 \cdots v_m\big]
    \]
    を定義すると、$V^TV=E_m$が成り立つ(ただし$VV^T \ne E_n$)
  \item 部分空間内のベクトルを$w = \sum_i a_i v_i$と表すと、$\displaystyle \frac{w^TAw}{w^Tw}$が極大値を取る(本当の固有ベクトルにできるだけ平行になる)条件は、
    \[
    \frac{\partial}{\partial a_i} \frac{w^TAw}{w^Tw} \sim \sum_j H_{ij}a_j - \lambda a_i = 0
    \]
  \end{itemize}
\end{frame}
