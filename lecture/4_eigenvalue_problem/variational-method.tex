\section{変分法}

\begin{frame}[t,fragile]{変分法}
  \begin{itemize}
    %\setlength{\itemsep}{1em}
  \item 波動関数を互いに直交する正規化された波動関数(基底関数)の線形結合で近似する (変分波動関数、試行関数)
    \begin{align*}
      | \psi \rangle = \sum_{p=1}^m C_p | \phi_p \rangle \qquad (\langle \phi_p | \phi_q \rangle = \delta_{pq})
    \end{align*}
  \item エネルギーの期待値
    \begin{align*}
      E &= \frac{\langle \psi | H | \psi \rangle}{\langle \psi | \psi \rangle} = \frac{\sum_{p,q} C_p^* H_{pq} C_q}{\sum_{p,q} C_p^* \delta_{pq} C_q} \\
      H_{pq} &= \langle \phi_p | H | \phi_q \rangle
    \end{align*}
  \item $E$ができるだけ小さくなるよう係数$C_p$を最適化 (変分原理)
  \end{itemize}
\end{frame}

\begin{frame}[t,fragile]{変分法}
  \begin{itemize}
    %\setlength{\itemsep}{1em}
  \item $\delta E = 0$から
    \begin{align*}
      \sum_{q} (H_{pq} - E \delta_{pq} ) C_q = 0 \qquad \text{for $^\forall p$}
    \end{align*}
  \item $H_{pq}$, $\delta_{pq}$を$m \times m$行列と考えると、固有値問題とみなせる
    \begin{align*}
      H C = E C
    \end{align*}
  \item $H$はエルミート行列
  \item $\{ \phi_p \}$の張る部分空間での最適化 (= Rayleigh-Ritzの方法)
  \item 変分波動関数と真の波動関数の差が$\epsilon$程度の時、$E$と真の固有エネルギーの差は$\epsilon^2$程度
  \end{itemize}
\end{frame}

\begin{frame}[t,fragile]{非直交基底関数による変分法}
  \begin{itemize}
    %\setlength{\itemsep}{1em}
  \item 重なり積分
    \begin{align*}
      S_{pq} = \langle \phi_p | \phi_q \rangle \ne \delta_{pq}
    \end{align*}
  \item 変分波動関数の正規化条件
    \begin{align*}
      \langle \psi | \psi \rangle = \sum_{p,q} C_p^* \langle \phi_p | \phi_q \rangle C_q = \sum_{p,q} C_p^* S_{pq} C_q = 1
    \end{align*}
  \item エネルギー期待値
    \begin{align*}
      E = \frac{\sum_{p,q} C_p^* H_{pq} C_q}{\sum_{p,q} C_p^* S_{pq} C_q}
    \end{align*}
  \item $\delta E = 0$から
    \begin{align*}
      \sum_q (H_{pq} - E S_{pq}) C_q = 0 \ \Rightarrow \ HC = ESC \ \text{(一般化固有値問題)}
    \end{align*}
  \end{itemize}
\end{frame}

\begin{frame}[t,fragile]{一般化固有値問題}
  \begin{itemize}
    %\setlength{\itemsep}{1em}
  \item 重なり行列 $S_{pq} = \langle \phi_p | \phi_q \rangle$
    \begin{itemize}
      \item エルミート行列: $S_{pq} = S_{qp}^*$
      \item 正定値 ($\{\phi_p\}$が線形独立の場合):
        \begin{align*}
          x^\dagger S x = \sum_{pq} \langle \phi_p | \phi_q \rangle x_p^* x_q = || \sum_p x_p | \phi_p \rangle ||^2 > 0
        \end{align*}
    \end{itemize}
  \item 一般化固有値問題 $\Rightarrow$ 2回の固有値分解により解くことができる
    \begin{itemize}
      \item $S$を固有値分解: $S = U D U^\dagger$
      \item $S$固有値は全て正 $\Rightarrow$ $D^{-1/2}$を定義可
      \item $HC=ESC$ $\Rightarrow$ $D^{-1/2} U^\dagger H U D^{-1/2} D^{1/2} U^\dagger C = E D^{1/2} U^\dagger C$
      \item $H' = D^{-1/2} U^\dagger H U D^{-1/2}$、$C'D^{1/2} U^\dagger C$とおくと
        \begin{align*}
          H'C' = EC' \qquad \text{(通常の)固有値問題}
        \end{align*}
      \item (1回目の固有値分解はコレスキー分解$A=L L^\dagger$ ($L$は下三角行列)を用いてもよい)
    \end{itemize}
  \end{itemize}
\end{frame}
