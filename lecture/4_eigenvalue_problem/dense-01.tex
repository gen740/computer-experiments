\begin{frame}[t,fragile]{基本方針}
  \begin{itemize}
    %\setlength{\itemsep}{1em}
  \item やってはいけない方法: 特性方程式
    \[
    |\lambda I - A| = 0
    \]
    の係数を求めて、代数方程式として解く
    \begin{itemize}
    \item 数値的に不安定 (代数方程式の解は係数の誤差に対して敏感)
    \item 計算コスト大[$\sim O(N!)$]
    \end{itemize}
  \item スタンダードな方法: 行列を次々に直交変換して、対角行列(あるいは三重対角行列)に近づけていく
    \[
    A \rightarrow U_1^T A U_1 \rightarrow U_2^T (U_1^T A U_1) U_2 \rightarrow U_3^T (U_2^T (U_1^T A U_1) U_2) U_3 \rightarrow \cdots
    \]
  \item 固有値は変換された行列の固有値、固有ベクトルは変換後の行列の固有ベクトルに左から$U_1 U_2 U_3 \cdots$を掛けたもの
  \end{itemize}
\end{frame}
