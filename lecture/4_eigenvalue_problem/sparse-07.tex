\begin{frame}[t,fragile]{Lanczos法}
  \begin{itemize}
    \setlength{\itemsep}{1em}
  \item 初期(ランダム)ベクトル$v_1$に加えて
    \[
    Av_1, Av_2, \cdots A^{m-1}v_1
    \]
    を正規直交化して$v_1,v_2,\cdots,v_m$を作る(Krylov部分空間)
  \item 部分空間でのRitz値を固有値の近似値とする
  \item $A^nv_1$はどんどん最大固有ベクトルに近づいていくので、$m \ll n$でも良い近似固有値が得られると期待される
  \item Lanczos法
  \end{itemize}
\end{frame}
