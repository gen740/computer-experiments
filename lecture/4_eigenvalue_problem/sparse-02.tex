\begin{frame}[t,fragile]{べき乗法(Power Method)}
  \begin{itemize}
    %\setlength{\itemsep}{1em}
  \item 適当なベクトル$v_1$から出発する
  \item $v_1$が最大固有ベクトル$\xi_1$と直交していないとすると
    \[
    v_1 = c_1 \xi_1 + c_2 \xi_2 + c_3 \xi_3 + \cdots + c_n \xi_n
    \]
    と展開できる($c_1 \ne 0$)。この両辺に$A$を次々と掛けていくと
    \begin{align*}
      v_2 = A v_1 &= c_1 \lambda_1 \xi_1 + c_2 \lambda_2 \xi_2 + c_3 \lambda_3 \xi_3 + \cdots + c_n \lambda_n \xi_n \\
      v_3 = A^2 v_1 &= c_1 \lambda_1^2 \xi_1 + c_2 \lambda_2^2 \xi_2 + c_3 \lambda_3^2 \xi_3 + \cdots + c_n \lambda_n^2 \xi_n \\
      \vdots \\
      v_{k+1} = A^k v_1 &= c_1 \lambda_1^k \xi_1 + c_2 \lambda_2^k \xi_2 + c_3 \lambda_3^k \xi_3 + \cdots + c_n \lambda_n^k \xi_n \\
      &= c_1 \lambda_1^k \Big[ \xi_1 + \sum_{i=2}^n \frac{c_i}{c_1} \big( \frac{\lambda_i}{\lambda_1}\big)^k \xi_i \Big] \approx c_1 \lambda_1^k \xi_1 \\
    \end{align*}
  \end{itemize}
\end{frame}
