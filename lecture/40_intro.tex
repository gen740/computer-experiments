\section{行列の対角化}

\begin{frame}[t,fragile]{時間依存しないシュレディンガー方程式}
  \begin{itemize}
    \setlength{\itemsep}{1em}
  \item 井戸型ポテンシャル中の一粒子問題
    \begin{align*}
      \big[ -\frac{\hbar^2}{2m}\frac{d^2}{dx^2} + V(x) \big] \psi(x) = E \psi(x) \\
      V(x) = \begin{cases}
        0 & \text{$a \le x \le b$} \\ \infty & \text{otherwise}
      \end{cases}
    \end{align*}
  \item $\hbar^2/2m = 1$、$a=0$、$b=1$となるように変数変換して
    \begin{align*}
      \big( \frac{d^2}{dx^2} + E \big) \psi(x) = 0 \qquad 0 \le x \le 1
    \end{align*}
    を境界条件$\psi(0) = \psi(1) = 0$のもとで解けば良い
  \end{itemize}
\end{frame}

\begin{frame}[t,fragile]{常微分方程式の解法の利用}
  \begin{itemize}
    \setlength{\itemsep}{1em}
  \item $x_i=h \times i$ ($h=1/n$)、$x_0=0$、$x_n=1$とする
  \item $\psi(x_0)=0$、$\psi(x_1) = 1$を仮定 ($\psi'(x_0)=1/h$と与えたことに相当)
  \item $E = 0$とおく
  \item 初期条件のもとで、$x=x_n$まで積分
  \item $\psi(x_n)$の符号がかわるまで、$E$を少しずつ増やす
  \item 符号が変わったら、$E$の区間を半分ずつに狭めていき、$\psi(x_n)=0$となる$E$ (固有エネルギー)と$\psi(x)$ (波動関数)を得る
  \end{itemize}
\end{frame}

\begin{frame}[t,fragile]{シュレディンガー方程式の行列表示}
  \begin{itemize}
    %\setlength{\itemsep}{1em}
  \item シュレディンガー方程式
    \[
    -\frac{d^2}{dx^2}\psi(x) = E \psi(x)
    \]
  \item 連立差分方程式を行列の形で表す($\psi(x_0)=\psi(x_n)=0$)
    \[
    \begin{pmatrix}
      \frac{2}{h^2} & -\frac{1}{h^2} \\
      -\frac{1}{h^2} & \frac{2}{h^2} & -\frac{1}{h^2} \\
      & -\frac{1}{h^2} & \frac{2}{h^2} & -\frac{1}{h^2} \\
      & & \ddots & \ddots \\
      & & & -\frac{1}{h^2} & \frac{2}{h^2} \\
    \end{pmatrix}
    \begin{pmatrix}
      \psi(x_1) \\
      \psi(x_2) \\
      \psi(x_3) \\
      \vdots \\
      \psi(x_{n-1}) \\
    \end{pmatrix}
    = E
    \begin{pmatrix}
      \psi(x_1) \\
      \psi(x_2) \\
      \psi(x_3) \\
      \vdots \\
      \psi(x_{n-1}) \\
    \end{pmatrix}
    \]
  \item $(n-1) \times (n-1)$の疎行列の固有値問題
    \begin{itemize}
    \item 固有値: 固有エネルギー
    \item 固有ベクトル: 波動関数
    \end{itemize}
  \end{itemize}
\end{frame}

\begin{frame}[t,fragile]{固体物理・量子統計物理に現れる行列}
  \begin{itemize}
    %\setlength{\itemsep}{1em}
  \item 強束縛近似(tight-binding approx.)のもとでの第二量子化表示
    \[
    H = -t \sum_{\langle i,j \rangle \sigma} (c_{i,\sigma}^\dagger c_{j,\sigma} + h.c.) + \text{(相互作用)}
    \]
  \item 局所スピン模型(ハイゼンベルグ模型)
    \[
    H = \sum_{\langle i,j \rangle} S_i \cdot S_j
    \]
  \item 格子点の数を$n$とすると、ハミルトニアンはそれぞれ$4^n \times 4^n$、$2^n \times 2^n$の(疎)行列で表される。
  \item $n$が大きくなると、行列の次元は指数関数的に増加
  \item 量子多体系に共通する困難
  \end{itemize}
\end{frame}

\begin{frame}[t,fragile]{実対称行列(エルミート行列)の性質}
  \begin{itemize}
    %\setlength{\itemsep}{1em}
  \item $N \times N$実対称行列$A$ ($=A^T$)の固有値問題
    \[
    A x = \lambda x
    \]
  \item $N$個の固有値($\lambda_1,\lambda_2,\cdots,\lambda_N$)は全て実。固有ベクトル($\xi_1,\xi_2,\cdots,\xi_N$)は互いに正規直交するようにとることができる。行列$U$を
    \[
    U = \Big( \xi_1 \, \xi_2 \, \cdots \, \xi_N \Big)
    \]
    と定義すると、$U$は直交(ユニタリ)行列($U^T U = U^{-1} U = I$)
  \item $A$の固有分解(固有値分解)
    \[
    A = U \Lambda U^T \qquad \Lambda = \text{diag}(\lambda_1,\cdots,\lambda_N)
    \]
  \end{itemize}
\end{frame}

\begin{frame}[t,fragile]{行列のべき乗・指数関数}
  \begin{itemize}
    %\setlength{\itemsep}{1em}
  \item 行列のべき乗
    \begin{align*}
      A^p &= (U \Lambda U^T)(U \Lambda U^T) \cdots (U \Lambda U^T) \\
      &= U \Lambda^p U^T \qquad \Lambda^p = \text{diag}(\lambda_1^p,\cdots,\lambda_N^p)
    \end{align*}
  \item 行列の指数関数
    \begin{align*}
      e^{xA} &= \sum_{n=0}^{\infty} \frac{1}{n!}(xA)^n = U \Big[ \sum_{n=0}^{\infty} \frac{1}{n!}(x\Lambda)^n \Big] U^T \\
      &= U e^{x \Lambda} U^T \qquad e^{x \Lambda} = \text{diag}(e^{x\lambda_1},\cdots,e^{x\lambda_N})
    \end{align*}
  \item 逆行列 $A^{-1} = U \Lambda^{-1} U^T$
  \item 行列式 $|A| = \prod_i \lambda_i$、対角和(トレース) ${\rm tr} A = \sum_i \lambda_i$
  \end{itemize}
\end{frame}

\begin{frame}[t,fragile]{行列の数値対角化}
  \begin{itemize}
    %\setlength{\itemsep}{1em}
  \item 一般的に次元が5以上の行列の固有値は、あらかじめ定まる有限回の手続きでは求まらない
  \item 必ず何らかの反復法(+収束判定)が必要となる
  \item 密行列向きの方法
    \begin{itemize}
    \item Jacobi法
    \item Givens変換・Householder法(三重対角化) + QR法など
    \end{itemize}
  \item 疎行列向きの方法
    \begin{itemize}
    \item べき乗法
    \item Lanczos法(三重対角化) + QR法など
    \end{itemize}
  \item 固有ベクトル
    \begin{itemize}
    \item 逆反復法
    \end{itemize}
  \end{itemize}
\end{frame}
