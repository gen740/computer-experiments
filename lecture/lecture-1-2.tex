% -*- coding: utf-8 -*-

\documentclass[10pt,dvipdfmx]{beamer}
\usepackage{tutorial}

\title{計算機実験I (第2回)}
\date{2020/05/07}

\begin{document}

\begin{frame}
  \titlepage
  \tableofcontents
\end{frame}

\begin{frame}[t]{本日の課題}
  \begin{itemize}
    %\setlength{\itemsep}{1em}
  \item 「\href{https://utphys-comp.github.io}{計算機実験のための環境整備}」({\small \href{https://utphys-comp.github.io}{https://utphys-comp.github.io}})を参考に、各自必要な環境を引き続き整備する
  \item 5月5日までに「計算機実験SSH公開鍵登録フォーム」から登録を行った人にはアカウント・公開鍵の登録が完了済
    \begin{itemize}
    \item ユーザ名: 「ce」の後に学籍番号8桁の数字(ハイフンなし) \\ 例: ce05201500
    \item aiのホームディレクトリに{\tt random}という名前のファイルが作成されている。その中の6桁の数字を「作業レポート」で報告
    \end{itemize}
  \item 「\href{https://github.com/utphys-comp/handbook/releases/download/handbook-2019/handbook.pdf}{計算機実験ハンドブック}」の第2章 例2.1.4・例2.2.2を作成・コンパイル・実行
  \item {\color{red} 講義終了後(18:00まで)にITC-LMSで「作業レポート(2020/05/07)」を提出}
  \item その他…
  \end{itemize}
\end{frame}

\section{数値誤差}

\begin{frame}[t,fragile]{数値誤差の原因}
  \begin{itemize}
    \setlength{\itemsep}{1em}
  \item 丸め誤差: 無理数や10進数を有限のビットの2進数で表現することによる誤差
    (例: 0.1 が 0.0999999999998 になる)
  \item 打ち切り誤差: テイラー展開による近似を有限項で打ち切ることによる誤差
    (例: 数値微分)
  \item 桁落ち: 非常に近い数の引き算により生じる
  \item 情報落ち: 非常に大きな数に小さな数を足し込む場合に生じる
    (例: 数値積分や常微分方程式の初期値問題で刻み幅を小さくしすぎると生じる)
  \item オーバーフロー(桁あふれ): 表現できる値を超えてしまう
  \end{itemize}
\end{frame}

\begin{frame}[t,fragile]{桁落ち}
  \begin{itemize}
    \setlength{\itemsep}{1em}
  \item 2次方程式 $ax^2+bx+c=0$の解の公式
    \[
    x_{\pm} = \frac{-b \pm \sqrt{b^2-4ac}}{2a}
    \]
    $b^2 \gg |ac|$の時、桁落ちが生じる
  \item 例) $2.718282x^2 - 684.4566x+0.3161592=0$ の解を7桁の精度で計算してみる(伊理・藤野1985)
    \begin{align*}
      \sqrt{D} &= \sqrt{(684.4566)^2 - 4 \times 2.718282 \times 0.3161592} = 684.4541 \\
      x_+ &= \frac{684.4566+684.4541}{2 \times 2.718282} = \frac{1368.911}{5.436564} = 251.7970 \\
      x_- &= \frac{684.4566-684.4541}{2 \times 2.718282} = \frac{0.0025}{5.436564} = 0.00045\underline{98493}
    \end{align*}
  \end{itemize}
\end{frame}

\begin{frame}[t,fragile]{桁落ちを防ぐ方法}
  \begin{itemize}
    \setlength{\itemsep}{1em}
  \item $b$の符号に応じて、一方を求める(この例では$x_+$)
  \item 他方は解と係数の関係を使って求める
    \[
    x_- = \frac{c/a}{x_+} = \frac{0.3161592 / 2.718282}{251.7970} = 0.000461913\underline{8}
    \]
  \item 回避できない例: 重解に近い場合 $2.718282x^2 - 1.854089x + 0.3161592=0$
    \begin{align*}
      \sqrt{D} &= \sqrt{(1.854089)^2 - 4 \times 2.718282 \times 0.3161592} \\ &= 0.002\underline{64575} \\
      x_\pm &= 1.854089 \pm 0.002\underline{64575} = 1.856\underline{737}, 1.851\underline{445}
    \end{align*}
  \end{itemize}
\end{frame}

\begin{frame}[t,fragile]{数値微分}
  \begin{itemize}
    \setlength{\itemsep}{1em}
  \item 関数のテイラー展開
    \[
    f(x+h) = f(x) + h f'(x) + h^2 f''(x)/2 + h^3 f'''(x)/6 + \cdots
    \]
  \item 数値微分の最低次近似
    \[
    f_1(x,h) \equiv \frac{f(x+h)-f(x)}{h} = f'(x) + h f''(x)/2 + O(h^2)
    \]
  \item より高次の近似
    \[
    f_2(x,h) \equiv \frac{f(x+h)-f(x-h)}{2h} = f'(x) + h^2 f'''(x)/6 + O(h^3)
    \]
  \item 刻み$h$を小さくすると打ち切り誤差は減少するが、小さすぎると今度は桁落ちが大きくなる
  \end{itemize}
\end{frame}

\begin{frame}[t,fragile]{刻み幅を変えた計算}
  \begin{itemize}
    \setlength{\itemsep}{1em}
  \item 刻み幅を変えて何度か計算を行い、収束の様子をみる
  \item グラフ化して目で見てみる
  \item 理論式と比較
    \begin{itemize}
    \item 計算式の正しさの確認
    \item 近似の改良 (収束の加速・補外)
    \end{itemize}
  \item 桁落ち・情報落ちの影響の有無
  \end{itemize}
\end{frame}

% -*- coding: utf-8 -*-

\section{ニュートン法}

\begin{frame}[t,fragile]{ニュートン法}
  \begin{itemize}
    \setlength{\itemsep}{1em}
  \item 反復法により方程式$f(x)=0$の解を求める
  \item 真の解を$x_0$、現在の解の候補を$x_n=x_0+\epsilon$とすると
    \[
    0 = f(x_0) = f(x_0+\epsilon-\epsilon) = f(x_n) - f'(x_n) \epsilon + O(\epsilon^2)
    \]
  \item 次の解の候補 (反復法、逐次近似法)
    \[
    \epsilon \approx \frac{f(x_n)}{f'(x_n)} \quad\quad x_{n+1} = x_n - \frac{f(x_n)}{f'(x_n)}
    \]
  \item 複素変数の複素関数や多変数の場合にも自然に拡張可
  \end{itemize}
\end{frame}

\begin{frame}[t,fragile]{ニュートン法の収束}
  \begin{itemize}
    \setlength{\itemsep}{1em}
  \item $x_n$が$x_0$に十分近い時
    \begin{align*}
      f(x_n) &\approx f'(x_0) (x_n-x_0) + f''(x_0) \frac{(x_n - x_0)^2}{2} \\
      f'(x_n) &\approx f'(x_0) + f''(x_0) (x_n - x_0)
    \end{align*}
  \item ニュートン法で一回反復すると
    \begin{align*}
      x_{n+1} =  x_n - \frac{f(x_n)}{f'(x_n)} &\approx x_n - (1-\frac{f''(x_0)}{f'(x_0)}\frac{(x_n-x_0)}{2})(x_n-x_0) \\
      (x_{n+1}-x_0) &\approx \frac{f''(x_0)}{2f'(x_0)} (x_n - x_0)^2
    \end{align*}
    \item 一回の反復で誤差が2乗で減る(正しい桁数が倍に増える) ⇒ 二次収束
  \end{itemize}
\end{frame}

\begin{frame}[t,fragile]{多次元の場合}
  \begin{itemize}
    \setlength{\itemsep}{1em}
  \item $f(x)=0$: $d$次元(非線形)連立方程式
  \item $x$は$d$次元のベクトル: $x = (x_1,x_2,\cdots,x_d)$
  \item $f(x)$も$d$次元のベクトル: $f(x) = (f_1(x), f_2(x),\cdots,f_d(x))$
  \item 真の解のまわりでの展開 ($x_n = x_0 + \epsilon$)
    \[
    0 = f(x_0) = f(x_0+\epsilon-\epsilon) = f(x_n) - \frac{\partial f(x_n)}{\partial x} \cdot \epsilon + O(|\epsilon|^2)
    \]
  \item ヤコビ行列($d\times d$): $\displaystyle \Big(\frac{\partial f(x_n)}{\partial x}\Big)_{ij} = \frac{\partial f_i(x_n)}{\partial x_j}$
  \item 次の解の候補: $\displaystyle x_{n+1} = x_n - \Big(\frac{\partial f(x_n)}{\partial x}\Big)^{-1} f(x_n)$
  \end{itemize}
\end{frame}

\begin{frame}[t,fragile]{ニュートン法による最適化}
  \begin{itemize}
    \setlength{\itemsep}{1em}
  \item $x$は$d$次元のベクトル: $x = {}^t(x_1,x_2,\cdots,x_d)$, $g(x)$はスカラー
  \item 勾配ベクトル: $\displaystyle [\nabla g(x)]_i = \frac{\partial g(x)}{\partial x_i}$
  \item 極小値(最小値)となる条件: $\nabla g(x)=0$
  \item ニュートン法で$f(x)$を$\nabla g(x)$で置き換えればよい
  \item 次の解の候補: $\displaystyle x_{n+1} = x_n - H^{-1}(x_n) \nabla g(x_n)$
  \item ヘッセ行列(Hessian): $\displaystyle H_{ij}(x_n) = \frac{\partial^2 g}{\partial x_i \partial x_j}(x_n)$
  \end{itemize}
\end{frame}

\begin{frame}[t,fragile]{準ニュートン法}
  \begin{itemize}
    %\setlength{\itemsep}{1em}
  \item ニュートン法では、ヘッセ行列の計算・保存が必要
  \item 準ニュートン法: それまでの反復で計算した勾配ベクトルから、ヘッセ行列を近似($B_n$)
  \item BFGS法(Broyden-Fletcher-Goldfarb-Shanno)
    \[
    B_{n+1} = B_{n} + \frac{y_n y_n^T}{y_n^T s_n} - \frac{B_{n} s_n (B_{n} s_n)^T}{s_n^T B_n s_n}
    \]
  \item $s_n = x_{n+1} - x_n$、$y_n = \nabla g(x_{n+1}) - \nabla g(x_n)$
  \item 直接$B_{n}$の逆行列$C_{n}$を更新することも可能
    \[
    C_{n+1} = B_{n+1}^{-1} = C_n + \Big( 1 + \frac{y_n^T C_n y_n}{y_n^T s_n} \Big)
    \frac{s_n s_n^T}{y_n^T s_n} - \frac{C_n y_n s_n^T + s_n y_n^T C_n^T}{y_n^T s_n} \]
  \item 他にも、SR1法、BHHH法、記憶制限BFGS法
  \end{itemize}
\end{frame}

\begin{frame}[t,fragile]{計算をいつやめるか?}
  \begin{itemize}
    %\setlength{\itemsep}{1em}
  \item 残差による判定
    \[
    |f(x)| < \delta
    \]
  \item 誤差による判定
    \[
    | x_{n+1} - x_{n} | < \epsilon
    \]
  \item 解$x=x_0$が$m$重解の場合、$x=x_0$のまわりで展開すると
    \[
    f(x) \simeq \alpha (x-x_0)^m
    \]
    残差が$\delta$程度になったときの誤差は、$\delta^{1/m}$程度

    逆に$|x-x_0|$が$\delta^{1/m}$以下になると、$f(x)$の値がそれ以上変化しない $\Rightarrow$ $m$重解の精度は計算精度の$1/m$桁程度しかない

  \item 残差による判定と誤差による判定を併用するのがよい
  \end{itemize}
\end{frame}

\begin{frame}[t,fragile]{反復計算}
  \begin{itemize}
    %\setlength{\itemsep}{1em}
  \item {\tt while}による反復 (ハンドブック3.2.3節)
\begin{lstlisting}
double residual = 1;    /* 残差 */
double error = 1;       /* 誤差 */
double delta = 1.0e-12; // 欲しい精度
while (residual > delta && error > delta) {
  /* ニュートン法の漸化式 */
  /* residual と error を計算 */
}
\end{lstlisting}
残差と誤差のどちらかが欲しい精度に達したら計算を終了
\item {\tt break}を使う例 (ハンドブック3.2.4節)
\begin{lstlisting}
for (;;) {
  /* ニュートン法の漸化式 */
  /* residual と error を計算 */
  if (residual < delta || error < delta) break;
}
\end{lstlisting}
  \end{itemize}
\end{frame}

\begin{frame}[t,fragile]{初期段階における収束の改善}
  \begin{itemize}
    %\setlength{\itemsep}{1em}
  \item Newton法は初期値によっては収束しない
  \item 発散や振動を抑える方法として「減速」が有効な場合も
  \item 減速
    \begin{itemize}
    \item 反復式を少し修正する
      \[
      x_{n+1} = x_n - \mu_n \frac{f(x_n)}{f'(x_n)}
      \]
    \item まずは$\mu_n=1$として計算
    \item $|f(x_{n+1})| < |f(x_{n})|$が成り立たないようであれば、$\mu_n$を半分にして再計算
    \item $\mu_n$が十分に小さくなれば、$|f(x)|$は必ず減少する
    \end{itemize}
  \end{itemize}
\end{frame}


% -*- coding: utf-8 -*-

\section{二分法}

\begin{frame}[t,fragile]{二分法}
  \begin{itemize}
    % \setlength{\itemsep}{1em}
  \item 反復法により一次元の方程式$f(x)=0$の解を求める
  \item 導関数を使わず関数値のみを利用 (c.f. ニュートン法)
  \item 初期条件として、$f(a) \times f(b) < 0$を満たす2点の組($a<b$)で解をはさみ込み、領域を狭めていく
  \item $a$と$b$の中点$x=(a+b)/2$を考える
    \begin{itemize}
    \item $|f(x)|$が十分小さい場合: $x$が解
    \item $f(a) \times f(x) < 0$の場合: $[a,x]$を新しい領域にとる
    \item $f(x) \times f(b) < 0$の場合: $[x,b]$を新しい領域にとる
    \end{itemize}
  \item 領域$[a,b]$の幅が十分小さくなったら終了
  \item 反復のたびに領域の幅は半分になる
  \item 全ての解を得られる保証はない
  \item 二分法の例: \href{https://github.com/todo-group/computer-experiments/blob/master/exercise/basics/bisection.c}{bisection.c}
  \end{itemize}
\end{frame}


\section{囲い込み法}

\begin{frame}[t,fragile]{囲い込み法(一次元の最適化)}
  \begin{itemize}
    \setlength{\itemsep}{1em}
  \item $f(a) > f(b) < f(c)$を満たす3点の組$a < b < c$の領域を狭めていく
  \item $[a,b]$、$[b,c]$の広い方(例えば後者)を$b$から見て、黄金比
    [$1:(1+\sqrt{5})/2 \approx 0.382:0.618$]に内分する点を$x$とする
    \begin{itemize}
    \item $f(b) > f(x)$の場合: $[b,c]$を新しい領域にとる
    \item $f(b) < f(x)$の場合: $[a,x]$を新しい領域にとる
    \end{itemize}
  \item もともとの$b$が$[a,c]$を$0.382:0.618$に内分する点だった場合、
    新しい領域の幅は、どちらの場合も0.618
  \item 最初の比率が黄金比からずれていたとしても、黄金比に収束
  \item 黄金分割法(golden section)とも呼ばれる
  \end{itemize}
\end{frame}

\begin{frame}[t,fragile]{最初の囲い込み}
  \begin{itemize}
    \setlength{\itemsep}{1em}
  \item 1点を選び、適当な$\Delta x$を取る
  \item 左右に$\Delta x$動かしてみて、関数値が小さくなる方へ動く
  \item どちらに進んでも関数値が大きくなる場合には、囲い込み完了
  \item 小さくなった場合、その方向へ再び増えるまで$\Delta x$を倍々に増やしながら進む
  \item 最後の3点で極小値を囲い込むことができる
  \item 囲い込み法のプログラムの例: \href{https://github.com/todo-group/computer-experiments/blob/master/exercise/optimization/golden_section.c}{golden\_section.c}
  \end{itemize}
\end{frame}

\begin{frame}[t,fragile]{極小値をとる$x$の精度}
  \begin{itemize}
    \setlength{\itemsep}{1em}
  \item 実数の有効桁数を16桁($\epsilon \approx 10^{-16}$)とする(倍精度)
  \item 真の極小($x_0$)のまわりでテイラー展開
    \[
    f(x) \approx f(x_0) + \frac{1}{2} f''(x_0) (x-x_0)^2
    \]
  \item $f''(x_0) / f(x_0)$が$O(1)$だとすると
    \[
    |x-x_0| \sim \sqrt{\epsilon} \sim 10^{-8}
    \]
    以下になると、第二項の第一項に対する比が$\epsilon$よりも小さくなる
  \item それ以上領域を狭めても、関数値は変化しない
  \end{itemize}
\end{frame}


% -*- coding: utf-8 -*-

\section{常微分方程式の初期値問題}

\begin{frame}[t,fragile]{準備: 微分方程式の書き換え}
  \begin{itemize}
    %\setlength{\itemsep}{1em}
  \item 2階の常微分方程式の一般形
    \[
    \frac{d^2y}{dx^2} + p(x)\frac{dy}{dx} + q(x)y = r(x)
    \]
  \item $y_1 \equiv y$, $y_2 \equiv \frac{dy}{dx}$とおくと
    \[
    \left\{
    \begin{array}{ccl}
      \frac{dy_1}{dx} & = & y_2 \\
      \frac{dy_2}{dx} & = & r(x) - p(x) y_2 - q(x) y_1
    \end{array}
    \right.
    \]
  \item さらに$\bm{y}\equiv(y_1, y_2)$, $\bm{f}(x, \bm{y})\equiv \left(y_2, r(x)-p(x)y_2 - q(x)y_1\right)$
    \[
    \frac{d\bm{y}}{dx} = \bm{f}(x, \bm{y})
    \]
  \item $n$階常微分方程式 $\Rightarrow$ $n$次元の1階常微分方程式
  \end{itemize}
\end{frame}

\begin{frame}[t,fragile]{初期値問題と境界値問題}
  \begin{itemize}
    \setlength{\itemsep}{1em}
  \item 初期値問題
    \begin{itemize}
    \item 微分方程式において、ある1点に関する全ての境界条件(初期値)が与えられているもの
    \item 質点の運動など(時系列の問題)
  \end{itemize}
  \item 境界値問題
    \begin{itemize}
    \item 複数の点に関する境界条件が与えられているもの
    \item 物体のゆがみの計算や静電場の計算など(空間的に解く問題)
  \end{itemize}
  \item 初期値問題は初期値から逐次的に解くことが可能
  \item 境界値問題は初期値問題に比べて計算法が複雑
  \end{itemize}
\end{frame}

\begin{frame}[t,fragile]{初期値問題の解法 (Euler法)}
  \begin{itemize}
    %\setlength{\itemsep}{1em}
  \item $h$を微小量として微分を差分で近似する(前進差分)
    \[
    \frac{dy}{dt} \approx \frac{y(t+h) - y(t)}{h} = f(t, y)
    \]
  \item $t=0$における$y(t)$の初期値を$y_0$、$t_n \equiv nh$、$y_n$を$y(t_n)$の近似値とおくと、
    \[
    y_{n+1}-y_n = h f( t_n, y_n)
    \]
  \item Euler法
    \begin{itemize}
    \item $y_0$からはじめて、$y_1,y_2,\cdots$を順次求めていく
    \end{itemize}
  \end{itemize}
\end{frame}

\begin{frame}[t,fragile]{Euler法の精度}
  \begin{itemize}
    \setlength{\itemsep}{1em}
  \item 微分方程式の両辺を$t_n$から$t_{n+1}$まで積分(積分方程式)
    \[
    y(t_{n+1}) - y(t_n) = \int^{t_{n+1}}_{t_n} \!\! f(t, y(t)) dt = h \int^1_0 \! f(t_n+h\tau, y(t_n+h\tau)) d\tau
    \]
  \item Euler法は、被積分関数を定数で近似することに対応
    \[
    f(t_n+h\tau, y(t_n+h\tau)) = f(t_n, y(t_n)) + O(h)
    \]
  \item $t=0$からある$t_f$まで積分すると、反復回数$N = t_f / h$
  \item $t=t_f$における誤差 $\sim N \times h \times O(h) = O(h)$
  \end{itemize}
\end{frame}

\begin{frame}[t,fragile]{Euler法の改良}
  \begin{itemize}
    \setlength{\itemsep}{1em}
  \item 積分方程式の被積分関数をもう1次高次まで展開
    \[
    f(t_n+h\tau, y(t_n+h\tau)) = f(t_n, y(t_n)) +
    \tau h
    \left\{
    \frac{\partial f}{\partial t}
    + f \frac{\partial f}{\partial y}
    \right\}_{t=t_n, y=y_n}
    \!\!\!\!\!\!\!\!\!\!\!\! + O(h^2)
    \]
  \item 積分を実行すると
    \[
    y(t_{n+1}) = y(t_n) + h f(t_n, y_n) + \frac{1}{2}h^2
    \left\{
    \frac{\partial f}{\partial t}
    + f \frac{\partial f}{\partial y}
    \right\}_{t=t_n, y=y_n}
    \!\!\!\!\!\!\!\!\!\!\!\! + O(h^3)
    \]
  \end{itemize}
\end{frame}

\begin{frame}[t,fragile]{中点法(2次Runge-Kutta法)}
  \begin{itemize}
    %\setlength{\itemsep}{1em}
  \item 2次公式
    \[
    \begin{array}{rcl}
      k_1 & = & h f(t_n, y_n) \\
      k_2 & = & h f(t_n + \frac{1}{2}h, y_n + \frac{1}{2}k_1) \\
      y_{n+1} & = & y_n + k_2
    \end{array}
    \]
  \item このとき
    \[
    k_2 = h 
    \left\{
    f(t_n, y_n)
    + \frac{1}{2}h \frac{\partial f}{\partial t}
    + \frac{1}{2}k_1 \frac{\partial f}{\partial y}
    + O(h^2)
    \right\}
    \]
  \item したがって
    \[
    y_{n+1} = y_n + h f(t_n, y_n) + \frac{1}{2}h^2
    \left\{
    \frac{\partial f}{\partial t}
    + f \frac{\partial f}{\partial y}
    \right\}_{t=t_n, y=y_n}
    \!\!\!\!\!\!\!\!\!\!\!\!+ O(h^3)
    \]
  \end{itemize}
\end{frame}

\begin{frame}[t,fragile]{高次のRunge-Kutta法}
  \begin{itemize}
    %\setlength{\itemsep}{1em}
  \item 3次Runge-Kutta法
    \[
    \begin{array}{rcl}
      k_1 & = & h f(t_n, y_n) \\
      k_2 & = & h f(t_n + \frac{2}{3}h, y_n + \frac{2}{3}k_1) \\
      k_3 & = & h f(t_n + \frac{2}{3}h, y_n + \frac{2}{3}k_2) \\
      y_{n+1} & = & y_n + \frac{1}{4}k_1 + \frac{3}{8}k_2
      + \frac{3}{8}k_3
    \end{array}
    \]
  \item 4次Runge-Kutta法
    \[
    \begin{array}{rcl}
      k_1 & = & h f(t_n, y_n) \\
      k_2 & = & h f(t_n + \frac{1}{2}h, y_n + \frac{1}{2}k_1) \\
      k_3 & = & h f(t_n + \frac{1}{2}h, y_n + \frac{1}{2}k_2) \\
      k_4 & = & h f(t_n + h, y_n + k_3) \\
      y_{n+1} & = & y_n + \frac{1}{6}k_1 + \frac{1}{3}k_2
      + \frac{1}{3}k_3 + \frac{1}{6}k_4
    \end{array}
    \]
  \item 4次までは次数と$f$の計算回数が等しい
  \end{itemize}
\end{frame}

\begin{frame}[t,fragile]{計算コストと精度}
  \begin{itemize}
    %\setlength{\itemsep}{1em}
  \item 実際の計算では$f(t,y)$の計算にほとんどのコストがかかる
  \item 計算回数と計算精度の関係
    \begin{center}
      \begin{tabular}[h]{c|cccc}
        & 1次(Euler法) & 2次(中点法) & 3次 & 4次 \\
        \hline
        計算精度 & $O(h)$ & $O(h^2)$ & $O(h^3)$ & $O(h^4)$ \\
        計算回数 & $N$ & $2N$ & $3N$ & $4N$
      \end{tabular}
    \end{center}
  \item 高次のRunge-Kuttaを使う方が効率的
  \item どれくらい小さな$h$が必要となるか、前もっては分からない
  \item 刻み幅を変えて($h,h/2,h/4,\dots$)計算してみることが大事
    \begin{itemize}
    \item 誤差の評価
    \item 公式の間違いの発見
    \end{itemize}
  \end{itemize}
\end{frame}

\begin{frame}[t,fragile]{陽解法と陰解法}
  \begin{itemize}
    %\setlength{\itemsep}{1em}
  \item 陽解法: 右辺が既知の変数のみで書かれる(例: Euler法)
    \begin{itemize}
    \item プログラムがシンプル
    \end{itemize}
  \item 陰解法: 右辺にも未知変数が含まれる
    \begin{itemize}
    \item 例: 逆Euler法
      \begin{align*}
        y(t) &= y(t+h-h) = y(t+h) - h f(t+h,y(t+h)) + O(h^2) \\
        y_{n+1} &= y_n + h f(t+h,{\color{red}y_{n+1}})
      \end{align*}
    \item 数値的により安定な場合が多い
    \item 一般的には、Newton法などを使って非線形方程式を解く必要がある
    \end{itemize}
  \end{itemize}
\end{frame}

\begin{frame}[t,fragile]{Euler法の安定性}
  \begin{itemize}
    %\setlength{\itemsep}{1em}
  \item 方程式$\displaystyle \frac{dy}{dt} = k y(t)$を初期条件$y(0)=1$のもとで解くと$y(t)=e^{kt}$
    \begin{itemize}
      \item ${\rm Re}\, k < 0$であれば、$\displaystyle \lim_{t\rightarrow \infty} y(t) = 0$となる
    \end{itemize}
  \item (陽的) Euler法
    \[
    y_{n+1} = y_n + h f(t_n,y_n) = y_n + h k y_n = (1+hk)y_n
    \]
    \begin{itemize}
    \item $\displaystyle \lim_{t\rightarrow \infty} y(t) = 0$となるための条件
      \[
      |  1 + hk | < 1
      \]
    \item $k$が負の実数であっても、$h > 2 / |k|$では発散 $\Rightarrow$ 不安定
    \end{itemize}
  \end{itemize}
\end{frame}

\begin{frame}[t,fragile]{陰解法の安定性}
  \begin{itemize}
    %\setlength{\itemsep}{1em}
  \item (陰的) 逆Euler法
    \begin{align*}
    y_{n+1} &= y_n + h f(t_n,y_{n+1}) = y_n + h k y_{n+1} \\
    y_{n+1} &= \frac{1}{1-hk} y_n
    \end{align*}
    \begin{itemize}
    \item $\displaystyle \lim_{t\rightarrow \infty} y(t) = 0$となるための条件
      \[
      |  1 - hk | > 1
      \]
    \item $k$の実部が負であれば、常に$\displaystyle \lim_{t\rightarrow \infty} y(t) = 0$
    \item 真の解がゼロに収束する$k$の全領域において数値解も収束

      $\Rightarrow$ 「A安定」という
    \end{itemize}
  \end{itemize}
\end{frame}



\section{Numerov法}

\begin{frame}[t,fragile]{Numerov法}
  \begin{itemize}
    %\setlength{\itemsep}{1em}
  \item Numerov法
    \begin{itemize}
    \item 二階の常微分方程式で一階の項がない場合に使える
    \item 連立微分方程式に直さずに直接二階微分方程式を解く
    \item 4次の陰解法
    \item 方程式が線形の場合は陽解法に書き直せる
    \end{itemize}
  \item 微分方程式
    \[
    \frac{d^2y}{dx^2} = f(x,y)
    \]
  $y=y(x)$を$x=x_i$のまわりでテイラー展開する。$x_{i \pm 1} = x_i \pm h$での表式は
      \[
      y(x_{i \pm 1}) = y(x_i) \pm h y'(x_i) + \frac{h^2}{2} y''(x_i) \pm \frac{h^3}{6} y'''(x_i) + \frac{h^4}{24} y''''(x_i)  + O(h^5)
      \]
  \end{itemize}
\end{frame}

\begin{frame}[t,fragile]{Numerov法}
  \begin{itemize}
    \setlength{\itemsep}{1em}
  \item 二階微分の差分近似 ($y_i \equiv y(x_i)$等と書く)
    \[
    \frac{y_{i+1} - 2 y_i + y_{i-1}}{h^2} = y''_{i} + \frac{h^2}{12} y''''_{i} + O(h^4)
    \]
  一方で、微分方程式より
    \[
    y''''_i = \frac{d^2f}{dx^2}\Big|_{x=x_i} = \frac{f_{i+1}-2f_i+f_{i-1}}{h^2} + O(h^2)
    \]
    組み合わせると
    \[
    y_{i+1} = 2y_i - y_{i-1} + \frac{h^2}{12} (f_{i+1} + 10f_{i} + f_{i-1}) + O(h^6)
    \]
  \end{itemize}
\end{frame}

\begin{frame}[t,fragile]{Numerov法}
  \begin{itemize}
    %\setlength{\itemsep}{1em}
  \item 方程式が線形の場合、$f(x,y) = -a(x) y(x)$を代入すると
    \[
    y_{i+1} = 2y_i - y_{i-1} - \frac{h^2}{12} (a_{i+1}y_{i+1} + 10a_{i}y_{i} + a_{i-1}y_{i-1}) + O(h^6)
    \]
  $y_{i+1}$を左辺に集めると、陽解法となる
    \[
    y_{i+1} = \frac{2 (1-\frac{5h^2}{12} a_i)y_i - (1 + \frac{h^2}{12} a_{i-1}) y_{i-1}}{1 + \frac{h^2}{12} a_{i+1}} + O(h^6)
    \]
  \end{itemize}
\end{frame}


% -*- coding: utf-8 -*-

\section{シンプレクティック積分法}

\begin{frame}[t,fragile]{ハミルトン力学系}
  \begin{itemize}
    % \setlength{\itemsep}{1em}
  \item 時間をあらわに含まない場合のハミルトン方程式
    \[
    \frac{dq}{dt} = \frac{\partial H}{\partial p}, \ \frac{dp}{dt} = -\frac{\partial H}{\partial q}
    \]
    \begin{itemize}
    \item エネルギー保存則
      \[
      \frac{dH}{dt} = \frac{\partial H}{\partial q} \frac{dq}{dt} + \frac{\partial H}{\partial p} \frac{dp}{dt} = 0
      \]
    \item 位相空間の体積が保存(Liouvilleの定理)

      位相空間上の流れの場$\bm{v} = (\frac{dq}{dt},\frac{dp}{dt})$について
      \[
      \text{div} \bm{v} = \frac{\partial}{\partial q} \frac{dq}{dt} + \frac{\partial}{\partial p} \frac{dp}{dt} = 0
      \]
    \end{itemize}
  \item Euler法、Runge-Kutta法などはいずれの性質も満たさない
  \end{itemize}
\end{frame}

\begin{frame}[t,fragile]{シンプレクティック数値積分法(Symplectic Integrator)}
  \begin{itemize}
    %\setlength{\itemsep}{1em}
  \item 体積保存を満たす解法
  \item 例: 調和振動子$H=\frac{1}{2}(p^2+q^2)$の運動方程式
    \[
    \frac{dq}{dt} = p, \ \frac{dp}{dt} = -q
    \]
    の一方をEuler法で、他方を逆オイラー法で解く
    \begin{align*}
      q_{n+1} &= q_n + h p_n \\
      p_{n+1} &= p_n - h q_{n+1} = (1-h^2) p_n - h q_n \\
      \begin{pmatrix} q_{n+1} \\ p_{n+1} \end{pmatrix} &= \begin{pmatrix} 1 & h \\ -h & 1-h^2 \end{pmatrix} \begin{pmatrix} q_{n} \\ p_{n} \end{pmatrix}
    \end{align*}
  \end{itemize}
\end{frame}

\begin{frame}[t,fragile]{体積・エネルギーの保存}
  \begin{itemize}
    %\setlength{\itemsep}{1em}
  \item 体積保存
    \begin{align*}
      \det \begin{pmatrix} 1 & h \\ -h & 1-h^2 \end{pmatrix} = 1
    \end{align*}
  \item エネルギーの保存
    \begin{align*}
      \frac{1}{2}(p_{n+1}^2+q_{n+1}^2) + {\color{red}\frac{h}{2} p_{n+1} q_{n+1}} = \frac{1}{2}(p_{n}^2+q_{n}^2) + {\color{red}\frac{h}{2} p_{n} q_{n}}
    \end{align*}
  \item 位相空間の体積は厳密に保存
  \item エネルギーは$O(h)$の範囲で保存し続ける
  \end{itemize}
\end{frame}

\begin{frame}[t,fragile]{2次のシンプレクティック積分法}
  \begin{itemize}
    %\setlength{\itemsep}{1em}
  \item ハミルトニアンが$H(p,q) = T(p) + V(q)$の形で書けるとする
  \item リープ・フロッグ法
    \begin{align*}
      {\color{red} p(t+h/2)} &= p(t) - \frac{h}{2} \frac{\partial V(q)}{\partial q}|_{q=q(t)} \\
      {\color{blue} q(t+h)} &= q(t) + h {\color{red}p(t+h/2)} \\
      p(t+h) &= {\color{red}p(t+h/2}) - \frac{h}{2} \frac{\partial V(q)}{\partial q}|_{q=q(t+h)}
    \end{align*}
  \end{itemize}
\end{frame}

\begin{frame}[t,fragile]{シンプレクティック積分法}
  \begin{itemize}
    \setlength{\itemsep}{1em}
  \item ハミルトン力学系の満たすべき特性(位相空間の体積保存)を満たす
  \item 一般的には陰解法
  \item ハミルトニアンが$H(p,q) = T(p) + V(q)$の形で書ける場合は陽的なシンプレクティック積分法が存在する
  \item エネルギーは近似的に保存する
  \item $n$次のシンプレクティック積分法では、エネルギーは$O(h^n)$の範囲で振動(発散しない)
  \end{itemize}
\end{frame}



%% \section{固有値問題}
%% \begin{frame}[t,fragile]{時間依存しないシュレディンガー方程式}
  \begin{itemize}
    \setlength{\itemsep}{1em}
  \item 井戸型ポテンシャル中の一粒子問題
    \begin{align*}
      \big[ -\frac{\hbar^2}{2m}\frac{d^2}{dx^2} + V(x) \big] \psi(x) = E \psi(x) \\
      V(x) = \begin{cases}
        0 & \text{$a \le x \le b$} \\ \infty & \text{otherwise}
      \end{cases}
    \end{align*}
  \item $\hbar^2/2m = 1$、$a=0$、$b=1$となるように変数変換して
    \begin{align*}
      \big( \frac{d^2}{dx^2} + E \big) \psi(x) = 0 \qquad 0 \le x \le 1
    \end{align*}
    を境界条件$\psi(0) = \psi(1) = 0$のもとで解けば良い
  \end{itemize}
\end{frame}

%% \begin{frame}[t,fragile]{固有値問題の解法}
  \begin{itemize}
    \setlength{\itemsep}{1em}
  \item $x_i=h \times i$ ($h=1/n$)、$x_0=0$、$x_n=1$とする
  \item $\psi(x_0)=0$、$\psi(x_1) = 1$を仮定 ($\psi'(x_0)=1/h$と与えたことに相当)
  \item $E = 0$とおく
  \item Runge-Kutta法、Numerov法などを用いて$x=x_n$まで積分
  \item $\psi(x_n)$の符号がかわるまで、$E$を少しずつ増やす
  \item 符号が変わったら、$E$の区間を半分ずつに狭めていき、$\psi(x_n)=0$となる$E$ (固有エネルギー)と$\psi(x)$ (波動関数)を得る
  \end{itemize}
\end{frame}


%% \section{ポインタと配列}

%% \begin{frame}[t,fragile]{C言語におけるポインタ}
  \begin{itemize}
    \setlength{\itemsep}{1em}
  \item 変数はメモリ上のどこかに格納されている
    \begin{itemize}
    \item 変数の値: メモリに格納されている数値
    \item アドレス: 変数の値が格納されているメモリ上の番地
    \end{itemize}
  \item ポインタ変数
    \begin{itemize}
    \item 値としてアドレスを格納する変数のこと
    \item ポインタ変数の値(アドレス)とポインタ変数のアドレスは異なるものであることに注意
    \end{itemize}
  \item ポインタ変数の宣言、代入、実体へのアクセス
    \begin{itemize}
    \item 整数型ポインタ変数の宣言: {\color{red} \verb+int *p;+}
    \item 整数型変数の宣言: \verb+int q;+
    \item 変数\verb+q+のアドレスをポインタ変数\verb+p+に代入: {\color{red} \verb+p = &q;+}
    \item ポインタ変数\verb+p+に格納されているアドレスに格納されている値の参照(間接参照): {\color{red} \verb+*p+}
    \end{itemize}
  \end{itemize}
\end{frame}

%% \begin{frame}[t,fragile]{ポインタの例(1)}
  \begin{itemize}
    %\setlength{\itemsep}{1em}
  \item 例2.5.1 (ハンドブック2.5節)
\begin{lstlisting}
#include <stdio.h>
int main() {
  int *p;
  int q;
  q = 200;
  p = &q;
  printf("q is %d and *p is %d.\n", q, *p);
  return 0;
}
\end{lstlisting}
\begin{itemize}
\item \verb+q+のアドレスを\verb+p+に代入
\item \verb+q+と\verb+*p+の値を出力 → 両者とも200
\end{itemize}
  \end{itemize}
\end{frame}

%% \begin{frame}[t,fragile]{ポインタの例(2)}
  \begin{itemize}
    %\setlength{\itemsep}{1em}
  \item 例2.5.2 (ハンドブック2.5節)
\begin{lstlisting}
#include <stdio.h>
int main() {
  int *p;
  int q;
  p = &q;
  *p = 300;
  printf("q is %d and *p is %d.\n", q, *p);
  return 0;
}
\end{lstlisting}
\begin{itemize}
\item \verb+q+のアドレスを\verb+p+に代入
\item \verb+*p+に300を代入 (ここで\verb+q=300;+と書いても等価)
\item \verb+q+と\verb+*p+の値を出力 → 両者とも300
\end{itemize}
  \end{itemize}
\end{frame}

%% \begin{frame}[t,fragile]{関数呼び出し(ポインタ渡し)}
  \begin{itemize}
    %\setlength{\itemsep}{1em}
  \item 例2.7.4 (ハンドブック2.7節)
\begin{lstlisting}
#include <stdio.h>
void division(int divident, int divisor, int *quotient,
              int *residual) {
  *quotient = divident / divisor;
  *residual = divident % divisor;
}
int main() {
  int josuu = 3;
  int hi_josuu = 13;
  int shou, amari;
  division(hi_josuu, josuu, &shou, &amari);
  printf("%d / %d = %d ... %d\n", hi_josuu, josuu,
         shou, amari);
}
\end{lstlisting}
  \end{itemize}
\end{frame}

%% \begin{frame}[t,fragile]{間違った例(値渡し)}
  \begin{itemize}
    %\setlength{\itemsep}{1em}
  \item 例2.7.5 (ハンドブック2.7節)
\begin{lstlisting}
#include <stdio.h>
void division(int divident, int divisor, int quotient,
  int residual) {
  quotient = divident / divisor;
  residual = divident % divisor;
}
int main() {
  int josuu = 3;
  int hi_josuu = 13;
  int shou, amari;
  division(hi_josuu, josuu, shou, amari);
  printf("%d / %d = %d ... %d\n", hi_josuu, josuu,
         shou, amari);
}
\end{lstlisting}
\begin{itemize}
\item 誤った答えが出力される。なぜ?
\end{itemize}
  \end{itemize}
\end{frame}

%% \begin{frame}[t,fragile]{ポインタと一次元配列}
  \begin{itemize}
    %\setlength{\itemsep}{1em}
  \item 一次元配列を表す変数は、(実は)最初の要素を指すポインタ  (ハンドブック2.5.3節)
    \begin{itemize}
    \item \verb+v+ と \verb+&v[0]+ は等価
    \item \verb^(v+2)^ と \verb^&v[2]^ は等価
    \item \verb+*v+ と \verb+v[0]+ は等価
    \item \verb^*(v+2)^ と \verb^v[2]^ は等価
    \item \verb^(v+2)[3]^ は?
    \end{itemize}
  \item C言語では配列の添字は0から始まることに注意
  \item \verb^double v[10];^ と宣言した場合、\verb^v[0]^ 〜 \verb^v[9]^ の10個の要素を持つ配列が作られる。\verb^v[10]^ は存在しない。値を代入したり参照しようとするとエラーとなる
  \item ポインタのテストプログラム: \href{https://github.com/todo-group/computer-experiments/blob/master/exercise/matrix/pointer.c}{pointer.c}
  \end{itemize}
\end{frame}

%% \begin{frame}[t,fragile]{一次元配列}
  \begin{itemize}
    \setlength{\itemsep}{1em}
  \item (静的)一次元配列 (ハンドブック3.3.1節)
\begin{lstlisting}
double v[10];
v[0] = 1.0;
v[1] = 2.0;
...
\end{lstlisting}
    要素数はコンパイル時にすでに決まっている定数でなければならない
  \item (動的)一次元配列 (ハンドブック3.11節)
\begin{lstlisting}
double *v; /* ポインタ */
v = (double*)malloc((size_t)(10 * sizeof(double));
...
free(v); /* 確保した領域を開放 */
\end{lstlisting}
実行時に要素数を指定可能
  \end{itemize}
\end{frame}


\end{document}
