\section{実習その3}

\begin{frame}[t,fragile]{EX3-1: サンプルプログラムの実行}
  \begin{itemize}
    %\setlength{\itemsep}{1em}
  \item[3-1-1] ガウスの消去法のサンプルプログラム(\href{https://github.com/todo-group/computer-experiments/blob/master/exercise/linear_system/gauss.c}{exercise/linear\_system/gauss.c})をコンパイル・実行せよ。実行時にコマンドライン引数に行列の内容が書かれたファイル名({\tt input1.dat})を指定する必要があることに注意
\begin{lstlisting}
$ cc gauss.c -o gauss
$ ./gauss input1.dat
\end{lstlisting}
  \item[3-1-2] LU分解のサンプルプログラム(\href{https://github.com/todo-group/computer-experiments/blob/master/exercise/linear_system/lu_decomp.c}{exercise/linear\_system/lu\_decomp.c})をコンパイル・実行せよ。コンパイル時にLAPACKをリンク({\tt -llapack})する必要があることに注意(ハンドブック3.1.6節)
\begin{lstlisting}
$ cc lu_decomp.c -o lu_decomp -llapack
$ ./lu_decomp input1.dat
\end{lstlisting}
  \end{itemize}
\end{frame}

\begin{frame}[t,fragile]{EX3-2: ピボット選択、境界条件}
  \begin{itemize}
    %\setlength{\itemsep}{1em}
  \item[3-2-1] {\tt gauss.c}では、ピボット選択を行っていないため、入力が{\tt input2.dat}の場合には正しい解が得られない。ピボット選択を行うよう{\tt gauss.c}を修正せよ
  \item[3-2-2] \href{https://github.com/todo-group/computer-experiments/blob/master/exercise/linear_system/laplace_lu.c}{exercise/linear\_system/laplace\_lu.c}では、ディリクレ型の境界条件[$u(0,y) = \sin(\pi y)$, $u(x,0)=u(x,1)=u(1,y)=0$]のもとでのラプラス方程式の解をLU分解により求めている。境界条件を変えてみて解がどのように変化するか、Gnuplotを用いてプロットして確認せよ(Gnuplotの{\tt splot}コマンドを使う)
  \end{itemize}
\end{frame}

\begin{frame}[t,fragile]{EX3-3: ヤコビ法、ガウス・ザイデル法、SOR法}
  \begin{itemize}
    %\setlength{\itemsep}{1em}
  \item[3-3-1] \href{https://github.com/todo-group/computer-experiments/exercise/blob/master/linear_system/laplace_jacobi.c}{exercise/linear\_system/laplace\_jacobi.c}は、作りかけのヤコビ法のプログラムである。収束判定のコードを追加し、プログラムを完成せよ。計算結果や計算速度を{\tt laplace\_lu.c}と比較せよ
  \item[3-3-2] ヤコビ法のプログラム({\tt lapalace\_jacobi.c})を元に、ガウス・ザイデル法、SOR法のプログラムを作成せよ。収束までの回数を比較せよ。
特にSOR法の場合、パラメータ$\omega$の選び方により、どのように収束回数が変化するか観察し、最適な$\omega$の値について考察せよ
  \end{itemize}
\end{frame}
