\section{C言語における行列・LAPACKの利用}

\begin{frame}[t,fragile]{C言語におけるポインタ}
  \begin{itemize}
    \setlength{\itemsep}{1em}
  \item 変数はメモリ上のどこかに格納されている
    \begin{itemize}
    \item 変数の値: メモリに格納されている数値
    \item アドレス: 変数の値が格納されているメモリ上の番地
    \end{itemize}
  \item ポインタ変数
    \begin{itemize}
    \item 値としてアドレスを格納する変数のこと
    \item ポインタ変数の値(アドレス)とポインタ変数のアドレスは異なるものであることに注意
    \end{itemize}
  \item ポインタ変数の宣言、代入、実体へのアクセス
    \begin{itemize}
    \item 整数型ポインタ変数の宣言: {\color{red} \verb+int *p;+}
    \item 整数型変数の宣言: \verb+int q;+
    \item 変数\verb+q+のアドレスをポインタ変数\verb+p+に代入: {\color{red} \verb+p = &q;+}
    \item ポインタ変数\verb+p+に格納されているアドレスに格納されている値の参照(間接参照): {\color{red} \verb+*p+}
    \end{itemize}
  \end{itemize}
\end{frame}

\begin{frame}[t,fragile]{ポインタの例(1)}
  \begin{itemize}
    \setlength{\itemsep}{1em}
  \item 例3.5.1 (ハンドブック3.5節)
\begin{lstlisting}
#include <stdio.h>
int main() {
  int *p;
  int q;
  q = 200;
  p = &q;
  printf("q is %d and *p is %d.\n", q, *p);
  return 0;
}
\end{lstlisting}
\begin{itemize}
\item \verb+q+のアドレスを\verb+p+に代入
\item \verb+q+と\verb+*p+の値を出力 → 両者とも200
\end{itemize}
  \end{itemize}
\end{frame}

\begin{frame}[t,fragile]{ポインタの例(2)}
  \begin{itemize}
    \setlength{\itemsep}{1em}
  \item 例3.5.2 (ハンドブック3.5節)
\begin{lstlisting}
#include <stdio.h>
int main() {
  int *p;
  int q;
  p = &q;
  *p = 300;
  printf("q is %d and *p is %d.\n", q, *p);
  return 0;
}
\end{lstlisting}
\begin{itemize}
\item \verb+q+のアドレスを\verb+p+に代入
\item \verb+*p+に300を代入 (ここで\verb+q=300;+と書いても等価)
\item \verb+q+と\verb+*p+の値を出力 → 両者とも300
\end{itemize}
  \end{itemize}
\end{frame}

\begin{frame}[t,fragile]{関数呼び出し(ポインタ渡し)}
  \begin{itemize}
    \setlength{\itemsep}{1em}
  \item 例3.6.4 (ハンドブック3.6節)
\begin{lstlisting}
#include <stdio.h>
void division(int divident, int divisor, int *quotient,
              int *residual) {
  *quotient = divident / divisor;
  *residual = divident % divisor;
}
int main() {
  int josuu = 3;
  int hi_josuu = 13;
  int shou, amari;
  division(hi_josuu, josuu, &shou, &amari);
  printf("%d / %d = %d ... %d\n", hi_josuu, josuu,
         shou, amari);
}
\end{lstlisting}
  \end{itemize}
\end{frame}

\begin{frame}[t,fragile]{間違った例(値渡し)}
  \begin{itemize}
    \setlength{\itemsep}{1em}
  \item 例3.6.5 (ハンドブック3.6節)
\begin{lstlisting}
#include <stdio.h>
void division(int divident, int divisor, int quotient,
  int residual) {
  quotient = divident / divisor;
  residual = divident % divisor;
}
int main() {
  int josuu = 3;
  int hi_josuu = 13;
  int shou, amari;
  division(hi_josuu, josuu, shou, amari);
  printf("%d / %d = %d ... %d\n", hi_josuu, josuu,
         shou, amari);
}
\end{lstlisting}
\begin{itemize}
\item 誤った答えが出力される。なぜ?
\end{itemize}
  \end{itemize}
\end{frame}

\begin{frame}[t,fragile]{一次元配列}
  \begin{itemize}
    \setlength{\itemsep}{1em}
  \item (静的)一次元配列 (ハンドブック3.3.1節)
\begin{lstlisting}
double v[10];
v[0] = 1.0;
v[1] = 2.0;
...
\end{lstlisting}
    要素数はコンパイル時にすでに決まっている定数でなければならない
  \item (動的)一次元配列 (ハンドブック3.11節)
\begin{lstlisting}
double *v; /* ポインタ */
v = (double*)malloc((size_t)(10 * sizeof(double));
...
free(v); /* 確保した領域を開放 */
\end{lstlisting}
実行時に要素数を指定可能
  \end{itemize}
\end{frame}

\begin{frame}[t,fragile]{ポインタと一次元配列}
  \begin{itemize}
    \setlength{\itemsep}{1em}
  \item 一次元配列を表す変数は、(実は)最初の要素を指すポインタ  (ハンドブック3.5.3節)
    \begin{itemize}
    \item \verb+v+ と \verb+&v[0]+ は等価
    \item \verb^(v+2)^ と \verb^&v[2]^ は等価
    \item \verb+*v+ と \verb+v[0]+ は等価
    \item \verb^*(v+2)^ と \verb^v[2]^ は等価
    \item \verb^(v+2)[3]^ は?
    \end{itemize}
  \item C言語では配列の添字は0から始まることに注意
  \item \verb^double v[10];^ と宣言した場合、\verb^v[0]^ 〜 \verb^v[9]^ の10個の要素を持つ配列が作られる。\verb^v[10]^ は存在しない。値を代入したり参照しようとするとエラーとなる
  \item ポインタ確認プログラム: \href{https://github.com/todo-group/computer-experiments/blob/master/exercise/matrix/pointer.c}{pointer.c}
  \end{itemize}
\end{frame}

\begin{frame}[t,fragile]{二次元配列}
  \begin{itemize}
    \setlength{\itemsep}{1em}
  \item C言語では、二次元配列は一次元配列の先頭をさす(ポインタ)の配列として表される(と理解しておけば良い)
  \item \verb+m[i]+は、要素\verb+m[i][0]+を指すポインタ
    \begin{itemize}
    \item \verb+m+ と \verb+&m[0]+ は等価 (\verb+&m[0][0]+ ではない)
    \item \verb+m[0]+ と \verb+&m[0][0]+ は等価
    \item \verb+m[2]+ と \verb+&m[2][0]+ は等価
    \item \verb^(m+2)^ と \verb^&m[2]^ は等価
    \item \verb^(*(m+2))[3]^ と \verb^*(*(m+2)+3)^ と \verb^m[2][3]^ は等価
    \item \verb^*(m+2)[3]^ と \verb^*((m+2)[3])^ と \verb^*(m[5])^ と\verb^m[5][0]^ は等価
    \item \verb^[]^は\verb^*^よりも強い
    \end{itemize}
  \item ポインタ確認プログラム: \href{https://github.com/todo-group/computer-experiments/blob/master/exercise/matrix/pointer.c}{pointer.c}
  \end{itemize}
\end{frame}

\begin{frame}[t,fragile]{動的二次元配列の確保}
  \begin{itemize}
    \setlength{\itemsep}{1em}
  \item 各行を表す配列とそれぞれの先頭アドレスを保持する配列の二種類が必要
\begin{lstlisting}
double **a;
m = 10;  
n = 10;  
a = (double**)malloc((size_t)(m * sizeof(double*));
for (int i = 0; i < m; ++i)
  a[i] = (double*)malloc((size_t)(n * sizeof(double));
\end{lstlisting}
\item 各行を保持する配列が、メモリ上で連続に確保される保証はない
\item 行列用のライブラリ(LAPACK等)を使うときに問題となる
  \end{itemize}
\end{frame}

\begin{frame}[t,fragile]{動的二次元配列の確保}
  \begin{itemize}
    \setlength{\itemsep}{1em}
  \item 二次元配列の要素を格納する長い配列を用意する
\begin{lstlisting}
double **a;
m = 10;  
n = 10;  
a = (double**)malloc((size_t)(m * sizeof(double*));
a[0] = (double*)malloc((size_t)(m*n * sizeof(double));
for (int i = 1; i < m; ++i)
  a[i] = a[i-1] + n;
\end{lstlisting}
  \item 開放は逆の順序で行う
\begin{lstlisting}
free(a[0]);
free(a);
\end{lstlisting}
  \end{itemize}
\end{frame}

\begin{frame}[t,fragile]{Column-majorとraw-major}
  \begin{itemize}
    % \setlength{\itemsep}{1em}
  \item CとFortranで、二次元配列のメモリ上での並びが違う \\
    Cはrow-major: {\tt a[0][0], a[0][1], a[0][2], $\cdots$} \\
    Fortranはcolumn-major: {\tt a(1,1), a(2,1), a(3,1), $\cdots$}
  \item 多くの線形代数ライブラリはFortranで書かれている
  \item Cで作成した行列をFortranに渡すと転置されてしまう

    {\tt a[i][j]}はFortranでは行列の(j,i)成分と解釈される
  \item あらかじめ転置して(i,j)成分を{\tt a[j][i]}にセットすれば良い
  \item C言語のマクロを使うと(少し?)便利
\begin{lstlisting}
#define mat_elem(mat, i, j) (mat)[j][i]
\end{lstlisting}
このマクロを使うと(i,j)成分の操作は以下のように書ける
\begin{lstlisting}
mat_elem(a, i, j) = ...;
\end{lstlisting}
  \end{itemize}
\end{frame}

\begin{frame}[t,fragile]{動的二次元配列の確保}
  \begin{itemize}
    %\setlength{\itemsep}{1em}
  \item Column-major形式の二次元配列の確保({\tt alloc\_dmatrix})、開放({\tt free\_dmatrix})、出力({\tt print\_dmatrix})、読み込み({\tt read\_dmatrix})を行うためのユーティリティ関数、(i,j)成分にアクセスするためのマクロ({\tt mat\_elem})他を準備
  \item ソースコード: \href{https://github.com/todo-group/computer-experiments/blob/master/exercise/matrix/cmatrix.h}{cmatrix.h}
  \item 使用例
\begin{lstlisting}
#include "cmatrix.h"
...
double **mat;
mat = alloc_dmatrix(m, n);
mat_elem(mat, 1, 3) = 5.0;
...
free_dmatrix(mat);
\end{lstlisting}
  \item サンプルコード: \href{https://github.com/todo-group/computer-experiments/blob/master/exercise/matrix/matrix_example.c}{matrix\_example.c}
  \end{itemize}
\end{frame}

\begin{frame}[t,fragile]{BLASライブラリ}
  \begin{itemize}
    \setlength{\itemsep}{1em}
  \item 行列・行列積、行列・ベクトル積などを高速に行う最適化された関数群
  \item 行列・行列積を計算するサブルーチン {\tt dgemm} \\
    \url{http://www.netlib.org/lapack/explore-html/d7/d2b/dgemm_8f.html}
    \begin{itemize}
    \item $C = \alpha A \times B + \beta C$ を計算
    \item BLASもFortranで書かれている
    \end{itemize}
  \item 例: \href{https://github.com/todo-group/computer-experiments/blob/master/exercise/matrix/multiply.c}{multiply.c}, \href{https://github.com/todo-group/computer-experiments/blob/master/exercise/matrix/multiply_dgemm.c}{multiply\_dgemm.c}
  \end{itemize}
\end{frame}

\begin{frame}[t,fragile]{LAPACK (Linear Algebra PACKage)}
  \begin{itemize}
    %\setlength{\itemsep}{1em}
  \item 線形計算のための高品質な数値計算ライブラリ
    \begin{itemize}
    \item \url{http://www.netlib.org/lapack}
    \item 線形方程式、固有値問題、特異値問題、線形最小二乗問題など
    \item (FFT 高速フーリエ変換は入っていない)
    \item LAPACK自体はFortranで書かれている
    \end{itemize}
  \item ほぼ全てのPC、ワークステーション、スーパーコンピュータで利用可 (インストール済)
  \item Netlibでソースが公開されているリファレンス実装は遅いが、それぞれのベンダー(Intel、Fujitsu、etc)による最適化されたLAPACKが用意されている場合が多い(MKL、SSL2、etc)
  \item LAPACKを使うことにより、高速で信頼性が高く、ポータブルなコードを書くことが可能になる
  \end{itemize}
\end{frame}

\begin{frame}[t,fragile]{LAPACKによる連立一次方程式の求解}
  \begin{itemize}
    \setlength{\itemsep}{1em}
  \item LU分解を行うサブルーチン {\tt dgetrf} \\
    \url{http://www.netlib.org/lapack/explore-html/d3/d6a/dgetrf_8f.html}
  \item Fortranによる関数宣言
\begin{lstlisting}
subroutine dgetrf(integer M, integer N,
         double precision, dimension(lda, *) A,
         integer LDA, integer, dimension(*) IPIV,
         integer INFO)
\end{lstlisting}
\item {\tt A}: 左辺の行列、{\tt N,M}: 次元、{\tt IPIV}: 選択されたピボット行のリスト、{\tt lda}: 通常{\tt M} (行数)と同じで良い
  \end{itemize}
\end{frame}

\begin{frame}[t,fragile]{LAPACKによる連立一次方程式の求解}
  \begin{itemize}
    \setlength{\itemsep}{1em}
  \item C言語から呼び出すための関数宣言を作成 (ハンドブック3.6.4節)
\begin{lstlisting}
void dgetrf_(int *M, int *N, double *A,
             int *LDA, int*IPIV, int *INFO);
\end{lstlisting}
関数名は全て小文字。関数名の最後に {\tt \_} (下線)を付ける
\item LU分解の例
\begin{lstlisting}
m = 10;
m = 10;
lda = 10;
dgetrf_(&M, &N, mat_ptr(A), &M, mat_ptr(IPIV), &INFO);
\end{lstlisting}
完全なソースコード: \href{https://github.com/todo-group/computer-experiments/blob/master/exercise/linear_system/lu_decomp.c}{lu\_decomp.c}
  \end{itemize}
\end{frame}

\begin{frame}[t,fragile]{CからFortranのライブラリを呼び出す際の注意事項}
  \begin{itemize}
    \setlength{\itemsep}{1em}
  \item スカラーも配列も全てポインタ渡しとする
  \item 行列やベクトルは最初の要素へのポインタを渡す
    \begin{itemize}
      \item 行列の最初の要素(0,0)へのポインタ: \verb+&a[0][0]+
      \item ベクトルの最初の要素(0)へのポインタ: \verb+&v[0]+
      \item \href{https://github.com/todo-group/computer-experiments/blob/master/exercise/matrix/cmatrix.h}{cmatrix.h}にマクロ({\tt mat\_ptr}、{\tt vec\_ptr})が準備されているのでそれぞれ、{\tt mat\_ptr(a)}、{\tt vec\_ptr(v)}と書ける
    \end{itemize}
  \item コンパイル時には{\tt -llapack -lblas}オプションを指定し、LAPACKライブラリとBLASライブラリをリンクする(ハンドブック3.1.6節)
  \end{itemize}
\end{frame}
