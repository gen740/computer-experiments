\section{講義・実習の概要}

\begin{frame}[t]{講義・実習の目的}
  \begin{itemize}
    %\setlength{\itemsep}{1em}
  \item 理論・実験を問わず、学部〜大学院〜で必要となる現代的かつ普遍的な計算機の素養を身につける
  \item {\color{gray}UNIX環境に慣れる(シェル、ファイル操作、エディタ)}
  \item {\color{gray}ネットワークの活用 (リモートログイン、バージョン管理、共同作業)}
  \item {\color{gray}プログラムの作成(C言語、コンパイラ、プログラム実行)}
  \item 基本的な数値計算アルゴリズム・数値計算の常識を学ぶ
  \item {\color{gray}科学技術文書作成に慣れる(\LaTeX, グラフ作成)}
  \item {\color{gray}(Mathematica (数式処理)、Python (スクリプト言語)等の利用)}
  \item {\color{red}物理学における具体的な問題を通して実践的な知識と経験を身につける}
  \end{itemize}
\end{frame}

\begin{frame}[t]{身に付けて欲しいこと}
  \begin{itemize}
    %\setlength{\itemsep}{1em}
  \item ツールとしてないものは自分で作る (物理の伝統)
  \item すでにあるものは積極的に再利用する (車輪の再発明をしない)
  \item 数学公式と数値計算アルゴリズムは別物
  \item 刻み幅・近似度合いを変えて何度か計算を行う
  \item グラフ化して目で見てみる
  \item 計算量(コスト)のスケーリング(次数)に気をつける
  \item 記録に残す・再現性を確保する
  \item {\color{red}問題の解き方は一通りではない}
  \end{itemize}
\end{frame}

\begin{frame}[t]{「計算機実験I」で学んだ項目}
  \begin{itemize}
    %\setlength{\itemsep}{1em}
  \item UNIX、ネットワーク、\LaTeX、gnuplot
  \item プログラミング: C言語、配列、数値計算ライブラリ
  \item 数値計算の基礎: 数値誤差、ニュートン法
  \item 常微分方程式: Euler法・Runge-Kutta法、Numerov法、シンプレクティック積分法、陰解法
  \item 連立一次方程式: ガウスの消去法、LU分解、ヤコビ法・ガウスザイデル法・SOR法
  \item 行列の対角化: Jacobi法、Householder法、べき乗法、Lancos法、特異値分解
  \item 線形回帰の基礎
  \end{itemize}
\end{frame}

\begin{frame}[t]{講義・実習内容}
  問題解決型
  \begin{itemize}
    \setlength{\itemsep}{1em}
  \item (数値対角化と)量子力学
  \item (モンテカルロ法と)統計物理
  \item (最適化と)実験データ解析
  \item スパコンと並列計算
  \end{itemize}
\end{frame}

\begin{frame}[t,fragile]{講義と実習}
  \begin{itemize}
    \setlength{\itemsep}{1em}
  \item スタッフ \href{mailto:computer@exa.phys.s.u-tokyo.ac.jp}{computer@exa.phys.s.u-tokyo.ac.jp}
    \begin{itemize}
    \item 講義: 藤堂
    \item 実習: 鈴木(早野研)、斉藤(古澤研)
    \item 実習TA: 鈴木(藤堂研M1)、井坂(古澤研M1)
    \end{itemize}
  \item 評価
    \begin{itemize}
    \item 出席(講義・実習)
    \item レポート
    \end{itemize}    
  \end{itemize}    
\end{frame}

\begin{frame}[t]{質問がある場合には、、、}
  \begin{enumerate}
    %\setlength{\itemsep}{1em}
  \item ITC-LMS の掲示板を見る
  \item ハンドブック、講義資料を確認
  \item まわりの人に質問してみる
  \item ネットで検索
  \item 計算機実験担当者(\href{mailto:computer@exa.phys.s.u-tokyo.ac.jp}{computer@exa.phys.s.u-tokyo.ac.jp}) に相談
  \end{enumerate}
  メールで質問するときに注意すべきこと
  \begin{itemize}
  \item (メールの)標題をきちんとつける、きちんと名乗る
  \item 実行環境を明示する
  \item 問題を再現する手順を明記する
  \item 関連するファイル(Cや \LaTeX のソースコード等)を添付する
  \item エラーメッセージを添付する
  \end{itemize}
\end{frame}

\begin{frame}[t,fragile]{実習環境}
  \begin{itemize}
    \setlength{\itemsep}{1em}
  \item 情報基盤センター大演習室 (iMac端末)
    \begin{itemize}
    \item Cプログラミング、\LaTeX、Gnuplot、などに利用
    \end{itemize}
  \item 計算機端末室
    \begin{itemize}
    \item 理学部4号館1215室 (iMac 16台)
    \item {\color{red}(長期休暇を除き)週7日24時間利用可}
    \end{itemize}
  \item 物理学教室ワークステーションクラスタ photon
    \begin{itemize}
    \item SSHでリモートログインして使用する(ハンドブック2.2節)
    \item あらかじめ公開鍵の登録が必要(計算機実験I 実習EX0 準備練習2,3)
    \end{itemize}
  \item MateriApps LIVE! (USBメモリで配布)
    \begin{itemize}
    \item Mac, Windows PC 上で動作する仮想UNIX環境
    \item 再度インストールを行いたい人はUSBを貸与するので申し出ること
    \end{itemize}
  \end{itemize}
\end{frame}
