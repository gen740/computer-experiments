\section{解析計算による次元削減}

\begin{frame}[t,fragile]{シュレディンガー方程式の一般解}
  \begin{itemize}
    %\setlength{\itemsep}{1em}
  \item 二重井戸ポテンシャル
    \begin{align*}
      V(x) = \begin{cases}
        \infty & \text{$x < 0$, $x > 1$} \\
        0 & \text{$0 < x < a$, $b < x < 1$} \\
        v & \text{$a < x < b$}
      \end{cases}
    \end{align*}
  \item それぞれの領域内では手で解ける
  \item 領域1 ($0 < x < a$)、領域3 ($b < x < 1$)では
    \begin{align*}
      -\frac{d^2}{dx^2}\psi(x) = E\psi(x)
    \end{align*}
  \end{itemize}
\end{frame}

\begin{frame}[t,fragile]{シュレディンガー方程式の一般解}
  \begin{itemize}
    %\setlength{\itemsep}{1em}
  \item 領域1 ($0 < x < a$)、領域3 ($b < x < 1$)における一般解
    \begin{align*}
      \psi(x) &= A_1 e^{i\sqrt{E}x} + B_1 e^{-i\sqrt{E}x} \\
      \psi(x) &= A_3 e^{i\sqrt{E}x} + B_3 e^{-i\sqrt{E}x}
    \end{align*}
    あきらかに$E>0$であるので
    \begin{align*}
      \psi(x) &= \alpha_1 \cos(\sqrt{E}x) + \beta_1 \sin(\sqrt{E}x) \\
      \psi(x) &= \alpha_3 \cos(\sqrt{E}x) + \beta_3 \sin(\sqrt{E}x)
    \end{align*}
  \end{itemize}
\end{frame}

\begin{frame}[t,fragile]{シュレディンガー方程式の一般解}
  \begin{itemize}
    %\setlength{\itemsep}{1em}
  \item 領域2 ($a < x < b$)における一般解
    \begin{align*}
      \psi(x) &= A_2 e^{i\sqrt{(E-v)}\,x} + B_1 e^{-i\sqrt{(E-v)}\,x}
    \end{align*}
  \item $E < v$の場合
    \begin{align*}
      \psi(x) &= \alpha_2 \exp(-\sqrt{(v-E)}\,x) + \beta_2 \exp(\sqrt{(v-E)}\,x)
    \end{align*}
  \item $E > v$の場合
    \begin{align*}
      \psi(x) &= \alpha_2 \cos(\sqrt{(E-v)}\,x) + \beta_2 \sin(\sqrt{(E-v)}\,x)
    \end{align*}
  \end{itemize}
\end{frame}

\begin{frame}[t,fragile]{シュレディンガー方程式の一般解}
  \begin{itemize}
    %\setlength{\itemsep}{1em}
  \item 境界条件($E < v$の場合)
    \begin{align*}
      &\alpha_1 = 0 \\
      &\alpha_1 \cos(\sqrt{E}a) + \beta_1 \sin(\sqrt{E}a) \\
      & \qquad =
      \alpha_2 \exp(-\sqrt{(v-E)}\,a) + \beta_2 \exp(\sqrt{(v-E)}\,a) \\
      &-\alpha_1 \sqrt{E} \sin(\sqrt{E}a) + \beta_1 \sqrt{E} \sin(\sqrt{E}a) \\
      & \qquad =
      - \alpha_2 \sqrt{(v-E)} \exp(-\sqrt{(v-E)}\,a) + \beta_2 \sqrt{(v-E)} \exp(\sqrt{(v-E)}\,a) \\
      &\cdots
    \end{align*}
  \item $\beta_1$, $\alpha_2$, $\beta_2$, $\alpha_3$, $\beta_3$ に関する連立方程式: $M x = 0$
  \item $5 \times 5$行列$M$は$E$の(非線形な)関数
  \item 非自明な解が存在するための条件: $\det M=0$
  \end{itemize}
\end{frame}
