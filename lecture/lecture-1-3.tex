% -*- coding: utf-8 -*-

\documentclass[10pt,dvipdfmx]{beamer}
\usepackage{tutorial}

\title{計算機実験I (第3回)}
\date{2021/04/28}

\begin{document}

\begin{frame}
  \titlepage
  \tableofcontents
\end{frame}

\begin{frame}[t]{本日の課題}
  \begin{itemize}
    %\setlength{\itemsep}{1em}
  \item (まだ完了していない人は「\href{https://utphys-comp.github.io}{計算機実験のための環境整備}」({\small \href{https://utphys-comp.github.io}{https://utphys-comp.github.io}})を参考に、必要な環境を引き続き整備する)
    
  \item 「計算機実験I」実習課題: [数値課題・アルゴリズム基礎] [常微分方程式] に取り組む
  \item 講義終了後(18:00まで)にITC-LMSで「作業レポート(2021/04/28)」を提出
  \item レポートNo.1: 提出締切2021/5/19(水) 18:00 (レポート内容・提出方法についてはITC-LMSを参照のこと)
  \item 質問は、Slackの「\#3 数値誤差」あるいは「\#4 常微分方程式」チャンネルで
  \end{itemize}
\end{frame}

\section{常微分方程式の解法}

\begin{frame}[t,fragile]{初期値問題の解法 (Euler法)}
  \begin{itemize}
    %\setlength{\itemsep}{1em}
  \item $h$を微小量として微分を差分で近似する(前進差分)
    \[
    \frac{dy}{dt} \approx \frac{y(t+h) - y(t)}{h} = f(t, y)
    \]
  \item $t=0$における$y(t)$の初期値を$y_0$、$t_n \equiv nh$、$y_n$を$y(t_n)$の近似値とおくと、
    \[
    y_{n+1}-y_n = h f( t_n, y_n)
    \]
  \item Euler法
    \begin{itemize}
    \item $y_0$からはじめて、$y_1,y_2,\cdots$を順次求めていく
    \end{itemize}
  \end{itemize}
\end{frame}

\begin{frame}[t,fragile]{高次のRunge-Kutta法}
  \begin{itemize}
    %\setlength{\itemsep}{1em}
  \item 3次Runge-Kutta法
    \[
    \begin{array}{rcl}
      k_1 & = & h f(t_n, y_n) \\
      k_2 & = & h f(t_n + \frac{2}{3}h, y_n + \frac{2}{3}k_1) \\
      k_3 & = & h f(t_n + \frac{2}{3}h, y_n + \frac{2}{3}k_2) \\
      y_{n+1} & = & y_n + \frac{1}{4}k_1 + \frac{3}{8}k_2
      + \frac{3}{8}k_3
    \end{array}
    \]
  \item 4次Runge-Kutta法
    \[
    \begin{array}{rcl}
      k_1 & = & h f(t_n, y_n) \\
      k_2 & = & h f(t_n + \frac{1}{2}h, y_n + \frac{1}{2}k_1) \\
      k_3 & = & h f(t_n + \frac{1}{2}h, y_n + \frac{1}{2}k_2) \\
      k_4 & = & h f(t_n + h, y_n + k_3) \\
      y_{n+1} & = & y_n + \frac{1}{6}k_1 + \frac{1}{3}k_2
      + \frac{1}{3}k_3 + \frac{1}{6}k_4
    \end{array}
    \]
  \item 4次までは次数と$f$の計算回数が等しい
  \end{itemize}
\end{frame}

\begin{frame}[t,fragile]{陽解法と陰解法}
  \begin{itemize}
    %\setlength{\itemsep}{1em}
  \item 陽解法: 右辺が既知の変数のみで書かれる(例: Euler法)
    \begin{itemize}
    \item プログラムがシンプル
    \end{itemize}
  \item 陰解法: 右辺にも未知変数が含まれる
    \begin{itemize}
    \item 例: 逆Euler法
      \begin{align*}
        y(t) &= y(t+h-h) = y(t+h) - h f(t+h,y(t+h)) + O(h^2) \\
        y_{n+1} &= y_n + h f(t+h,{\color{red}y_{n+1}})
      \end{align*}
    \item 数値的により安定な場合が多い
    \item 一般的には、Newton法などを使って非線形方程式を解く必要がある
    \end{itemize}
  \end{itemize}
\end{frame}


\section{Numerov法}

\begin{frame}[t,fragile]{Numerov法}
  \begin{itemize}
    %\setlength{\itemsep}{1em}
  \item Numerov法
    \begin{itemize}
    \item 二階の常微分方程式で一階の項がない場合に使える
    \item 連立微分方程式に直さずに直接二階微分方程式を解く
    \item 4次の陰解法
    \item 方程式が線形の場合は陽解法に書き直せる
    \end{itemize}
  \item 微分方程式
    \[
    \frac{d^2y}{dx^2} = f(x,y)
    \]
  $y=y(x)$を$x=x_i$のまわりでテイラー展開する。$x_{i \pm 1} = x_i \pm h$での表式は
      \[
      y(x_{i \pm 1}) = y(x_i) \pm h y'(x_i) + \frac{h^2}{2} y''(x_i) \pm \frac{h^3}{6} y'''(x_i) + \frac{h^4}{24} y''''(x_i)  + O(h^5)
      \]
  \end{itemize}
\end{frame}

\begin{frame}[t,fragile]{Numerov法}
  \begin{itemize}
    \setlength{\itemsep}{1em}
  \item 二階微分の差分近似 ($y_i \equiv y(x_i)$等と書く)
    \[
    \frac{y_{i+1} - 2 y_i + y_{i-1}}{h^2} = y''_{i} + \frac{h^2}{12} y''''_{i} + O(h^4)
    \]
  一方で、微分方程式より
    \[
    y''''_i = \frac{d^2f}{dx^2}\Big|_{x=x_i} = \frac{f_{i+1}-2f_i+f_{i-1}}{h^2} + O(h^2)
    \]
    組み合わせると
    \[
    y_{i+1} = 2y_i - y_{i-1} + \frac{h^2}{12} (f_{i+1} + 10f_{i} + f_{i-1}) + O(h^6)
    \]
  \end{itemize}
\end{frame}

\begin{frame}[t,fragile]{Numerov法}
  \begin{itemize}
    %\setlength{\itemsep}{1em}
  \item 方程式が線形の場合、$f(x,y) = -a(x) y(x)$を代入すると
    \[
    y_{i+1} = 2y_i - y_{i-1} - \frac{h^2}{12} (a_{i+1}y_{i+1} + 10a_{i}y_{i} + a_{i-1}y_{i-1}) + O(h^6)
    \]
  $y_{i+1}$を左辺に集めると、陽解法となる
    \[
    y_{i+1} = \frac{2 (1-\frac{5h^2}{12} a_i)y_i - (1 + \frac{h^2}{12} a_{i-1}) y_{i-1}}{1 + \frac{h^2}{12} a_{i+1}} + O(h^6)
    \]
  \end{itemize}
\end{frame}


\section{固有値問題}
\begin{frame}[t,fragile]{時間依存しないシュレディンガー方程式}
  \begin{itemize}
    \setlength{\itemsep}{1em}
  \item 井戸型ポテンシャル中の一粒子問題
    \begin{align*}
      \big[ -\frac{\hbar^2}{2m}\frac{d^2}{dx^2} + V(x) \big] \psi(x) = E \psi(x) \\
      V(x) = \begin{cases}
        0 & \text{$a \le x \le b$} \\ \infty & \text{otherwise}
      \end{cases}
    \end{align*}
  \item $\hbar^2/2m = 1$、$a=0$、$b=1$となるように変数変換して
    \begin{align*}
      \big( \frac{d^2}{dx^2} + E \big) \psi(x) = 0 \qquad 0 \le x \le 1
    \end{align*}
    を境界条件$\psi(0) = \psi(1) = 0$のもとで解けば良い
  \end{itemize}
\end{frame}

\begin{frame}[t,fragile]{固有値問題の解法}
  \begin{itemize}
    \setlength{\itemsep}{1em}
  \item $x_i=h \times i$ ($h=1/n$)、$x_0=0$、$x_n=1$とする
  \item $\psi(x_0)=0$、$\psi(x_1) = 1$を仮定 ($\psi'(x_0)=1/h$と与えたことに相当)
  \item $E = 0$とおく
  \item Runge-Kutta法、Numerov法などを用いて$x=x_n$まで積分
  \item $\psi(x_n)$の符号がかわるまで、$E$を少しずつ増やす
  \item 符号が変わったら、$E$の区間を半分ずつに狭めていき、$\psi(x_n)=0$となる$E$ (固有エネルギー)と$\psi(x)$ (波動関数)を得る
  \end{itemize}
\end{frame}


% -*- coding: utf-8 -*-

\section{シンプレクティック積分法}

\begin{frame}[t,fragile]{ハミルトン力学系}
  \begin{itemize}
    % \setlength{\itemsep}{1em}
  \item 時間をあらわに含まない場合のハミルトン方程式
    \[
    \frac{dq}{dt} = \frac{\partial H}{\partial p}, \ \frac{dp}{dt} = -\frac{\partial H}{\partial q}
    \]
    \begin{itemize}
    \item エネルギー保存則
      \[
      \frac{dH}{dt} = \frac{\partial H}{\partial q} \frac{dq}{dt} + \frac{\partial H}{\partial p} \frac{dp}{dt} = 0
      \]
    \item 位相空間の体積が保存(Liouvilleの定理)

      位相空間上の流れの場$\bm{v} = (\frac{dq}{dt},\frac{dp}{dt})$について
      \[
      \text{div} \bm{v} = \frac{\partial}{\partial q} \frac{dq}{dt} + \frac{\partial}{\partial p} \frac{dp}{dt} = 0
      \]
    \end{itemize}
  \item Euler法、Runge-Kutta法などはいずれの性質も満たさない
  \end{itemize}
\end{frame}

\begin{frame}[t,fragile]{シンプレクティック数値積分法(Symplectic Integrator)}
  \begin{itemize}
    %\setlength{\itemsep}{1em}
  \item 体積保存を満たす解法
  \item 例: 調和振動子$H=\frac{1}{2}(p^2+q^2)$の運動方程式
    \[
    \frac{dq}{dt} = p, \ \frac{dp}{dt} = -q
    \]
    の一方をEuler法で、他方を逆オイラー法で解く
    \begin{align*}
      q_{n+1} &= q_n + h p_n \\
      p_{n+1} &= p_n - h q_{n+1} = (1-h^2) p_n - h q_n \\
      \begin{pmatrix} q_{n+1} \\ p_{n+1} \end{pmatrix} &= \begin{pmatrix} 1 & h \\ -h & 1-h^2 \end{pmatrix} \begin{pmatrix} q_{n} \\ p_{n} \end{pmatrix}
    \end{align*}
  \end{itemize}
\end{frame}

\begin{frame}[t,fragile]{体積・エネルギーの保存}
  \begin{itemize}
    %\setlength{\itemsep}{1em}
  \item 体積保存
    \begin{align*}
      \det \begin{pmatrix} 1 & h \\ -h & 1-h^2 \end{pmatrix} = 1
    \end{align*}
  \item エネルギーの保存
    \begin{align*}
      \frac{1}{2}(p_{n+1}^2+q_{n+1}^2) + {\color{red}\frac{h}{2} p_{n+1} q_{n+1}} = \frac{1}{2}(p_{n}^2+q_{n}^2) + {\color{red}\frac{h}{2} p_{n} q_{n}}
    \end{align*}
  \item 位相空間の体積は厳密に保存
  \item エネルギーは$O(h)$の範囲で保存し続ける
  \end{itemize}
\end{frame}

\begin{frame}[t,fragile]{2次のシンプレクティック積分法}
  \begin{itemize}
    %\setlength{\itemsep}{1em}
  \item ハミルトニアンが$H(p,q) = T(p) + V(q)$の形で書けるとする
  \item リープ・フロッグ法
    \begin{align*}
      {\color{red} p(t+h/2)} &= p(t) - \frac{h}{2} \frac{\partial V(q)}{\partial q}|_{q=q(t)} \\
      {\color{blue} q(t+h)} &= q(t) + h {\color{red}p(t+h/2)} \\
      p(t+h) &= {\color{red}p(t+h/2}) - \frac{h}{2} \frac{\partial V(q)}{\partial q}|_{q=q(t+h)}
    \end{align*}
  \end{itemize}
\end{frame}

\begin{frame}[t,fragile]{シンプレクティック積分法}
  \begin{itemize}
    \setlength{\itemsep}{1em}
  \item ハミルトン力学系の満たすべき特性(位相空間の体積保存)を満たす
  \item 一般的には陰解法
  \item ハミルトニアンが$H(p,q) = T(p) + V(q)$の形で書ける場合は陽的なシンプレクティック積分法が存在する
  \item エネルギーは近似的に保存する
  \item $n$次のシンプレクティック積分法では、エネルギーは$O(h^n)$の範囲で振動(発散しない)
  \end{itemize}
\end{frame}



\begin{frame}[t]{今後の予定}
  \begin{itemize}
    % \setlength{\itemsep}{1em}
  \item 全8回 (水曜2限 10:25-11:55)
    \begin{itemize}
    \item {\color{gray} 4月7日 第1回: 環境整備・数値誤差}
    \item {\color{gray} 4月14日 もくもく会}
    \item {\color{gray} 4月21日 第2回: ニュートン法・二分法・常微分方程式}
    \item {\color{gray} 4月28日 第3回: 固有値問題・シンプレクティック積分法}
    \item 5月12日 もくもく会
    \item 5月19日 第4回: 行列演算・複素数・ライブラリ [レポートNo.1締切]
    \item 5月26日 第5回: 連立一次方程式・直接解法・反復解法
    \item 6月2日 もくもく会
    \item 6月9日 第6回: 行列の対角化 [レポートNo.2締切]
    \item 6月16日 第7回: 疎行列に対する反復法・変分法
    \item 6月23日 第8回: 特異値分解・最小二乗法
    \item 7月7日 [レポートNo.3締切]
     \end{itemize}
  \end{itemize}
\end{frame}

\end{document}
