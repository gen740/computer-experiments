%-*- coding:utf-8 -*-

\documentclass[dvipdfmx]{beamer}
\usepackage{tutorial}

\title{計算機実験II (L3) --- 転送行列・分子動力学}
\date{2017/11/24}

\begin{document}

\begin{frame}
  \titlepage
  \tableofcontents
\end{frame}

\section{多体系の統計力学}

\begin{frame}[t,fragile]{典型的な統計力学モデル}
  \begin{itemize}
    %\setlength{\itemsep}{1em}
  \item 古典粒子系
    \begin{itemize}
    \item 調和振動子 \ \ $\displaystyle H = \frac{p^2}{2m} + \frac{k}{2}x^2$
    \item 多粒子系
      \[
      H = \sum \frac{p_i^2}{2m} + \sum_{ij} V(x_i, x_j)
      \]
    \item バネビーズ模型
      \[
      H = \sum \frac{p_i^2}{2m} + \frac{k}{2} \sum_{ij} (x_i-x_j)^2
      \]
  \end{itemize}
  \item 磁性体
    \begin{itemize}
    \item イジング模型 \ \ $\displaystyle H = -J \sum_{ij} \sigma_i \sigma_j$ \ \ \ $\sigma_i = \pm 1$
    \end{itemize}
  \end{itemize}
\end{frame}

\begin{frame}[t,fragile]{多体系の統計力学}
  \begin{itemize}
    \setlength{\itemsep}{1em}
  \item カノニカル分布 \ $P(c) = \exp [- \beta H(c) ] / Z$ \ \ \ ($\beta = k_{\rm B} T$)
  \item 分配関数・自由エネルギー
    \begin{align*}
      Z(T) &= \int \exp [- \beta H(p,x) ] \, dp \, dx \qquad \text{(粒子系)} \\
      &= \sum_c \exp [- \beta H(c) ] \qquad \text{(イジング模型)} \\
      f(T) &= - \beta^{-1} \log Z(T)
    \end{align*}
  \item 物理量の期待値
    \begin{align*}
      \langle A \rangle &= Z^{-1} \int A(p,x) \exp [- \beta H(p,x) ] \, dp \, dx \qquad \text{(粒子系)} \\
      &= Z^{-1} \sum_c A(c) \exp [- \beta H(c) ] \qquad \text{(イジング模型)}
    \end{align*}
  \end{itemize}
\end{frame}

\begin{frame}[t,fragile]{多体系の統計力学}
  \begin{itemize}
    \setlength{\itemsep}{1em}
  \item 内部エネルギー
    \begin{align*}
      E &= -\frac{\partial}{\partial\beta} \log Z = Z^{-1} \sum_c H(c) \exp [- \beta H(c) ]
    \end{align*}
  \item 比熱
    \begin{align*}
      C &= \frac{1}{N} \frac{\partial E}{\partial T} = \frac{\beta^2}{N} (\langle H^2 \rangle - \langle H \rangle^2)
    \end{align*}
  \end{itemize}
\end{frame}

\begin{frame}[t,fragile]{代表的な数値計算手法}
  \begin{itemize}
    % \setlength{\itemsep}{1em}
  \item {\color{red} 数え上げ}
    \begin{itemize}
      \item 計算コスト ✕ (指数関数的)
      \item メモリコスト ○ (${\cal O}(1)$)
    \end{itemize}
  \item {\color{red} 転送行列法}
    \begin{itemize}
      \item 計算コスト △ (指数関数的)
      \item メモリコスト △ (指数関数的)
    \end{itemize}
  \item {\color{red} 分子動力学法}
    \begin{itemize}
      \item 計算コスト ○ (${\cal O}(N)$)
      \item メモリコスト ○ (${\cal O}(N)$)
      \item 統計誤差あり
    \end{itemize}
  \item マルコフ連鎖モンテカルロ法
    \begin{itemize}
      \item 計算コスト ○ (${\cal O}(N)$)
      \item メモリコスト ○ (${\cal O}(N)$)
      \item 統計誤差あり
    \end{itemize}
  \end{itemize}
\end{frame}

\section{数え上げ}

\begin{frame}[t,fragile]{厳密な数え上げ}
  \begin{itemize}
    \setlength{\itemsep}{1em}
  \item 全ての状態についてボルツマン重みを計算し足し合わせる
    \begin{itemize}
    \item 状態数 $= 2^N$ \ ($N$スピン数)
    \end{itemize}
  \item 状態を$N$ビットの整数で表す (0から$2^N-1$)
    \begin{itemize}
    \item $i$番目の格子点のスピンの状態を$i$ビット目に保存
    \item $\sigma = \pm 1$をビットの 1 or 0で表現
    \item シフト演算(\verb+>>+)とAND演算(\verb+&+)でスピン状態を取り出す
    \end{itemize}
  \item 例: \href{https://github.com/todo-group/computer-experiments/blob/master/exercise/monte_carlo/exact_counting.c}{example-2-L3/exact\_counting.c}
  \end{itemize}
\end{frame}

\begin{frame}[t,fragile]{Log-sum-exp法}
  \begin{itemize}
    %\setlength{\itemsep}{1em}
  \item Boltzmann重みは低温で非常に大きくなる
    \begin{itemize}
    \item そのまま足し合わせていくと、桁あふれの可能性
    \item C言語のdoubleで表せる最大の数 〜 $10^{308}$
    \item そのままの数ではなく、その{\color{red} 対数の値を保存}しておけばよい
    \item それらの和を計算する時、どうすればよいのか?
    \end{itemize}
  \item Log-sum-exp法
    \begin{itemize}
    \item 対数の値から元の値に戻すと桁があふれるので、{\color{red} 途中で大きな数が出てこないように}する
    \item $a > b$ の時: $e^c = e^a + e^b = e^a (1 + e^{-(a-b)})$
    \item 対数を取ると $c = log(e^a + e^b) = a + log(1 + e^{-(a-b)})$
    \item 右辺の指数関数の中身はかならず負
    \item $a < b$の時も同様に考える。まとめると \\
      $\color{red} c = max(a,b) + log(1 + e^{-|a-b|})$
    \end{itemize}
  \item 別の方法: 基底状態のエネルギーが分かっている場合には$e^{-\beta E_0}$で重みを規格化しておく
  \end{itemize}
\end{frame}

\section{転送行列法}

\begin{frame}[t,fragile]{転送行列: 一次元イジング模型}
  \begin{itemize}
    \setlength{\itemsep}{1em}
  \item ハミルトニアン: $H = - \sum_i \sigma_i \sigma_{i+1}$
  \item 分配関数
    \[
    Z = \sum_{\sigma_1}\sum_{\sigma_2}\cdots\sum_{\sigma_L} e^{\beta \sigma_1 \sigma_2} e^{\beta \sigma_2 \sigma_3} \cdots e^{\beta \sigma_L \sigma_1}
    \]
    \begin{itemize}
    \item $e^{\beta \sigma_1 \sigma_2}$は4通りの値を持つ→$2\times2$行列$T_{\sigma_1 \sigma_2}$の形に書くと
      \begin{align*}
      \sum_{\sigma_2} e^{\beta \sigma_1 \sigma_2} e^{\beta \sigma_2 \sigma_3} &= \sum_{\sigma_2} T_{\sigma_1 \sigma_2} T_{\sigma_2 \sigma_3} = (T^2)_{\sigma_1 \sigma_3} \\
      Z &= {\rm tr} T^L
      \end{align*}
    \end{itemize}
  \end{itemize}
\end{frame}

\begin{frame}[t,fragile]{分配関数の計算方法}
  \begin{itemize}
    \setlength{\itemsep}{1em}
  \item ${\rm tr} T^L$の計算方法
    \begin{enumerate}
    \item $L$個の行列$T$を掛けて、最後にトレースを取る (行列・行列積)
    \item $(1,0)^t$と$(0,1)^t$にそれぞれ行列$T$を$L$回掛けて、それぞれの第1成分と第2成分を足し合わせる (行列・ベクトル積)
    \item 行列$T$を固有値分解: $T = U\Lambda U^{-1}$しておくと
      \[
        {\rm tr} T^L = {\rm tr} (U \Lambda U^{-1})^L = {\rm tr} U \Lambda^L U^{-1} = {\rm tr} \Lambda^L = \lambda_1^L + \lambda_2^L
        \]
        特に$|\lambda_1| > |\lambda_2|$とすると$L\rightarrow\infty$で
        \[
          {\rm tr} T^L \simeq \lambda_1^L
          \]
    \end{enumerate}
  \end{itemize}
\end{frame}

\begin{frame}[t,fragile]{二次元正方格子への拡張}
  \begin{itemize}
    \setlength{\itemsep}{1em}
  \item $L \times M$の正方格子
    \begin{itemize}
    \item 第$i$列のスピンをまとめて$s_i = ( \sigma_{i,1}, \sigma_{i,2}, \cdots, \sigma_{i,M})$とする
    \item $s_i$は$2^M$通りの値をとる
    \end{itemize}
  \item 転送行列
    \begin{itemize}
    \item 横方向の相互作用: $\exp[\beta \sum_j \sigma_{i,j} \sigma_{i+1,j}]$ \ ($2^M \times 2^M$の密行列)
    \item 縦方向の相互作用: $\exp[\beta \sum_j \sigma_{i,j} \sigma_{i,j+1}]$ \ ($2^M \times 2^M$の対角行列)
    \item それぞれ$U$, $D$と表すと
      \[
      Z = {\rm tr} \, DUDU \cdots DU
      \]
    \item さらに$T=D^{1/2}TD^{1/2}$と定義すると、$T$は対称行列となり
      \[
      Z = {\rm tr} \, T^L
      \]
    \end{itemize}
  \end{itemize}
\end{frame}

\begin{frame}[t,fragile]{計算コスト}
  \begin{itemize}
    \setlength{\itemsep}{1em}
  \item 必要メモリと必要計算量の見積もり
    \begin{enumerate}
    \item $L$個の行列$T$を掛けて、最後にトレースを取る \\
      メモリ〜$(2^M)^2$, 計算量〜$L(2^M)^3$
    \item $2^M$個の基底ベクトルにそれぞれ行列$T$を$L$回掛けて、それぞれの対応する成分を足し合わせる \\
      メモリ〜$(2^M)^2$, 計算量〜$L(2^M)^3$
    \item 行列$T$を固有値分解(Householder法) \\
      メモリ〜$(2^M)^2$, 計算量〜$(2^M)^3$
    \end{enumerate}
  \end{itemize}
\end{frame}

\begin{frame}[t,fragile]{疎行列分解}
  \begin{itemize}
    \setlength{\itemsep}{1em}
  \item 行列$U$は疎行列の積に分解できる: $U = U_1 U_2 \cdots U_M$ \\
    ここで
    \[
    (U_k)_{s_i, s_{i+1}} = \exp[\beta \sigma_{i,k} \sigma_{i+1,k}] \prod_{j \ne k} \delta_{\sigma_{i,j} \sigma_{i+1,j}}
    \]
    \begin{itemize}
    \item 各行各列に非零の要素は2つだけ
    \item $U_k$の要素はその場で簡単に計算できる(メモリコスト=0)
    \item ベクトルと行列$U_k$の積: 計算量〜$2 \times 2^M$
    \item 密行列と行列$U_k$の積: 計算量〜$2 \times (2^M)^2$
    \end{itemize}
  \end{itemize}
\end{frame}

\begin{frame}[t,fragile]{計算コスト再見積もり}
  \begin{itemize}
    \setlength{\itemsep}{1em}
  \item 必要メモリと必要計算量の見積もり
    \begin{enumerate}
    \item $L$個の行列$T$を掛けて、最後にトレースを取る \\
      メモリ〜$(2^M)^2$, 計算量〜$LM(2^M)^2$
    \item $2^M$個の基底ベクトルにそれぞれ行列$T$を$L$回掛けて、それぞれの対応する成分を足し合わせる \\
      メモリ〜$2^M$, 計算量〜$LM(2^M)^2$
    \item 行列$T$を固有値分解
      \begin{itemize}
        \item $L$有限の場合 \\
          メモリ〜$(2^M)^2$, 計算量〜$(2^M)^3$
        \item $L\rightarrow\infty$の極限: 最大固有値$\lambda_1$のみ必要 \\
          メモリ〜$2^M$, 計算量〜$O(100)\times M(2^M)$
      \end{itemize}
    \end{enumerate}
  \end{itemize}
\end{frame}

\begin{frame}[t,fragile]{BLASライブラリ}
  \begin{itemize}
    %\setlength{\itemsep}{1em}
  \item 行列・行列積、行列・ベクトル積などを高速に行う最適化された関数群
  \item コンパイル時に {\tt -lblas} オプションを指定してBLASライブラリをリンクする
  \item 行列・行列積を計算するサブルーチン {\tt dgemm} \\
    \url{http://www.netlib.org/lapack/explore-html/d7/d2b/dgemm_8f.html}
    \begin{itemize}
    \item $C = \alpha A \times B + \beta C$ を計算
    \item BLASもFortranで書かれているので行列が転置される
    \item 行列$A$と$B$については、引数{\tt TRANSA}、{\tt TRANSB}で転置するかどうか指定可能だが、行列$C$については不可
    \item $C = A \times B$のかわりに$C^t = B^t \times A^t$を計算すればよい
    \end{itemize}
  \item 例: \href{https://github.com/todo-group/computer-experiments/blob/master/exercise/linear_system/multiply.c}{example-2-L3/multiply.c}, \href{https://github.com/todo-group/computer-experiments/blob/master/exercise/linear_system/multiply_dgemm.c}{example-2-L3/multiply\_dgemm.c}
  \end{itemize}
\end{frame}

\section{C言語における行列・LAPACKの利用}

\begin{frame}[t,fragile]{一次元配列}
  \begin{itemize}
    \setlength{\itemsep}{1em}
  \item (静的)一次元配列 (ハンドブック3.3.1節)
\begin{lstlisting}
double v[10];
v[0] = 1.0;
v[1] = 2.0;
...
\end{lstlisting}
    要素数はコンパイル時にすでに決まっている定数でなければならない
  \item (動的)一次元配列 (ハンドブック3.11節)
\begin{lstlisting}
double *v; /* ポインタ */
v = (double*)malloc((size_t)(10 * sizeof(double));
...
free(v); /* 確保した領域を開放 */
\end{lstlisting}
実行時に要素数を指定可能
  \end{itemize}
\end{frame}

\begin{frame}[t,fragile]{ポインタと一次元配列}
  \begin{itemize}
    \setlength{\itemsep}{1em}
  \item 一次元配列を表す変数は、(実は)最初の要素を指すポインタ  (ハンドブック3.5.3節)
    \begin{itemize}
    \item \verb+v+ と \verb+&v[0]+ は等価
    \item \verb^(v+2)^ と \verb^&v[2]^ は等価
    \item \verb+*v+ と \verb+v[0]+ は等価
    \item \verb^*(v+2)^ と \verb^v[2]^ は等価
    \item \verb^(v+2)[3]^ は?
    \end{itemize}
  \item C言語では配列の添字は0から始まることに注意
  \item \verb^double v[10];^ と宣言した場合、\verb^v[0]^ 〜 \verb^v[9]^ の10個の要素を持つ配列が作られる。\verb^v[10]^ は存在しない。値を代入したり参照しようとするとエラーとなる
  \item ポインタ確認プログラム: \href{https://github.com/todo-group/computer-experiments/blob/master/exercise/matrix/pointer.c}{pointer.c}
  \end{itemize}
\end{frame}

\begin{frame}[t,fragile]{二次元配列}
  \begin{itemize}
    \setlength{\itemsep}{1em}
  \item C言語では、二次元配列は一次元配列の先頭をさす(ポインタ)の配列として表される(と理解しておけば良い)
  \item \verb+m[i]+は、要素\verb+m[i][0]+を指すポインタ
    \begin{itemize}
    \item \verb+m+ と \verb+&m[0]+ は等価 (\verb+&m[0][0]+ ではない)
    \item \verb+m[0]+ と \verb+&m[0][0]+ は等価
    \item \verb+m[2]+ と \verb+&m[2][0]+ は等価
    \item \verb^(m+2)^ と \verb^&m[2]^ は等価
    \item \verb^(*(m+2))[3]^ と \verb^*(*(m+2)+3)^ と \verb^m[2][3]^ は等価
    \item \verb^*(m+2)[3]^ と \verb^*((m+2)[3])^ と \verb^*(m[5])^ と\verb^m[5][0]^ は等価
    \item \verb^[]^は\verb^*^よりも強い
    \end{itemize}
  \item ポインタ確認プログラム: \href{https://github.com/todo-group/computer-experiments/blob/master/exercise/matrix/pointer.c}{pointer.c}
  \end{itemize}
\end{frame}

\begin{frame}[t,fragile]{動的二次元配列の確保}
  \begin{itemize}
    \setlength{\itemsep}{1em}
  \item 各行を表す配列とそれぞれの先頭アドレスを保持する配列の二種類が必要
\begin{lstlisting}
double **a;
m = 10;  
n = 10;  
a = (double**)malloc((size_t)(m * sizeof(double*));
for (int i = 0; i < m; ++i)
  a[i] = (double*)malloc((size_t)(n * sizeof(double));
\end{lstlisting}
\item 各行を保持する配列が、メモリ上で連続に確保される保証はない
\item 行列用のライブラリ(LAPACK等)を使うときに問題となる
  \end{itemize}
\end{frame}

\begin{frame}[t,fragile]{動的二次元配列の確保}
  \begin{itemize}
    \setlength{\itemsep}{1em}
  \item 二次元配列の要素を格納する長い配列を用意する
\begin{lstlisting}
double **a;
m = 10;  
n = 10;  
a = (double**)malloc((size_t)(m * sizeof(double*));
a[0] = (double*)malloc((size_t)(m*n * sizeof(double));
for (int i = 1; i < m; ++i)
  a[i] = a[i-1] + n;
\end{lstlisting}
  \item 開放は逆の順序で行う
\begin{lstlisting}
free(a[0]);
free(a);
\end{lstlisting}
  \end{itemize}
\end{frame}

\begin{frame}[t,fragile]{Column-majorとraw-major}
  \begin{itemize}
    % \setlength{\itemsep}{1em}
  \item CとFortranで、二次元配列のメモリ上での並びが違う \\
    Cはrow-major: {\tt a[0][0], a[0][1], a[0][2], $\cdots$} \\
    Fortranはcolumn-major: {\tt a(1,1), a(2,1), a(3,1), $\cdots$}
  \item 多くの線形代数ライブラリはFortranで書かれている
  \item Cで作成した行列をFortranに渡すと転置されてしまう

    {\tt a[i][j]}はFortranでは行列の(j,i)成分と解釈される
  \item あらかじめ転置して(i,j)成分を{\tt a[j][i]}にセットすれば良い
  \item C言語のマクロを使うと(少し?)便利
\begin{lstlisting}
#define mat_elem(mat, i, j) (mat)[j][i]
\end{lstlisting}
このマクロを使うと(i,j)成分の操作は以下のように書ける
\begin{lstlisting}
mat_elem(a, i, j) = ...;
\end{lstlisting}
  \end{itemize}
\end{frame}

\begin{frame}[t,fragile]{動的二次元配列の確保}
  \begin{itemize}
    %\setlength{\itemsep}{1em}
  \item Column-major形式の二次元配列の確保({\tt alloc\_dmatrix})、開放({\tt free\_dmatrix})、出力({\tt print\_dmatrix})、読み込み({\tt read\_dmatrix})を行うためのユーティリティ関数、(i,j)成分にアクセスするためのマクロ({\tt mat\_elem})他を準備
  \item ソースコード: \href{https://github.com/todo-group/computer-experiments/blob/master/exercise/matrix/cmatrix.h}{cmatrix.h}
  \item 使用例
\begin{lstlisting}
#include "cmatrix.h"
...
double **mat;
mat = alloc_dmatrix(m, n);
mat_elem(mat, 1, 3) = 5.0;
...
free_dmatrix(mat);
\end{lstlisting}
  \item サンプルコード: \href{https://github.com/todo-group/computer-experiments/blob/master/exercise/matrix/matrix_example.c}{matrix\_example.c}
  \end{itemize}
\end{frame}

\begin{frame}[t,fragile]{BLASライブラリ}
  \begin{itemize}
    \setlength{\itemsep}{1em}
  \item 行列・行列積、行列・ベクトル積などを高速に行う最適化された関数群
  \item 行列・行列積を計算するサブルーチン {\tt dgemm} \\
    \url{http://www.netlib.org/lapack/explore-html/d7/d2b/dgemm_8f.html}
    \begin{itemize}
    \item $C = \alpha A \times B + \beta C$ を計算
    \item BLASもFortranで書かれている
    \end{itemize}
  \item 例: \href{https://github.com/todo-group/computer-experiments/blob/master/exercise/matrix/multiply.c}{multiply.c}, \href{https://github.com/todo-group/computer-experiments/blob/master/exercise/matrix/multiply_dgemm.c}{multiply\_dgemm.c}
  \end{itemize}
\end{frame}

\begin{frame}[t,fragile]{LAPACK (Linear Algebra PACKage)}
  \begin{itemize}
    %\setlength{\itemsep}{1em}
  \item 線形計算のための高品質な数値計算ライブラリ
    \begin{itemize}
    \item \url{http://www.netlib.org/lapack}
    \item 線形方程式、固有値問題、特異値問題、線形最小二乗問題など
    \item (FFT 高速フーリエ変換は入っていない)
    \item LAPACK自体はFortranで書かれている
    \end{itemize}
  \item ほぼ全てのPC、ワークステーション、スーパーコンピュータで利用可 (インストール済)
  \item Netlibでソースが公開されているリファレンス実装は遅いが、それぞれのベンダー(Intel、Fujitsu、etc)による最適化されたLAPACKが用意されている場合が多い(MKL、SSL2、etc)
  \item LAPACKを使うことにより、高速で信頼性が高く、ポータブルなコードを書くことが可能になる
  \end{itemize}
\end{frame}

\begin{frame}[t,fragile]{LAPACKによる連立一次方程式の求解}
  \begin{itemize}
    \setlength{\itemsep}{1em}
  \item LU分解を行うサブルーチン {\tt dgetrf} \\
    \url{http://www.netlib.org/lapack/explore-html/d3/d6a/dgetrf_8f.html}
  \item Fortranによる関数宣言
\begin{lstlisting}
subroutine dgetrf(integer M, integer N,
         double precision, dimension(lda, *) A,
         integer LDA, integer, dimension(*) IPIV,
         integer INFO)
\end{lstlisting}
\item {\tt A}: 左辺の行列、{\tt N,M}: 次元、{\tt IPIV}: 選択されたピボット行のリスト、{\tt lda}: 通常{\tt M} (行数)と同じで良い
  \end{itemize}
\end{frame}

\begin{frame}[t,fragile]{LAPACKによる連立一次方程式の求解}
  \begin{itemize}
    \setlength{\itemsep}{1em}
  \item C言語から呼び出すための関数宣言を作成 (ハンドブック3.6.4節)
\begin{lstlisting}
void dgetrf_(int *M, int *N, double *A,
             int *LDA, int*IPIV, int *INFO);
\end{lstlisting}
関数名は全て小文字。関数名の最後に {\tt \_} (下線)を付ける
\item LU分解の例
\begin{lstlisting}
m = 10;
m = 10;
lda = 10;
dgetrf_(&M, &N, mat_ptr(A), &M, mat_ptr(IPIV), &INFO);
\end{lstlisting}
完全なソースコード: \href{https://github.com/todo-group/computer-experiments/blob/master/exercise/linear_system/lu_decomp.c}{lu\_decomp.c}
  \end{itemize}
\end{frame}

\begin{frame}[t,fragile]{CからFortranのライブラリを呼び出す際の注意事項}
  \begin{itemize}
    \setlength{\itemsep}{1em}
  \item スカラーも配列も全てポインタ渡しとする
  \item 行列やベクトルは最初の要素へのポインタを渡す
    \begin{itemize}
      \item 行列の最初の要素(0,0)へのポインタ: \verb+&a[0][0]+
      \item ベクトルの最初の要素(0)へのポインタ: \verb+&v[0]+
      \item \href{https://github.com/todo-group/computer-experiments/blob/master/exercise/matrix/cmatrix.h}{cmatrix.h}にマクロ({\tt mat\_ptr}、{\tt vec\_ptr})が準備されているのでそれぞれ、{\tt mat\_ptr(a)}、{\tt vec\_ptr(v)}と書ける
    \end{itemize}
  \item コンパイル時には{\tt -llapack -lblas}オプションを指定し、LAPACKライブラリとBLASライブラリをリンクする(ハンドブック3.1.6節)
  \end{itemize}
\end{frame}


\section{分子動力学法}

\begin{frame}[t,fragile]{古典多粒子系}
  \begin{itemize}
    \setlength{\itemsep}{1em}
  \item ハミルトニアン
    \[
    H = \sum \frac{p_i^2}{2m} + \sum_{ij} V(x_i, x_j)
    \]
  \item 分配関数
    \[
    Z(T) = \int \exp [- \beta H(p,x) ] \, dp \, dx
    \]
    \begin{itemize}
    \item 運動量$p$に関する積分は簡単に実行できる(ガウス積分)
    \item 位置$x$に関する積分: 数値積分、マルコフ連鎖モンテカルロ法、分子動力学法
    \end{itemize}
  \end{itemize}
\end{frame}

\begin{frame}[t,fragile]{数値積分}
  \begin{itemize}
    \setlength{\itemsep}{1em}
  \item 一次元の場合
    \begin{itemize}
    \item 台形公式: 積分区間$(a,b)$を$M$個の幅$\Delta=(b-a)/M$の区間に分けて線形関数で近似
      \begin{align*}
        \int_a^b f(x) \, dx &= \sum_{i=1}^M \int_{a+\Delta (i-1)}^{a+\Delta i} f(x) \, dx \\
        &\simeq \sum_{i=1}^M \Delta \frac{f({a+\Delta (i-1)}) + f({a+\Delta i})}{2}
      \end{align*}
    \item より高次の公式: シンプソンの公式(区間を二次式で近似)など
    \end{itemize}
    \item 高次元になると区間の数が指数関数的に増える (次元の呪い)
  \end{itemize}
\end{frame}

\begin{frame}[t,fragile]{マルコフ連鎖モンテカルロ法}
  \begin{itemize}
    \setlength{\itemsep}{1em}
  \item メトロポリス法
    \begin{itemize}
    \item 一つの粒子を選ぶ(位置$x$)
    \item 新しい位置の候補をある確率分布に従って選ぶ($x'$)
    \item ポテンシャルエネルギーの変化量($\Delta E$)を計算し、確率$P=\min(1,exp[-\beta\Delta E])$で新しい位置を採択
    \end{itemize}
  \item 新しい位置の選び方
    \begin{itemize}
    \item 大きく変えすぎると棄却率が増加
    \item もとの位置を中心とする局所的な分布
    \item $\sigma={\cal O}(1)$の標準偏差をもつ正規分布など: $N(x, \sigma^2)$
    \end{itemize}
  \end{itemize}
\end{frame}

\begin{frame}[t,fragile]{Box-Muller法}
  \begin{itemize}
    \setlength{\itemsep}{1em}
  \item 一様分布乱数から正規分布乱数を生成する方法
    \begin{itemize}
    \item 2次元の(標準)ガウス分布を考える
      \[
      f(x,y)\,dx\,dy= \frac{1}{2\pi} e^{-(x^2+y^2)/2} \,dx\,dy
      \]
    \item 極座標$(r,\theta)$に変換 ($x=r\cos\theta$, $y=r\sin\theta$)
      \[
      \frac{1}{2\pi} e^{-(x^2+y^2)/2} \,dx\,dy = \frac{1}{2\pi} r \, e^{-r^2/2} \,dr\,d\theta
      \]
    \item $\theta$は$(0,2\pi)$の一様分布
    \item $r$は$f(r) = r \, e^{-r^2/2}$に従う
      \begin{align*}
        F(r) = \int_0^r f(r) \, dr = 1 - e^{-r^2/2}, \qquad F^{-1}(q) = \sqrt{- 2 \log(1-q)}
      \end{align*}
    \item 二つの一様乱数から二つの独立な正規分布乱数が生成される
    \end{itemize}
  \end{itemize}
\end{frame}

\begin{frame}[t,fragile]{分子動力学法}
  \begin{itemize}
    \setlength{\itemsep}{1em}
  \item 適当な初期条件から、運動方程式に従って位置と運動量を時間発展させる
    \begin{itemize}
    \item Euler法、Runge-Kutta法など
    \item $6N$次元の連立微分方程式
    \end{itemize}
  \item 時間発展に関する物理量の時間平均から平均を評価
    \begin{align*}
      \langle A(p,x) \rangle &= \frac{1}{Z(E)} \int A(p,x) \, \delta(H(p,x)-E) \, dp \, dx \\
      &\simeq \frac{1}{t_{\rm max}} \int_0^{t_{\rm max}} A(p(t),x(t)) \, dt
    \end{align*}
    \begin{itemize}
    \item 運動方程式では全エネルギーが保存する→ミクロカノニカル分布
    \end{itemize}
  \end{itemize}
\end{frame}

\begin{frame}[t,fragile]{Nose-Hoover熱浴}
  \begin{itemize}
    %\setlength{\itemsep}{1em}
  \item カノニカル分布の実現
    \begin{itemize}
    \item 巨大な環境(熱浴)を付ける必要がある?
    \end{itemize}
  \item Nose-Hoover法
    \begin{itemize}
    \item 熱浴をたった1つの自由度($\zeta$)だけで実現する
    \item 運動方程式
      \begin{align*}
        \dot{x}_i &= p_i / m \\
        \dot{p}_i &= F_i(x) - {\color{red}\zeta p_i} \\
        {\color{red}\dot{\zeta}} &{\color{red} = \frac{1}{\tau^2} \Big( \sum \frac{p_i^2}{m} - g k_B T\Big)} \qquad \text{($g$: 系の自由度)}
      \end{align*}
    \item $\sum \frac{p_i^2}{m}$: 全運動エネルギーの二倍 (熱平衡状態では期待値 $k_B T$)
    \item $\zeta$によりネガティブフィードバックをかける
    \item 平衡分布はカノニカル分布になる(証明略)
    \end{itemize}
  \end{itemize}
\end{frame}

\section{常微分方程式の初期値問題}

\begin{frame}[t,fragile]{準備: 微分方程式の書き換え}
  \begin{itemize}
    %\setlength{\itemsep}{1em}
  \item 2階の常微分方程式の一般形
    \[
    \frac{d^2y}{dx^2} + p(x)\frac{dy}{dx} + q(x)y = r(x)
    \]
  \item $y_1 \equiv y$, $y_2 \equiv \frac{dy}{dx}$とおくと
    \[
    \left\{
    \begin{array}{ccl}
      \frac{dy_1}{dx} & = & y_2 \\
      \frac{dy_2}{dx} & = & r(x) - p(x) y_2 - q(x) y_1
    \end{array}
    \right.
    \]
  \item さらに$\bm{y}\equiv(y_1, y_2)$, $\bm{f}(x, \bm{y})\equiv \left(y_2, r(x)-p(x)y_2 - q(x)y_1\right)$
    \[
    \frac{d\bm{y}}{dx} = \bm{f}(x, \bm{y})
    \]
  \item $n$階常微分方程式 $\Rightarrow$ $n$次元の1階常微分方程式
  \end{itemize}
\end{frame}

\begin{frame}[t,fragile]{初期値問題と境界値問題}
  \begin{itemize}
    \setlength{\itemsep}{1em}
  \item 初期値問題
    \begin{itemize}
    \item 微分方程式において、ある1点に関する全ての境界条件(初期値)が与えられているもの
    \item 質点の運動など(時系列の問題)
  \end{itemize}
  \item 境界値問題
    \begin{itemize}
    \item 複数の点に関する境界条件が与えられているもの
    \item 物体のゆがみの計算や静電場の計算など(空間的に解く問題)
  \end{itemize}
  \item 初期値問題は初期値から逐次的に解くことが可能
  \item 境界値問題は初期値問題に比べて計算法が複雑
  \end{itemize}
\end{frame}

\begin{frame}[t,fragile]{初期値問題の解法 (Euler法)}
  \begin{itemize}
    \setlength{\itemsep}{1em}
  \item 微分を差分で近似する(前進差分)
    \[
    \frac{dy}{dt} \approx \frac{y(t+\Delta t) - y(t)}{\Delta t} = f(t, y)
    \]
  \item $t=0$における$y(t)$の初期値を$y_0$、$h$を微少量、$t_n \equiv nh$、$y_n$を$y(t_n)$の近似値とおくと、
    \[
    y_{n+1}-y_n = h f( t_n, y_n)
    \]
  \item Euler法
    \begin{itemize}
    \item $y_0$からはじめて、$y_1,y_2,\cdots$を順次求めていく
    \end{itemize}
  \end{itemize}
\end{frame}

\begin{frame}[t,fragile]{Euler法の精度}
  \begin{itemize}
    \setlength{\itemsep}{1em}
  \item 微分方程式の両辺を$t_n$から$t_{n+1}$まで積分(積分方程式)
    \[
    y(t_{n+1}) - y(t_n) = \int^{t_{n+1}}_{t_n} \!\! f(t, y(t)) dt = h \int^1_0 \! f(t_n+h\tau, y(t_n+h\tau)) d\tau
    \]
  \item Euler法は、被積分関数を定数で近似することに対応
    \[
    f(t_n+h\tau, y(t_n+h\tau)) = f(t_n, y(t_n)) + O(h)
    \]
  \item $t=0$からある$t_f$まで積分すると、反復回数$N = t_f / h$
  \item $t=t_f$における誤差 $\sim N \times h \times O(h) = O(h)$
  \end{itemize}
\end{frame}

\begin{frame}[t,fragile]{Euler法の改良}
  \begin{itemize}
    \setlength{\itemsep}{1em}
  \item 積分方程式の被積分関数をもう1次高次まで展開
    \[
    f(t_n+h\tau, y(t_n+h\tau)) = f(t_n, y(t_n)) +
    \tau h
    \left\{
    \frac{\partial f}{\partial t}
    + f \frac{\partial f}{\partial y}
    \right\}_{t=t_n, y=y_n}
    \!\!\!\!\!\!\!\!\!\!\!\! + O(h^2)
    \]
  \item 積分を実行すると
    \[
    y(t_{n+1}) = y(t_n) + h f(t_n, y_n) + \frac{1}{2}h^2
    \left\{
    \frac{\partial f}{\partial t}
    + f \frac{\partial f}{\partial y}
    \right\}_{t=t_n, y=y_n}
    \!\!\!\!\!\!\!\!\!\!\!\! + O(h^3)
    \]
  \end{itemize}
\end{frame}

\begin{frame}[t,fragile]{中点法(2次Runge-Kutta法)}
  \begin{itemize}
    %\setlength{\itemsep}{1em}
  \item 2次公式
    \[
    \begin{array}{rcl}
      k_1 & = & h f(t_n, y_n) \\
      k_2 & = & h f(t_n + \frac{1}{2}h, y_n + \frac{1}{2}k_1) \\
      y_{n+1} & = & y_n + k_2
    \end{array}
    \]
  \item このとき
    \[
    k_2 = h 
    \left\{
    f(t_n, y_n)
    + \frac{1}{2}h \frac{\partial f}{\partial t}
    + \frac{1}{2}k_1 \frac{\partial f}{\partial y}
    + O(h^2)
    \right\}
    \]
  \item したがって
    \[
    y_{n+1} = y_n + h f(t_n, y_n) + \frac{1}{2}h^2
    \left\{
    \frac{\partial f}{\partial t}
    + f \frac{\partial f}{\partial y}
    \right\}_{t=t_n, y=y_n}
    \!\!\!\!\!\!\!\!\!\!\!\!+ O(h^3)
    \]
  \end{itemize}
\end{frame}

\begin{frame}[t,fragile]{高次のRunge-Kutta法}
  \begin{itemize}
    %\setlength{\itemsep}{1em}
  \item 3次Runge-Kutta法
    \[
    \begin{array}{rcl}
      k_1 & = & h f(t_n, y_n) \\
      k_2 & = & h f(t_n + \frac{2}{3}h, y_n + \frac{2}{3}k_1) \\
      k_3 & = & h f(t_n + \frac{2}{3}h, y_n + \frac{2}{3}k_2) \\
      y_{n+1} & = & y_n + \frac{1}{4}k_1 + \frac{3}{8}k_2
      + \frac{3}{8}k_3
    \end{array}
    \]
  \item 4次Runge-Kutta法
    \[
    \begin{array}{rcl}
      k_1 & = & h f(t_n, y_n) \\
      k_2 & = & h f(t_n + \frac{1}{2}h, y_n + \frac{1}{2}k_1) \\
      k_3 & = & h f(t_n + \frac{1}{2}h, y_n + \frac{1}{2}k_2) \\
      k_4 & = & h f(t_n + h, y_n + k_3) \\
      y_{n+1} & = & y_n + \frac{1}{6}k_1 + \frac{1}{3}k_2
      + \frac{1}{3}k_3 + \frac{1}{6}k_4
    \end{array}
    \]
  \item 4次までは次数と$f$の計算回数が等しい
  \end{itemize}
\end{frame}

\begin{frame}[t,fragile]{計算コストと精度}
  \begin{itemize}
    \setlength{\itemsep}{1em}
  \item 実際の計算では$f(t,y)$の計算にほとんどのコストがかかる
  \item 計算回数と計算精度の関係
    \begin{center}
      \begin{tabular}[h]{c|cccc}
        & 1次(Euler法) & 2次(中点法) & 3次 & 4次 \\
        \hline
        計算精度 & $O(h)$ & $O(h^2)$ & $O(h^3)$ & $O(h^4)$ \\
        計算回数 & $N$ & $2N$ & $3N$ & $4N$
      \end{tabular}
    \end{center}
  \item 高次のRunge-Kuttaを使う方が効率的
  \item どれくらい小さな$h$が必要となるか、前もっては分からない
  \item 刻み幅を変えて($h,h/2,h/4,\dots$)計算してみることが大事
    \begin{itemize}
    \item 誤差の評価
    \item 公式の間違いの発見
    \end{itemize}
  \end{itemize}
\end{frame}

\begin{frame}[t,fragile]{陽解法と陰解法}
  \begin{itemize}
    \setlength{\itemsep}{1em}
  \item 陽解法: 右辺が既知の変数のみで書かれる(例: Euler法)
    \begin{itemize}
    \item プログラムがシンプル
    \end{itemize}
  \item 陰解法: 右辺にも未知変数が含まれる
    \begin{itemize}
    \item 例: 逆Euler法
      \begin{align*}
        y(t) &= y(t+h-h) = y(t+h) - h f(t+h,y(t+h)) + O(h^2) \\
        y_{n+1} &= y_n + h f(t+h,{\color{red}y_{n+1}})
      \end{align*}
    \item 数値的により安定な場合が多い
    \item Newton法などを使って、非線形方程式を解く必要がある
    \end{itemize}
  \end{itemize}
\end{frame}


\end{document}
