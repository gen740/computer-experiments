\begin{frame}[t]{講義日程(予定)}
  \begin{itemize}
    % \setlength{\itemsep}{1em}
  \item 全8回 (水曜2限 10:25-{\color{red}11:55})
    \begin{itemize}
    \item 4月7日 第1回: 環境整備・数値誤差
    \item 4月14日 もくもく会
    \item 4月21日 第2回: ニュートン法・二分法・常微分方程式
    \item 4月28日 第3回: 固有値問題・シンプレクティック積分法
    \item 5月12日 もくもく会
    \item 5月19日 第4回: 行列演算・複素数・ライブラリ [レポートNo.1締切]
    \item 5月26日 第5回: 連立一次方程式・直接解法・反復解法
    \item 6月2日 もくもく会
    \item 6月9日 第6回: 行列の対角化 [レポートNo.2締切]
    \item 6月16日 第7回: 疎行列に対する反復法・変分法
    \item 6月23日 第8回: 特異値分解・最小二乗法
    \item 7月7日 [レポートNo.3締切]
     \end{itemize}
  \end{itemize}
\end{frame}

\begin{frame}[t]{もくもく会}
  \begin{itemize}
    % \setlength{\itemsep}{1em}
  \item 出席は自由
  \item 90分間、各自テーマを決めてもくもくと作業する
    \begin{itemize}
    \item Zoomに接続しっぱなしにする (できればビデオONで)
    \item 最初にZoomチャットでその時間に自分がやることを宣言
    \end{itemize}
  \item 質問自由
    \begin{itemize}
    \item ZoomでマイクをONにして / Zoom チャット / Slack
    \end{itemize}
  \item 作業内容の例
    \begin{itemize}
    \item C言語をマスターする(計算機ハンドブックの例を端から試す)
    \item レポート課題のどれかに取り組む
    \item 自分のプログラムを10倍速くする、等
    \end{itemize}
  \end{itemize}
\end{frame}
