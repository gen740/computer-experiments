\section{スーパーコンピューターと計算物理}

\begin{frame}[t,fragile]{計算機の進化}
  \begin{itemize}
    \setlength{\itemsep}{1em}
  \item 計算機の性能は指数関数的に伸び続けている
    \begin{itemize}
    \item 1年で1.9倍 ⇒ 4年で10倍 ⇒ 過去70年間で100兆倍
    \item 世界初のスパコンCray-1 (1976年)の演算性能 約160MFlops
    \item iPhone6 (2014年)の演算性能 約900MFlops
    \item 2020年代初頭には、1EFlops (エクサフロップス)へ
    \end{itemize}
  \item 現代のスーパーコンピュータは全て
    \begin{itemize}
      \item 並列コンピュータ (CPU数 1,000〜100,000)
      \item マルチコア or メニーコア (CPUあたりのコア数 8 〜 1,000)
      \item 多層にわたる階層構造
      \item 演算に比べて、データを移動するコストの方が高い
    \end{itemize}
  \end{itemize}
\end{frame}
