\section{SSH (Secure Shell)}

\begin{frame}[t,fragile]{SSHによるリモートログイン}
  \begin{itemize}
    \setlength{\itemsep}{1em}
  \item UNIX (Macも含む)では、SSH (Secure Shell)を使うことで、別のコンピュータに遠隔ログインして作業することができる

    例)
    
    {\tt \$ \underline{ssh -X remote.phys.s.u-tokyo.ac.jp -l {\it username}}}
  \item 二種類の認証方式
    \begin{itemize}
    \item パスワード認証: ハンドブック2.2節ではこちらを説明
    \item 公開鍵認証方式: よりセキュリティーの高い方法

      近年はこちらが主流 (photon や ECCS SSHサーバも公開鍵認証)
    \end{itemize}
  \end{itemize}
\end{frame}

\begin{frame}[t,fragile]{SSHの公開鍵認証}
  \begin{itemize}
    \setlength{\itemsep}{1em}
  \item あらかじめクライアント(接続元)側で、「秘密鍵」と「公開鍵」のペアを生成し、「公開鍵」をサーバ(接続先)に置いておく
    \begin{itemize}
    \item 生成には {\tt ssh-keygen} コマンドを使う(準備練習EX0-2)
    \item クライアント側に「秘密鍵」、サーバ側に「公開鍵」の両者が揃ってはじめて、クライアントからサーバにリモートログインできる
    \item たとえ「公開鍵」が盗まれてしまっても、それだけではリモートログインできないので安心
    \item 「秘密鍵」は絶対に人に見られてはならない
    \end{itemize}
  \item 鍵の場所
    \begin{itemize}
    \item 秘密鍵: {\tt \$HOME/.ssh/id\_rsa} に生成される
    \item 公開鍵: {\tt \$HOME/.ssh/id\_rsa.pub} に生成される $\Rightarrow$
      サーバの {\tt \$HOME/.ssh/authorized\_keys} にコピーする
    \end{itemize}
  \end{itemize}
\end{frame}
