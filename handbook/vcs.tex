\section{バージョン管理システムとは?}

計算機を使って何らかのシミュレーション(実験)を行う場合、実験の場合と同様に、どのような計算を行ったか、詳細な記録を残さなければならない。行った計算、特に結果を論文として発表した計算については、結果を再現するのに最低限必要な情報を正確に残しておくことが重要である。また、作業効率という観点からも、履歴をきちんと管理し、系統的にバックアップを残しておくことのメリットは大きい。計算機上においたファイルの場合、手書きのノートと異なり簡単に上書きできてしまう。さらに通常(最後に変更された日時以外)何の痕跡も残らない。その意味では、実際のノート以上に履歴管理に注意する必要があるといえる。

「ファイル管理」の方法として、ファイル名・ディレクトリ名による管理が広く行われている。例えば、名前に最終変更日付や、変更を行った人の名前、適当なバージョン番号などを付ける等々。あるいは、手書きのログファイルによる記録の場合もあるだろう。これらの方法の問題は、人間は記録を付け忘れる、あるいは記録を間違うことが多いという点である。あるいは、結果を再現するには記録が不完全な場合も多い。人により命名規則がばらばらであったり、コンピュータ間でコピーを繰り返すと、どれを修正したか、どれが新しいか分からなくなるといったことも頻発する。さらに面倒なのは、同じバージョンを元に、複数の人が独立に修正を行ってしまった場合である。いったん分岐してしまったバージョンを人間が手でマージするのは、ファイルを一から作成する以上の手間となってしまう場合さえある。

バージョン管理システム(Version Control System)とは、ファイルの変更履歴をリポジトリと呼ばれるデータベースで一括管理するシステムである。バージョン管理システムでは、更新(「チェックイン」と呼ばれる)毎に一意なバージョン番号(リビジョン)を付与し、
全ての修正履歴(差分)をデータベースに保存する。これにより、任意のリビジョン間の比較も簡単に行うことができる。バージョン管理システムは、もともとはCなどのプログラムのソースコードを管理するために開発されたシステムであるが、例えば \LaTeX のソースコードなど、それ以外の種類のファイル管理にも同様に使うことができる。また、ファイルの更新や参照はネットワーク経由で行うことが可能である。これにより、閉じた環境だけでなく、チームや分散環境においても一貫したバージョン管理が可能となる。複数箇所から同時に更新した場合には更新箇所の衝突(コンフリクト)が起こる場合も多いが、そのような衝突を検出し、矛盾なく解決するための仕組みもある。また、ブランチ・タグといったソースコードの公開やメンテナンスに不可欠な機能も備えている。バージョン管理システムは、一言でいうと、超優秀な(かつ超まじめな)秘書のようなものであり、一人で使っても複数人で使っても非常に便利なツールである。

\section{diff と patch}

バージョン管理システムについての説明を始める前に、UNIXで差分を管理するためのツール(コマンド)について紹介しておこう。UNIXでは、プログラムのソースコードや、\LaTeX のソースコードなど、人間が直接作成・修正するファイルは、バイナリ形式ではなくテキスト形式のファイルであることが多い。以下で見るように、{\tt diff}や{\tt patch}といったUNIXのツールは、その動作は非常にシンプルであるが、テキスト形式のファイルの差分を管理するには、非常にパワフルなツールである。

\begin{itemize}
\item {\tt diff}: 2つのテキストファイルの差分を出力するコマンド
  \begin{itemize}
  \item ファイル全体を保存するよりコンパクト
  \item 変更点を確認しやすい
    
    {\tt \$ \underline{diff -u file1.txt file2.txt > file.diff}}
  \end{itemize}
\item {\tt patch}: {\tt diff}コマンドが生成した差分をファイルに適用するユーティリティー
  \begin{itemize}
  \item もとのファイルと差分から変更後のファイルを生成できる
    
    {\tt \$ \underline{patch < file.diff}}
  \end{itemize}
\end{itemize}
単一ファイルの例を示す。まず、ファイルのオリジナルのコピーを取り、エディタを使って内容を修正してみる。
\begin{commandline2}
  \prompt \underline{cp /home/public/ce2015/ex2/prologue.txt prologue.txt} \\
  \prompt \underline{cp prologue.txt prologue-orig.txt} \\
  \prompt \underline{emacs prologue.txt}
\end{commandline2} \noindent
次に{\tt diff}コマンドを用いて、差分を{\tt prologue.diff}に出力する。
\begin{commandline2}
  \prompt \underline{diff -u prologue-orig.txt prologue.txt > prologue.diff}
\end{commandline2} \noindent
{\tt less}コマンドで中身を見てみると、例えば以下のようになっているはずである。
\begin{reidai}
\begin{verbatim}
$ less prologue.diff
--- prologue-orig.txt	2015-04-01 14:57:47.000000000 +0900
+++ prologue.txt	2015-04-01 14:58:57.000000000 +0900
@@ -5,6 +5,6 @@
 misadventured piteous overthrows Do with their death bury their
 parents' strife.  The fearful passage of their death-mark'd love, And
 the continuance of their parents' rage, Which, but their children's
-end, nought could remove, Is now the two hours' traffic of our stage;
+end, zero could remove, Is now the two hours' traffic of our stage;
 The which if you with patient ears attend, What here shall miss, our
 toil shall strive to mend.
\end{verbatim}
\end{reidai} \noindent
{\tt diff}コマンドでは、二つのファイルの差分が出力される。行頭の「{\tt -}」は元ファイルからの削除を、「{\tt +}」は追加を、つまり、「{\tt -}」の行が「{\tt +}」の行で置き換えられてたことを表している。

次に{\tt patch}コマンドを使ってみよう。
\begin{commandline2}
\prompt cp /home/public/ce2015/ex2/prologue.txt prologue.txt \\
\prompt patch < prologue.diff \\
\prompt less prologue.txt
\end{commandline2} \noindent
{\tt patch}コマンドを使うことにより、先ほど作成した差分({\tt prologue.diff})をオリジナルのファイルに適用し、修正点を反映することができる。

単一ファイルだけではなく、以下のようにディレクトリ内のファイルをまとめて差分をとることも可能である。
\begin{commandline2}
\prompt cp -r /home/public/ce2015/ex2 shake \\
\prompt cp -r shake shake.orig \\
\prompt emacs shake/verona.txt \\
\prompt emacs shake/prologue.txt \\
\prompt diff -urN shake.orig shake > shake.diff \\
\prompt less shake.diff
\end{commandline2} \noindent
この例では、{\tt shake.orig}と{\tt shake}に含まれている全てのファイルの差分が{\tt shake.diff}に保存される。{\tt shake.diff}を適用するには{\tt -p0}オプションを指定する。
\begin{commandline2}
\prompt rm -rf shake \\
\prompt cp -r /home/public/ce2015/ex2 shake \\
\prompt patch -p0 < shake.diff
\end{commandline2}

このように{\tt diff}と{\tt patch}により、テキストファイルやそれらを含むディレクトリの差分をかなり容易に扱うことが可能になるが、依然として、何らかの方法で履歴の記録は別途管理しておかなければならないことに変わりはない。

\section{主なバージョン管理システム}

\begin{itemize}
\item BitKeeper - かつて Linux のカーネルのソース管理に使われていた。
\item CVS (Concurrent Versions System) - ネットワークでの利用を考慮とした初めてのバージョン管理システム。最近はあまり使われない。
\item Git - Linux のカーネルの開発に使われている。分散型リポジトリ。最近利用が急速に伸びてきている。
\item Mercurial - Git のライバル。分散型リポジトリ。
\item SCCS (Source Code Control System) - 70年代にベル研で開発された世界初のバージョン管理システム。現在では使われない。
\item Subversion - CVSの改良版として開発された。現在最もポピュラーなバージョン管理システムの一つ。Mac OS X や多くの Linux には最初からインストールされている。
\end{itemize}
本ハンドブックでは、Subversionについて解説する。Subversionに関するより詳細な資料は、オンラインで参照できる。
\begin{itemize}
\item CVS/Subversionを使ったバージョン管理(前編:バージョン管理の基礎)
  \begin{itemize}
  \item \url{http://sourceforge.jp/magazine/08/09/09/1038233}
  \end{itemize}
\item CVS/Subversionを使ったバージョン管理(後編:SVNを使ったバージョン管理)
  \begin{itemize}
  \item \url{http://sourceforge.jp/magazine/08/09/24/113215}
  \end{itemize}
\item Subversion によるバージョン管理
  \begin{itemize}
  \item \url{http://svnbook.red-bean.com/index.ja.html}
  \end{itemize}
\item 「Subversion によるバージョン管理」の読み方
  \begin{itemize}
  \item \url{http://exa.phys.s.u-tokyo.ac.jp/ja/members/wistaria/log/subversion-intro}
  \end{itemize}
\end{itemize}
Subversionではなく、Gitを使いたい場合には、以下の資料を参考にすると良い。
\begin{itemize}
\item {CMSIハンズオン - バージョン管理システム}
  \begin{itemize}
  \item {\footnotesize \url{http://www.cms-initiative.jp/ja/research-support/develop-support/how-to-publish/develop-apps/dt0l33}}
  \end{itemize}
\end{itemize}

\section{Subversionリポジトリ}

Subversionや他のバージョン管理システムでは、「リポジトリ」とよばれる「データベース」にソースコードの全ての履歴を保存する。このリポジトリからソースコードの最新版を引き出したり(チェックアウト)、リポジトリに修正版を登録(チェックイン)したりすることになる。リポジトリの実体は、ハードディスク上のディレクトリ内に保存されている一連のファイルであるが、それらのファイルをユーザが直接触る機会はない。(むしろ直接触ってはいけない。) リポジトリへのアクセスは専用のコマンド(Subversionの場合、{\tt svn}コマンド)を使って行う。

前述の通り、リポジトリにはネットワークを通じてアクセスすることが可能である。Subversionでは、様々なアクセス方法が提供されているが、SSH (Secure Shell, \ref{sec:ssh}節)を利用するのが便利である。リポジトリは、{\tt svn+ssh://ユーザ名@ホスト名/リポジトリ名}のような書式で指定する。例:
\begin{quote}
  {\tt svn+ssh://ce05151598@cmp.phys.s.u-tokyo.ac.jp/home/ce05151598/svnroot}
\end{quote}

このような書式は、URI (Uniform Resource Indicator)と呼ばれ、(ネット上の)場所や名前を一意に表す方法として、広く使われている。例えば、webのアクセスで用いられる
\begin{quote}
  {{https://wistaria@itc-lms.ecc.u-tokyo.ac.jp/lms/course/view.php?id=74564}}
\end{quote}
のような形のURL (Uniform Resource Locator)もURIの一種である。ここで、{\tt https:}の部分は「スキーム(scheme)」と呼ばれ、プロトコル・アクセス方法を指定する。その後ろに、ユーザ情報、ホスト名・サーバ名が続くが、この部分は「オーソリティ(authority)」と呼ばれる。ホスト名の後には、「パス」({\tt /lms/course/view.php}や「クエリ(query)」({\tt ?iid=74564})が続く。クエリは、サーバへの指示や命令を表す。

Subversionのリポジトリ名を表すURIも同様に、スキーム({\tt svn+ssh:})、ユーザ情報、ホスト名・サーバ名、パスからなる。次節で見るように、パスは、サーバー上でリポジトリが置かれている実ディレクトリ名と、リポジトリ内の論理的なディレクトリ名が続けて書かれることに注意せよ。

\section{Subversion実習}

以下、実習ワークステーション名を{\tt cmp.phys.s.u-tokyo.ac.jp}、実習ワークステーションでのユーザ名を{\tt ce00}と仮定して進める。実習では自分のユーザ名({\tt ce05151598}等)に適宜読み替えること。

\subsection{リポジトリの作成}

Subversionの管理用コマンド{\tt svnadmin}を使い、実習用ワークステーション上の{\tt \$HOME/svnroot}にリポジトリを作成する。この作業はリポジトリ作成時に一回だけ行えば良い。作業は、実習用ワークステーションにSSHログインして行う。
\begin{commandline2}
\prompt \underline{ssh cmp.phys.s.u-tokyo.ac.jp -l ce00} \\
\prompt \underline{svnadmin create \$HOME/svnroot} \\
\prompt \underline{exit}
\end{commandline2} \noindent

以下では、iMacに戻り作業を続ける。

\subsection{リポジトリ内のディレクトリ作成}

リポジトリ内に{\tt prog}という名前のディレクトリを作成する。Subversionのコマンド名({\tt svn})の後に、ディレクトリを作成するサブコマンド({\tt mkdir})を指定する。
\begin{commandline2}
{\small \prompt \underline{svn mkdir -m 'test folder' svn+ssh://ce00@cmp.phys.s.u-tokyo.ac.jp/home/ce00/svnroot/prog}} \\
ce00@cmp.phys.s.u-tokyo.ac.jp's password: \\
Committed revision 1.
\end{commandline2} \noindent
コマンドを実行すると、実習用ワークステーションのパスワードを聞かれるので入力する\footnote{{\tt svn co}や{\tt svn update}など、リポジトリとの通信をともなうコマンドの実行時には毎回パスワードを聞かれるが、以下では省略する。}。{\tt -m}オプションの後の文字列はコミットログと呼ばれ、リポジトリにログとして記録される。なるべく分かりやすいログを残すのがよい。リポジトリ上に正しくディレクトリが作成されたかどうか確認してみよう。
\begin{commandline2}
\prompt \underline{svn ls svn+ssh://ce00@cmp.phys.s.u-tokyo.ac.jp/home/ce00/svnroot} \\
prog/
\end{commandline2} \noindent

\subsection{リポジトリからのチェックアウト}

{\tt prog}ディレクトリをリポジトリからチェックアウトし、作業コピーを作成する。
\begin{commandline2}
\prompt \underline{cd \$HOME} \\
\prompt \underline{svn co svn+ssh://ce00@cmp.phys.s.u-tokyo.ac.jp/home/ce00/svnroot/prog} \\
Checked out revision 1.
\end{commandline2} \noindent
今、作業をしているディレクトリ内に空のディレクトリ{\tt prog}が作成されていることを確認する\footnote{正確には、{\tt prog}ディレクトリの中には{\tt .svn}という名前のフォルダ(名前が{\tt .}「ピリオド」で始まっているので、{\tt -a}オプションを付けない限り、{\tt ls}コマンドでは表示されない)が作成されている。このディレクトリには、Subversionの管理用データが収まっているので、決して、中身を消したり変更したりしてなならない。}。
\begin{commandline2}
\prompt \underline{ls -l}
\vspace*{-.7em} 
\begin{verbatim}
...
drwxr-xr-x  3 s001500 student      102 Apr 16 03:40 prog
...
\end{verbatim}
\end{commandline2} \noindent

\subsection{ローカルファイルの作成とリポジトリへのチェックイン}

エディタを使って、簡単なソースコード({\tt hello.c})を作成する。
\begin{commandline2}
\prompt \underline{cd \$HOME/prog} \\
\prompt \underline{emacs hello.c}
\end{commandline2} \noindent
作成後、ファイルをSubverionの管理下に置く(Subversionで管理することをSubversionに伝える)。
\begin{commandline2}
\prompt \underline{svn add hello.c} \\
A \ \ \ hello.c
\end{commandline2} \noindent
{\tt svn stat}コマンドで状態を確認すると以下のように表示されるはずである。
\begin{commandline2}
\prompt \underline{svn stat} \\
A \ \ \ hello.c
\end{commandline2} \noindent
行頭の「{\tt A}」の記号は、ファイル{\tt hello.c}が追加されたが、まだリポジトリには反映されていないことを示す\footnote{同じディレクトリに他のファイルやフォルダが存在するときには、ファイル名の前に「{\tt ?}」が表示される。これは、そのファイルやフォルダがSubversionの管理下にないことを示す。}。

それでは、今作成した{\tt hello.c}をリポジトリにチェックイン(サーバーに送信)しよう。
\begin{commandline2}
\prompt \underline{svn ci -m 'Created my first program'} \\
Adding \ \ \ hello.c \\
Transmitting file data . \\
Committed revision 2.
\end{commandline2} \noindent
無事チェックインされ、リビジョン番号2が付けられた。

\subsection{別の場所での編集}

先程チェックインしたバージョンを別の作業ディレクトリでチェックアウトしてみよう。
\begin{commandline2}
\prompt \underline{cd \$HOME} \\
\prompt \underline{svn co svn+ssh://ce00@cmp.phys.s.u-tokyo.ac.jp/home/ce00/svnroot/prog other} \\
A \ \ \ other/hello.c \\
Checked out revision 2.
\end{commandline2} \noindent
{\tt svn co}コマンドの第2引数に適当な名前(ここでは{\tt other})を指定すると、その名前のフォルダの下にチェックアウトされる。ディレクトリ{\tt other}の中には、{\tt hello.c}が存在しているはずである。このファイルを少し編集してみよう。
\begin{commandline2}
\prompt \underline{cd \$HOME/other} \\
\prompt \underline{emacs hello.c} \\
\prompt \underline{svn stat} \\
M \ \ \ hello.c
\end{commandline2} \noindent
{\tt svn stat}コマンドの出力の行頭の「{\tt M}」の記号は、作業コピーが修正されており、その修正がまだリポジトリには反映されていないことを示す。それではチェックインしよう。
\begin{commandline2}
\prompt \underline{svn ci -m 'Fixed a serious bug'} \\
Sending \ \ \ hello.c \\
Transmitting file data . \\
Committed revision 3.
\end{commandline2} \noindent
リビジョン3としてリポジトリに登録された。

さてここで、もとのフォルダ({\tt \$HOME/prog})に戻ってみると、(当然のことながら) {\tt hello.c}は変更されていない。先ほど別のディレクトリからリポジトリに登録した修正を反映するには、{\tt svn update}コマンドを使う。
\begin{commandline2}
\prompt \underline{cd \$HOME/prog} \\
\prompt \underline{svn update} \\
Updating '.': \\
U \ \ \ hello.c \\
Updated to revision 3.
\end{commandline2} \noindent
このように、複数の場所(複数の人)でファイルを更新していく場合、作業コピーのあるディレクトリで{\tt svn update}⇒編集作業(修正・追加・削除)⇒チェックイン({\tt svn ci})を繰り返すことになる。

\subsection{ファイルの削除・移動}

ファイルを新たにSubversionの管理対象とするには、すでに見たとおり{\tt svn add}コマンドを使う。すでに管理対象となっているファイルを削除するには{\tt svn delte}、移動(名前変更)するには{\tt svn move}、複製するには{\tt svn copy}コマンドを使う。{\tt svn mkdir}コマンドにより作業コピーの下に新たにディレクトリを追加することもできる。
これらのコマンドの後は、{\tt svn ci}でリポジトリに変更点をチェックインする。

\subsection{変更履歴の参照}

ファイルの変更履歴を見るには、{\tt svn log}コマンドを使う。
\begin{commandline2}
\prompt \underline{svn log hello.c}
\end{commandline2} \noindent
また、ファイルの差分を出力することもできる。最後のチェックアウト・アップデートからの作業コピーの修正点を見るには、
\begin{commandline2}
\prompt \underline{svn diff}
\end{commandline2} \noindent
を実行する。特定のリビジョンからの差分を見るには、{\tt -r}オプションでリビジョン番号を指定する。
\begin{commandline2}
\prompt \underline{svn diff -r 1}
\end{commandline2} \noindent
二つのリビジョンの間の差分を見ることもできる。
\begin{commandline2}
\prompt \underline{svn diff -r 1:2}
\end{commandline2} \noindent
{\tt svn annotate}コマンドを使うと、ファイルのそれぞれの行が、どのリビジョンの時点で誰によって修正されたかを見ることができる。
\begin{commandline2}
\prompt \underline{svn annotate hello.c}
\end{commandline2} \noindent
さらに詳しくは、
\begin{itemize}
  \item 「Subversion によるバージョン管理」-- 履歴の確認 \\
    \url{http://jtdan.com/vcs/svn/svn-book/book.html#svn.tour.history}
\end{itemize}
を参照のこと。

\subsection{リポジトリ・作業コピーについての情報の取得}

作業ディレクトリ内で{\tt svn info}コマンドを実行することにより、リポジトリや作業コピーについての情報を取得することができる。
\begin{commandline2}
\prompt \underline{cd \$HOME/prog} \\
\prompt \underline{svn info}
\vspace*{-.7em} 
\begin{verbatim}
Path: .
Working Copy Root Path: ...
URL: svn+ssh://ce00@cmp.phys.s.u-tokyo.ac.jp/home/ce00/svnroot/prog
Relative URL: ^/prog
Repository Root: svn+ssh://ce00@cmp.phys.s.u-tokyo.ac.jp/home/ce00/svnroot/
Repository UUID: d30291ec-23e1-4ba7-bfc0-364b1455df9a
Revision: 7
Node Kind: directory
Schedule: normal
Last Changed Author: ce00
Last Changed Rev: 7
Last Changed Date: 2015-04-01 19:52:28 +0900 (Wed, 1 Apr 2015)
\end{verbatim}
\end{commandline2} \noindent

\subsection{コンフリクト}

チェックインしようとしたファイルがすでに他の人(or 他の場所)により更新されている場合、チェックインはエラーとなる。以下、{\tt \$HOME/prog}と{\tt \$HOME/other}で同時に{\tt hello.c}を編集することで、わざとコンフリクトを発生させてみよう。

まずは、{\tt \$HOME/prog}と{\tt \$HOME/other}の両方のディレクトリで最新版にアップデートしておく。
\begin{commandline2}
\prompt \underline{cd \$HOME/prog} \\
\prompt \underline{svn update} \\
\prompt \underline{cd \$HOME/other} \\
\prompt \underline{svn update}
\end{commandline2} \noindent
次に、{\tt \$HOME/prog/hello.c}を修正し、チェックインする。
\begin{commandline2}
\prompt \underline{cd \$HOME/prog} \\
\prompt \underline{emacs hello.c} \\
\prompt \underline{svn diff}
\vspace*{-.7em} 
\begin{verbatim}
Index: hello.c
===================================================================
--- hello.c    (revision 5)
+++ hello.c    (working copy)
@@ -1,5 +1,5 @@
 #include <stdio.h>
 int main() {
-  printf("> %lf\n", 10.0);
+  printf("> %lf\n", 1.0);
   return 0;
 }
\end{verbatim}
\vspace*{-.7em} 
\prompt \underline{svn ci -m 'Change value'}
\end{commandline2} \noindent
次に、{\tt \$HOME/other/hello.c}の同じ行を異なる値に修正し、チェックインしてみよう。
\begin{commandline2}
\prompt \underline{cd \$HOME/other} \\
\prompt \underline{emacs hello.c} \\
\prompt \underline{svn diff}
\vspace*{-.7em} 
\begin{verbatim}
Index: hello.c
===================================================================
--- hello.c    (revision 5)
+++ hello.c    (working copy)
@@ -1,5 +1,5 @@
 #include <stdio.h>
 int main() {
-  printf("> %lf\n", 10.0);
+  printf("> %lf\n", 2.0);
   return 0;
 }
\end{verbatim}
\vspace*{-.7em} 
\prompt \underline{svn ci -m 'Change value, too'} \\
Sending \ \ \ hello.c \\
Transmitting file data .svn: E160028: Commit failed (details follow): \\
svn: E160028: ファイル '/prog/hello.c' はリポジトリ側と比べて古くなっています
\end{commandline2} \noindent
このようにチェックインは失敗する。この問題は以下のような手順で解決することができる。

まず、{\tt \$HOME/other}で{\tt svn update}を実行する。
\begin{commandline2}
\prompt \underline{cd \$HOME/other} \\
\prompt \underline{svn update} \\
Updating '.': \\
C \ \ \ hello.c
Updated to revision 6. \\
Conflict discovered in file 'hello.c'. \\
Select: (p) postpone, (df) show diff, (e) edit file, (m) merge, \\
\ \ \   (mc) my side of conflict, (tc) their side of conflict, \\
\ \ \   (s) show all options: \underline{p} \\
Summary of conflicts: \\
\ \ Text conflicts: 1
\end{commandline2} \noindent
Subversionがコンフリクト(変更の衝突)を発見し、解決策を訪ねてくるので、ここでは{\tt p} (postpone)と答える\footnote{同じファイルが同時に変更されている場合でも、変更点が互いに離れた行である場合には、Subversionは修正点を自動でマージしてくれる。}。作業ディレクトリには、{\tt hello.c}以外に、{\tt hello.c.mine}、{\tt hello.c.r5}、{\tt hello.c.r6}などのファイルが作成されているはずである。それぞれ、先ほど修正をチェックインしようとした内容、修正の元となったリビジョン(この例ではリビジョン5)、リポジトリに登録されている最新リビジョン(この例ではリビジョン6)のファイルとなっている。また、{\tt hello.c}の中を見てみると、どこがどのようにコンフリクトしているかが書き込まれている。
\begin{commandline2}
\prompt \underline{cat hello.c}
\vspace*{-.7em} 
\begin{verbatim}
#include <stdio.h>
int main() {
<<<<<<< .mine
  printf("> %lf\n", 2.0);
=======
  printf("> %lf\n", 1.0);
>>>>>>> .r6
  return 0;
}
\end{verbatim}
\end{commandline2} \noindent
ここで「\verb+<<<<<<< .mine+」から「\verb+=======+」までが、作業コピーの変更点、「\verb+=======+」から「\verb+>>>>>> .r6+」までが、リポジトリに登録済みの変更点である。{\tt hello.c.mine}、{\tt hello.c.r5}、{\tt hello.c.r6}を参照しながら{\tt hello.c}を納得のいくまで修正する。ここでは、{\tt hello.c.mine}の内容を採用することにする。
\begin{commandline2}
\prompt \underline{emacs hello.c} \\
\prompt \underline{cat hello.c}
\vspace*{-.7em} 
\begin{verbatim}
#include <stdio.h>
int main() {
  printf("> %lf\n", 2.0);
  return 0;
}
\end{verbatim}
\end{commandline2} \noindent
編集が完了したら、{\tt svn resolved}コマンドでSubversionにコンフリクトが解消されたことを知らせ、チェックインする。
\begin{commandline2}
\prompt \underline{svn resolved hello.c} \\
Resolved conflicted state of 'hello.c' \\
\prompt \underline{svn ci -m 'Merge r5 and r6'} \\
Sending \ \ \ hello.c \\
Transmitting file data . \\
Committed revision 7.
\end{commandline2} \noindent
無事、コンフリクトが解消され、リポジトリに反映された。さらに詳しくは、
\begin{itemize}
  \item 「Subversion によるバージョン管理」-- 衝突の解消(他の人の変更点のマージ) \\
    \url{http://jtdan.com/vcs/svn/svn-book/book.html#svn.tour.cycle.resolve}
\end{itemize}
を参照のこと。
  

%% \subsection{Subversionを使う際の注意事項}
%% \begin{itemize}
%% \item 大きなバイナリファイル(pdf, exe, doc, tar.gz, zipなど)はなるべくSubversionで管理しない。
%%   \begin{itemize}
%%   \item バイナリファイルはうまく差分が扱えない(マージできない)ので、Subversionで管理しない方が良い。
%%   \end{itemize}
%% \item チェックアウトしたディレクトリ「以外」でのファイルの編集は危険。
%%   \begin{itemize}
%%   \item チェックアウト ⇒ ファイルをコピー ⇒ コピーしたファイルを変更 ⇒ チェックアウトしたディレクトリで{\tt svn update} ⇒ 変更したファイルをチェックアウトしたディレクトリに戻す ⇒ {\tt svn commit} ⇒ {\tt svn update}の変更点が上書きされ失われてしまう!
%%   \end{itemize}
%% \item svn stat, svn diff はネットワークにつながってなくても使用可。
%% \end{itemize}
