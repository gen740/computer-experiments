\documentclass[11pt]{jarticle}

\usepackage{amsmath}
\usepackage{graphics}
\usepackage{hyperref}

\setlength{\oddsidemargin}{-0.7cm}
\setlength{\topmargin}{-1.5cm}
\setlength{\textwidth}{16.5cm}
\setlength{\textheight}{26cm}
\pagestyle{empty}

\begin{document}

\noindent
{\bf\large 「計算機実験II」実習課題(EX4) 2018-12-25}
\\[-0.5em]

\noindent
\begin{itemize}
\item 講義のページ: \verb+http://exa.phys.s.u-tokyo.ac.jp/ja/lectures/2018w-computer2+

\item サンプルプログラム: 「講義L4サンプルプログラム」{\tt example-2-L4.zip}

\item 準備練習
  
\begin{enumerate}
  \item {\tt golden\_section.c}は、黄金分割法により一次元関数$f(x) = 5x+x^2+70\sin(x)$の極値を求めるプログラムである。コンパイルして実行せよ。コードの中身を読んで確認せよ(参考: 講義資料{\tt lecture-2.4.pdf} p.8)
  \item {\tt nelder\_mead\_2d.c}は、Nelder-Mead法により二次元関数$f(x, y) = −10(x^2 + y^2) + (x^2 + y^2)^2 − 2(x + y)$の極値を求めるプログラムである。コンパイルして実行せよ。コードの中身を読んで確認せよ(参考: 講義資料{\tt lecture-2.4.pdf} p.23)
\end{enumerate}

\item 基本課題
  \begin{enumerate}
  \item {\tt steepest\_descent.c}は、最急降下法により連立一次方程式を解くプログラムである。実行時には引数として、行列$A$と右辺ベクトル$b$が入ったファイル名({\tt input3.dat})を指定する。コンパイルして実行し、LU分解による解法{\tt lu\_decomp.c}と解が一致することを確かめよ
  \item {\tt steepest\_descent.c}を元にして、行列$A$と右辺ベクトル$b$をファイルから読み込み「共役勾配法」により連立一次方程式を解くプログラムを作成せよ。最急降下法と収束回数を比較せよ
  \item 測定データ{\tt measurement-3.dat}を関数$f(x)=ax+e^{−(x−c)^2}$で最小二乗フィッティングしよう。パラメータ$a, c$を、勾配降下法、Nelder-Mead法などを用いて残差を最小化することにより推定せよ
  \item 共役勾配法を用いて、Dirichlet型の境界条件のもとでの二次元Poisson方程式(あるいはLaplace方程式)の解を求めるプログラムを作成せよ。実行時間のメッシュ数依存性をLU分解を用いた場合と比較せよ。(参考: 計算機実験I 講義資料{\tt lecture-1-3.pdf} p.3、実習資料{\tt exercise-1-3.pdf} 基本課題2)

    (ヒント: Poisson方程式を行列形式に書き直すことが難しい場合には、{\tt poisson.h}を参考にしてもよい。行列生成({\tt poisson\_dense.c})、行列ベクトル積({\tt poisson\_sparse.c})、LU分解による求解({\tt poisson\_lu.c})のテストプログラムも用意されている)
  \end{enumerate}  
\item 応用課題
  \begin{enumerate}
  \item シミュレーテッド・アニーリングにより、離散最適化問題(巡回セールスマン問題など)を解くプログラムを作成せよ。温度のスケジューリングを変えることで、
正解を得られる確率がどのように変化するか調べよ
  \item 共役勾配法を用いた連立一次方程式の解法では「前処理」が非常に重要
であることが知られている。「前処理」とは何か? 前処理が必要となる理由は? また、実際の数値計算ではどのような前処理方法が使われているか、調べてみよ
  \end{enumerate}

\item レポート課題No.2

基本課題2,3,4についてレポートを作成し提出せよ。実習3(EX3)のレポート課題(基本課題1,3,5)とあわせて、一つのレポート(PDF)としてITC-LMSで提出すること。提出締め切りは1月11日とする。ソースコードを全て含める必要はないが、プログラム作成時に苦労した点、工夫した点などについて適宜ソースコードを引用して説明すること。レポート対象になっていない基本課題・応用課題についても解いている場合や、特に深い解析・考察を行っている場合は、加点の対象とする
\end{itemize}
\end{document}
