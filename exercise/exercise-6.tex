\documentclass[11pt]{jarticle}

\usepackage{amsmath}
\usepackage{graphics}
\usepackage{hyperref}

\setlength{\oddsidemargin}{-0.7cm}
\setlength{\topmargin}{-1.5cm}
\setlength{\textwidth}{16.5cm}
\setlength{\textheight}{26cm}
\pagestyle{empty}

\begin{document}

\noindent
{\bf\large 「計算機実験」実習課題(EX6)}
\\[-0.5em]

\noindent
\begin{itemize}
\item 講義のページ: \verb+http://exa.phys.s.u-tokyo.ac.jp/ja/lectures/2016S-computer+

\item サンプルプログラム: \\ {\small \verb+https://github.com/todo-group/computer-experiments/tree/master/exercise/monte_carlo+} \\ {\small \verb+https://github.com/todo-group/computer-experiments/tree/master/exercise/optimization+}
  
\item 準備練習
  \begin{enumerate}
  \item {\tt random.c}は、Mersenne-Twister乱数発生器({\tt mersenne\_twister.h})により、$(0,1)$の範囲で一様分布する実数乱数を生成するプログラムである。コマンドライン引数により乱数の種(seed)を指定できるようにプログラムを修正せよ。種を変えて何度か乱数を生成し、その時系列を比較してみよ。
  \item {\tt nelder\_mead\_1d.c}は、一次元のNelder-Mead滑降シンプレックス法により一次元関数$f(x)=(5x+x^2)+70\sin(x)$の極値を求めるプログラム、{\tt nelder\_mead\_2d.c}は、二次元関数$f(x,y)=-10(x^2+y^2)+(x^2+y^2)^2-2(x+y)$の極値を求めるプログラムである。コンパイルして実行せよ。初期値を変えてその収束の様子を観察せよ。
  \end{enumerate}

\item 基本課題
  \begin{enumerate}
  \item $X$と$Y$を$(0,1)$で一様分布するそれぞれ独立な(実数)確率変数とする。このとき$X^2$, $1/(X+1)$, $\log X$, $XY$のそれぞれの期待値を(解析的に)求めよ。また、実際に乱数を生成させて期待値を計算し、解析的な結果と比較せよ。
  \item 黄金分割による囲い込み法を用いて、一次元関数$f(x)=(5x+x^2)+70\sin(x)$の極値を求めるプログラムを作成せよ。({\tt golden\_section.c}は初期囲い込みまでを行うプログラムである。これに黄金分割探索を追加せよ)
  \item 測定データ{\tt measurement-3.dat}を関数$f(x)=ax+e^{-(x-c)^2}$でフィッティングしよう。パラメータ$a$, $c$を、二次元Nelder-Mead滑降シンプレックス法を用いて残差を最小化することにより推定せよ。
  \end{enumerate}
  
\item 応用課題
  \begin{enumerate}
  \item マルコフ連鎖モンテカルロ法により、二次元イジング模型のエネルギーと比熱の期待値を計算せよ。システムサイズを変えると、エネルギーや比熱はどのように振る舞うか?
  \item 正規分布にしたがう乱数の生成方法について調べよ。また、平均値$\mu_i$と分散共分散行列$\Sigma_{ij}$をもつ多次元正規分布にしたがう乱数の生成方法を考えよ
  \item 基本課題3のフィッティングを、最急降下法、あるいは共役勾配法を用いて行ってみよ
  \end{enumerate}  
\end{itemize}

\noindent
{\bf\large レポートNo.3}
%\\[-0.5em]
\noindent
\begin{itemize}
\item 実習EX5基本課題1〜3、実習EX6基本課題1〜3についてレポートをまとめ提出せよ。
\item 提出方法: \\
  実習ワークステーション上のSubversionリポジトリに適当な名前のディレクトリを作成し、計算に用いたソースコード({\tt *.c}、{\tt *.h})、計算出力結果、レポートの \LaTeX ソース({\tt *.tex})、図のEPSファイル({\tt *.eps})など、レポートの作成に必要なファイル一式と最終的なレポート({\tt *.pdf})をチェックインせよ。チェックイン後、作業ディレクトリ内で {\tt svn update} コマンド、ついで {\tt svn info} コマンドを実行し、その出力結果を『「計算機実験」レポートNo.3提出票』({\tt report-3.txt})に貼り付け、ITC-LMSにアップロードすること。
\item 提出締切: 7/29(金) 23:59
\end{itemize}

\end{document}
