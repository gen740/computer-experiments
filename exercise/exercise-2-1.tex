\documentclass[11pt]{jarticle}

\usepackage{amsmath}
\usepackage{graphics}
\usepackage{hyperref}

\setlength{\oddsidemargin}{-0.7cm}
\setlength{\topmargin}{-1.5cm}
\setlength{\textwidth}{16.5cm}
\setlength{\textheight}{26cm}
\pagestyle{empty}

\begin{document}

\noindent
{\bf\large 「計算機実験II」実習課題(EX1) 2019-10-11}
\\[-0.5em]

\noindent
\begin{itemize}
\item サンプルプログラム: {\verb+example-2-L1.zip+}

\item 準備練習

  \begin{enumerate}
    \item 計算機実験Iで行った基本的な操作・手順を復習する。復習は不要と思う場合にはそのまま基本課題に進んでよい
      \begin{itemize}
      \item Emacsあるいは他のエディタを用いたファイル編集(ハンドブック2.3節)
      \item Cプログラムの作成・コンパイル・実行(ハンドブック3.1.1節)
      \item LU分解のサンプルプログラム{\tt lu\_decomp.c}のコンパイル・実行
      \end{itemize}
    \item プログラム{\tt double\_well.c}は、ハウスホルダー法を用いた対角化により、無次元化されたシュレディンガー方程式
    \begin{equation*}
      {\cal H} \psi(x) = \Big[ -\frac{d^2}{dx^2} + V(x) \Big] \psi(x) = E \psi(x)
    \end{equation*}
    の固有値と固有ベクトル(エネルギーの低い順に10個)を計算するプログラムである。ポテンシャルは
    \begin{equation*}
      V(x) = \begin{cases}
        \infty & x < 0, \ x > 1 \\
        0 & 0 < x < \frac{1}{2} - w/2, \ \frac{1}{2} + w/2 < x < 1 \\
        v & \frac{1}{2} - w/2 < x < \frac{1}{2} + w/2
      \end{cases}
    \end{equation*}
    の形(二重井戸ポテンシャル)である。
    コマンドライン引数として、刻み数{\tt n}、障壁の高さ$v$ ({\tt v})、障壁の幅$w$ ({\tt width})を指定する。
    障壁の幅や高さを変えた時に固有値や固有ベクトルがどのように変化するか調べ、図示せよ。また、その物理的意味を考察せよ
  \end{enumerate}

\item 基本課題
  \begin{enumerate}
  \item 関数のゼロ点を求める二つの方法、ニュートン法[計算機実験I (L1) p.18, 計算機実験I (EX1) 基本課題2]と二分法[計算機実験II (L1) p.15]を比較しよう
    \begin{itemize}
    \item サンプルプログラム{\tt bisection.c}を、$f(x) = \tanh x+0.2x+0.3=0$を解くように修正せよ
    \item ニュートン法と二分法それぞれについて、反復にしたがって値がどのように真値に近づいていくか図示せよ
    \item 二分法では正しくゼロ点が見つかるが、ニュートン法が失敗するような関数の例を考え、実際に実験してみよ
    \end{itemize}
  \item 準備練習2と同じハミルトニアンを考える
    \begin{itemize}
      \item ハミルトニアン行列とベクトル(波動関数)の積を計算する関数を作成せよ。計算量が行列の次元{\tt dim = n-1}に比例するようなコードとすること
      \item 作成した行列ベクトル積の関数を用いて、べき乗法[計算機実験II (L1) p.24]によりハミルトニアンの基底状態と基底エネルギーを求め、準備練習2の結果と比較せよ。収束の様子や計算時間の刻み数$n$依存性のグラフを作成すること
    \end{itemize}
  \item 準備練習2と同じハミルトニアンを考える
    \begin{itemize}
    \item 変分波動関数として、$v=0$の場合の固有状態$\phi_n(x) = \sqrt{2} \sin(n \pi) \ (n=1,2,3\cdots)$の線形結合$\psi(x) = \sum_n C_n \phi_n(x)$を考える(ポテンシャルが左右対称なので実際には対称な波動関数(奇数の$n$)のみを考えれば十分である)。これらの波動関数に対して、$H_{pq} = \langle \phi_p | {\cal H} | \phi_q \rangle$の表式を解析的に求めよ
    \item 適当な$v$について、固有値問題$HC = EC$をハウスホルダー法を用いて数値的に解き、近似基底状態と近似基底エネルギーを求めよ[計算機実験II (L1) p.29]。基底の数を増やすにつれて、近似基底エネルギーはどのように準備練習2で得られた値に近づいていくか図示せよ
    \end{itemize}
  \end{enumerate}  
\item 応用課題
  \begin{enumerate}
  \item 準備練習2と同じハミルトニアンを考える
    \begin{itemize}
    \item Numerov法[計算機実験II (L1) p.11]を用いて、与えられたエネルギー固有値$E$の下でシュレディンガー方程式を$x=0$から$x=1$まで積分するプログラムを作成せよ。境界条件は$\psi(x_0=0)=0, \psi(x_1=h)=1$とする。
    \item $E$の値を変えると、それに従って解がどのように変化するか図示せよ
    \item シューティング[計算機実験II (L1) p.14]により、固有値と固有ベクトルを一組求めるプログラムを作成し、準備練習2の結果と比較せよ
    \end{itemize}
  \item 基底関数として$v=\infty$の場合の固有状態[$0<x<(w-1)/2$と$(w+1)/2<x<1$に二つの井戸型ポテンシャルがある場合の固有状態のうち対称なもののみを考慮]も加えて数値的変分計算を行え。基底関数が互いに直交しないので、一般化固有値問題[計算機実験II (L1) p.31]となることに注意
  \item 計算機実験II (L1) p.33で導入した手法(シュレディンガー方程式の一般解について、境界条件に関する非線形連立方程式が非自明な解を持つ条件を考える)を使って固有値を求めるプログラムを作成せよ[LU分解による行列式の計算({\tt determinant.c})と二分法({\tt bisection.c})を組み合わせる]
  \item べき乗法ではなくLanczos法を用いて、基底状態と基底エネルギーを計算せよ。また、べき乗法とLanczos法について、収束までの反復回数を比較せよ
  \end{enumerate}

\item レポート課題

基本課題1,2,3の中から2問選びレポートを作成せよ(それ以外の基本課題、応用課題はオプションであるがこれらも解いて提出した場合には加点の対象)。実習2 (EX2)のレポート課題とあわせて、一つのレポートとして提出すること。提出締め切りは11月15日(金)とする。ITC-LMSから提出すること

\end{itemize}

\end{document}
