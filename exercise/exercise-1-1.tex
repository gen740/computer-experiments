\documentclass[11pt]{jarticle}

\usepackage{amsmath}
\usepackage{graphics}
\usepackage{hyperref}

\setlength{\oddsidemargin}{-0.7cm}
\setlength{\topmargin}{-1.5cm}
\setlength{\textwidth}{16.5cm}
\setlength{\textheight}{26cm}
\pagestyle{empty}

\begin{document}

\noindent
{\bf\large 「計算機実験I」実習課題(EX1)}
\\[-0.5em]

\noindent
\begin{itemize}
\item 講義のページ: \verb+http://exa.phys.s.u-tokyo.ac.jp/ja/lectures/2018s-computer1+

\item 実習用ワークステーション「photon」について
  \begin{itemize}
  \item ホスト名: {\tt cmp.phys.s.u-tokyo.ac.jp}
  \item ユーザ名: {\tt ce}+学籍番号(ハイフン無し・8桁): \ 例: {\tt ce05171583}
  \item 前回作成した公開鍵を登録済
  \item 学外から直接SSHログインすることは不可。ただし、一旦、ECCS SSHサーバを経由すれば可 (\verb+http://www.ecc.u-tokyo.ac.jp/system/outside.html+)
  \end{itemize}
  
\item 準備練習
  \begin{enumerate}
  \item 実習用ワークステーション(photon)へ{\tt ssh}を使ってログイン(ハンドブック2.2節)。
  \item photon 上の{\tt /home/public/ce2018/ex1/hello.c}を{\tt scp}をつかってiMacへコピー(ハンドブック2.2節)。Cコンパイラでコンパイルし、実行(ハンドブック3.1.1節)。
  \item ハンドブック3.2.1〜3.2.2節(制御文), 3.6.2〜3.6.3節(関数)の例題を試す。
  \item 上記、準備練習3を、実習用ワークステーション(photon)でも行ってみよ。
  \end{enumerate}

\item 基本課題
  \begin{enumerate}
  \item $f(x)=\sin x$について、$x=0.3\pi$における$f'(x)$の値を数値微分により計算するプログラムを作成せよ。数値微分の刻みを$h=1,1/2,1/4,1/8,\cdots$と減少させていった時、誤差がどのように振る舞うか図示せよ。最低次近似(2点差分)とより高次の近似(3点差分)における誤差の振る舞いの違いを調べよ(前回講義スライドp.15)。
  \item $\sqrt[3]{x}$を求めるNewtonの反復式を書け。これを用いて、$\sqrt[3]{10}$を求めるプログラムを作成せよ。反復にしたがって、値がどのように真値に近づいていくか図示せよ。
  \end{enumerate}
  
\item 応用課題
  \begin{enumerate}
  \item 領域$[0,1]$から$[0,1]$への写像$x_{n+1} = \min(2x_n,2-2x_n)$を繰り返す時、解軌道$\{x_0,x_1,x_2,x_3,\cdots\}$の振る舞いが、初期値$x_0$にどのように依存するか予想せよ。また、プログラムを作成し、いくつかの初期値についてその解軌道を図示せよ。予想と全く異なる振る舞いが観測されるのはなぜか? その理由について考察せよ。
  \item 代数方程式の解をすべてもとめる方法について調べよ。Durand-Kerner-Aberth法を用いて、代数方程式の全ての解を求めるプログラムを作成せよ。方程式の次数を増やすにつれ、収束までにかかる時間がどのように増えるか調べよ。
  \item 教育用計算機システムECCSのiMacでは、数式処理ソフトウェアMathematicaが利用できる。Mathematicaを用いて、$\sqrt[3]{10}$を20桁の精度で求めてみよ。
  \end{enumerate}  

\item 追加課題(自宅で)
  \begin{enumerate}
  \item 学外からECCS SSHサーバにリモートログインし、さらにそこから photon にリモートログインしてみよ。事前に ECCS の iMac、あるいは「SSH公開鍵アップロード」により公開鍵を配置しておくこと。(\verb+http://www.ecc.u-tokyo.ac.jp/system/outside.html+)
  \end{enumerate}

\end{itemize}
\end{document}
