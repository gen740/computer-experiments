\documentclass[11pt]{jarticle}

\usepackage{amsmath}
\usepackage{graphics}
\usepackage{hyperref}

\setlength{\oddsidemargin}{-0.7cm}
\setlength{\topmargin}{-1.5cm}
\setlength{\textwidth}{16.5cm}
\setlength{\textheight}{26cm}
\pagestyle{empty}

\begin{document}

\noindent
{\bf\large 「計算機実験I」実習課題(EX1)}
\\[-0.5em]

\noindent
\begin{itemize}
\item 講義のページ: \verb+http://exa.phys.s.u-tokyo.ac.jp/ja/lectures/2018s-computer1+

\item 実習用ワークステーション「photon」へのSSHログイン
  \begin{itemize}
  \item ホスト名: {\tt cmp.phys.s.u-tokyo.ac.jp}
  \item ユーザ名: {\tt ce}+学籍番号(ハイフン無し・8桁): \ 例: {\tt ce05181583}
  \item 前回作成した公開鍵を登録済
  \item 学外から直接SSHログインすることは不可。ただし、一旦、ECCS SSHサーバを経由すれば可。(参考: \verb+http://www.ecc.u-tokyo.ac.jp/system/outside.html+)
\end{itemize}
  
\item 準備練習
  \begin{enumerate}
  \item 実習用ワークステーション(photon)へ{\tt ssh}を使ってログイン(ハンドブック2.2節)
  \item photon 上の{\tt /home/public/ce2018/ex1/hello.c}を{\tt scp}をつかってiMacへコピー(ハンドブック2.2節)。Cコンパイラでコンパイルし、実行(ハンドブック3.1.1節)
  \item ハンドブック3.2.1〜3.2.2節(制御文), 3.6.2〜3.6.3節(関数)の例題を試す
  \item 上記、準備練習3で作成したファイルを、photonに{\tt scp}して、コンパイル・実行
  \end{enumerate}

\item 基本課題
  \begin{enumerate}
  \item $f(x)=\sin x$について、$x=0.3\pi$における$f'(x)$の値を数値微分により計算するプログラムを作成せよ。数値微分の刻みを$h=1,1/2,1/4,1/8,\cdots$と減少させていった時、誤差がどのように振る舞うか図示せよ。最低次近似(2点差分)と3点差分(さらにはより高次の近似)における誤差の振る舞いの違いを調べよ(前回講義スライドp.15)。
  \item $f(x)=\tanh x + 0.2 x + 0.3 = 0$の解をNewton法により求めよ。反復にしたがって、値がどのように真値($x=-0.2544612950513368\cdots$)に近づいていくか図示せよ。また、初期値による収束の違いを調べ、その理由について考察せよ
  \end{enumerate}
  
\item 応用課題
  \begin{enumerate}
  \item C言語における倍精度実数({\tt double})、単精度実数({\tt float})の有効桁数、最大値、(正の)最小値を確認するプログラムを作成せよ
  \item 代数方程式の解をすべてもとめる方法について調べよ。Durand-Kerner-Aberth法を用いて、代数方程式の全ての解を求めるプログラムを作成せよ。方程式の次数を増やすにつれ、収束までにかかる時間がどのように増えるか調べよ
  \item 教育用計算機システムECCSのiMacでは、数式処理ソフトウェアMathematicaが利用できる。Mathematicaを用いて、基本課題2の解を20桁の精度で求めてみよ。
  \end{enumerate}  

\item 追加課題(自宅で)
  \begin{enumerate}
  \item 学外からECCS SSHサーバにリモートログインし、さらにそこから photon にリモートログインしてみよ。なお、事前に「SSH公開鍵アップロード」による公開鍵の配置が必要である。(\verb+http://www.ecc.u-tokyo.ac.jp/system/outside.html+)
  \end{enumerate}

\end{itemize}
\end{document}
