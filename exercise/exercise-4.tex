\documentclass[11pt]{jarticle}

\usepackage{amsmath}
\usepackage{graphics}
\usepackage{hyperref}

\setlength{\oddsidemargin}{-0.7cm}
\setlength{\topmargin}{-1.5cm}
\setlength{\textwidth}{16.5cm}
\setlength{\textheight}{26cm}
\pagestyle{empty}

\begin{document}

\noindent
{\bf\large 「計算機実験」実習課題(EX4)}
\\[-0.5em]

\noindent
\begin{itemize}
\item 講義のページ: \verb+http://exa.phys.s.u-tokyo.ac.jp/ja/lectures/2016S-computer+

\item サンプルプログラム: \\ {\small \verb+https://github.com/todo-group/computer-experiments/tree/master/exercise/eigenvalue_problem+}
  
\item 準備練習
  \begin{enumerate}
  \item ハウスホルダー法による対角化のサンプルプログラム({\tt diag.c})をコンパイル・実行せよ
  \item {\tt diag.c}で得られた固有ベクトルが互いに正規直交していることを確認するコードを作成し実行せよ。(それぞれの固有ベクトルを行とする行列とその転置行列をかけて、単位行列になることを確認すればよい)
  \end{enumerate}

\item 基本課題
  \begin{enumerate}
  \item 成分が$v_{ij}=\min(i,j)$ ($1 \le i \le n$, $1 \le j \le n$)の$n \times n$対称行列の固有値は、$1/[2 (1 - \cos (\pi (2 k - 1) / (2 n + 1)))]$ ($k=1,\cdots,n$)で与えられることが知られている。べき乗法を用いて、最大固有値を計算するプログラムを作成し、その結果を理論式と比較せよ。
  \item {\tt double\_well.c}は、対角化により、有限の障壁で隔てられた二重井戸ポテンシャル
    \begin{equation*}
      V(x) = \begin{cases}
        \infty & x < 0, \ x > 1 \\
        0 & 0 < x < \frac{1}{2} - w/2, \ \frac{1}{2} + w/2 < x < 1 \\
        v & \frac{1}{2} - w/2 < x < \frac{1}{2} + w/2
      \end{cases}
    \end{equation*}
    の固有値と固有ベクトルを計算するプログラムである。コマンドライン引数として、刻み数{\tt n}、障壁の高さ{\tt v}、障壁の幅{\tt width}を指定する。障壁の幅や高さを変えた時に固有値や固有ベクトルがどのように変化するか調べ、図示せよ。また、その物理的意味を考察せよ。(ヒント: 障壁が無限に高い極限からの摂動を考えてみよ)
  \item {\tt double\_well.c}では全ての変数が無次元化されている。粒子の質量、井戸の幅、障壁の幅、高さとして、(次元をもつ)物理的に妥当な値を仮定せよ。それらを無次元化すると、{\tt v}、{\tt width}の値はいくらになるか? また、それらの値から{\tt double\_well.c}により計算された固有値を、次元をもつ実際の値に換算してみよ。
  \end{enumerate}
  
\item 応用課題
  \begin{enumerate}
  \item Lanczos法により固有値を計算するプログラムを作成せよ。Ritz値が、繰り返しに従ってどのように振る舞うか図示してみよ。
  \item 基本課題2の行列は疎行列(三重対角行列)である。その性質を利用して、行列ベクトル積を効率的に計算するコードを作成し、Lanczos法に組み込み、その速度を計測せよ。
  \end{enumerate}  
\end{itemize}

\noindent
{\bf\large レポートNo.2}
%\\[-0.5em]
\noindent
\begin{itemize}
\item 実習EX3基本課題1〜3、実習EX4基本課題1〜3についてレポートをまとめ提出せよ。
\item 提出方法: \\
  実習ワークステーション上のSubversionリポジトリに適当な名前のディレクトリを作成し、計算に用いたソースコード({\tt *.c}、{\tt *.h})、計算出力結果、レポートの \LaTeX ソース({\tt *.tex})、図のEPSファイル({\tt *.eps})など、レポートの作成に必要なファイル一式と最終的なレポート({\tt *.pdf})をチェックインせよ。チェックイン後、作業ディレクトリ内で {\tt svn update} コマンド、ついで {\tt svn info} コマンドを実行し、その出力結果を『「計算機実験」レポートNo.2提出票』({\tt report-2.txt})に貼り付け、ITC-LMSにアップロードすること。
\item 提出締切: 実習(EX4)日の二週間後の23:59とする。
\end{itemize}

\end{document}
