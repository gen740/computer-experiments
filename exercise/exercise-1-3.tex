\documentclass[11pt]{jarticle}

\usepackage{amsmath}
\usepackage{graphics}
\usepackage{hyperref}

\setlength{\oddsidemargin}{-0.7cm}
\setlength{\topmargin}{-1.5cm}
\setlength{\textwidth}{16.5cm}
\setlength{\textheight}{26cm}
\pagestyle{empty}

\begin{document}

\noindent
{\bf\large 「計算機実験I」実習課題(EX3)}
\\[-0.5em]

\noindent
\begin{itemize}
\item サンプルプログラム: example-1-L3.zip
  
\item 準備練習
  \begin{enumerate}
  \item ベクトルや行列を扱うためのユーティリティー関数が({\tt cmatrix.h})に用意されている。サンプルプログラム{\tt matrix\_example.c}や行列・行列積を計算するプログラム{\tt multiply.c}の中身を見て、その使い方を確認せよ
  \item LU分解のサンプルプログラム({\tt lu\_decomp.c})をコンパイル・実行せよ。コンパイル時にLAPACKとBLASをリンク({\tt -llapack -lblas})する必要がある(ハンドブック3.1.6節)
    \begin{quote} \tt
      \$ \underline{cc lu\_decomp.c -o lu\_decomp -llapack} \\
      \$ \underline{./lu\_decomp input1.dat}
    \end{quote}
  \end{enumerate}

\item 基本課題
  \begin{enumerate}
  \item {\tt lu\_decomp.c}を参考にして、LU分解を用いて行列の行列式を計算するプログラムを作成せよ。$n \times n$のVandermonde行列($v_{ij}=x_j^{i-1}$) ($x_1 \cdots x_n$は実数)の行列式を計算し、厳密な値$\displaystyle \prod_{1 \le i < j \le n} (x_j-x_i)$と比較せよ。(ピボット選択で行を入れ替えると、行列式の符号が反転することに注意)
  \item LU分解を用いてDirichlet型の境界条件のもとでの二次元Laplace方程式の解を求めるプログラムを作成せよ。適当な境界条件[例えば$u(0,y) = \sin(2 \pi y)$, $u(1,y) = \sin(\pi y)$, $u(x,0)=u(x,1)=0$]や電荷分布を仮定して解を計算し、Gnuplotの{\tt splot}コマンドを用いて解をプロットせよ。また、メッシュ数を増やすと、解の形や計算時間がどのように変化するか調べよ(計算時間の測り方については、ハンドブック2.1.6節参照)
  \item 非線形連立方程式
    \begin{align*}
      & f(x,y) = x^2 + y^2 - 1 = 0 \\
      & g(x,y) = x^2(2+x) - y^2 (2-x) = 0
    \end{align*}
    を多次元のニュートン法(L1スライドp.20)を用いて解け。ヤコビ行列の逆行列をかける代わりに、LU分解で線形連立方程式を解くこと
  \end{enumerate}
  
\item 応用課題
  \begin{enumerate}
  \item C言語におけるポインタの振る舞いをテストするプログラム({\tt pointer-matrix.c})のソースコードを見て、どのような出力が生成されるか予想せよ。実際にコンパイル・実行して予想を確かめてみよ
  \item Laplace方程式の境界値問題をGauss-Seidel法、SOR法で解くプログラムを作成し、計算結果や計算速度をLU分解・Jacobi法と比較せよ。また、収束までの回数をJacobi法と比較せよ。特にSOR法の場合、パラメータ$\omega$の選び方により、どのように収束回数が変化するか観察し、最適な$\omega$の値について考察せよ
  \item 行列・行列積の計算を行うサンプルプログラム{\tt multiply.c}とBLASライブラリを使ってそれと等価な計算を行う{\tt multiply\_dgemm.c}の速度を比較せよ\footnote{aiでは、OS付属のBLAS、LAPACKではなく、Intel製のMKL (Math Kernel Library)に含まれるBLASやLAPACKを利用するのがよい。MKLを使うには、GNU Cコンパイラ({\tt cc}, {\tt gcc})の代わりにIntel Cコンパイラ({\tt icc})を使い、{\tt -llapack -lblas}の代わりに{\tt -mkl}を指定してリンクする。例: \underline{\tt icc -O3 multiply\_dgemm.c -mkl}}。行列サイズによっては数十倍もの性能差が出ことがあるが、BLASで使われている最適化手法について調べてみよ
  \end{enumerate}  
\end{itemize}

\end{document}
